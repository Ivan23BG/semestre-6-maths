\documentclass[french,a4paper,10pt]{article}
\makeatletter
%--------------------------------------------------------------------------------
\usepackage[T1]{fontenc} % font type
\usepackage[french]{babel} % language
\usepackage{lmodern} % font type
\usepackage[shortlabels]{enumitem}
\setlist[itemize,1]{label={\color{gray}\small \textbullet}} % customises itemize default -
\usepackage{fancyhdr} % customises head and foot-notes
\usepackage{centernot} % allows centering \not with \centernot
\usepackage{stmaryrd} % allows \llbracket
\usepackage[overload]{abraces} % allows \aoverbrace

\usepackage{xcolor} % colour customisation, extends to tables with {colortbl}
\definecolor{astral}{RGB}{46,116,181}
\definecolor{verdant}{RGB}{96,172,128}
\definecolor{algebraic-amber}{RGB}{255,179,102} % definition colour
\definecolor{calculus-coral}{RGB}{255,191,191} % exercice colour
\definecolor{divergent-denim}{RGB}{130,172,211} % proposition colour 
\definecolor{matrix-mist}{RGB}{204,204,204} % remark colour
\definecolor{numeric-navy}{RGB}{204,204,204} % theorem colour 
\definecolor{quadratic-quartz}{RGB}{204,153,153} % example colour 


\usepackage{latexsym}
\usepackage{amsmath}
\usepackage{amsfonts}
\usepackage{amssymb}
\usepackage{amsthm}
\usepackage{mathtools}
\usepackage{mathrsfs}
\usepackage{MnSymbol}
\usepackage{etoolbox}% http://ctan.org/pkg/etoolbox

%\usepackage{tikz}
%\usepackage{pgfplots}
%\pgfplotsset{compat=1.18}
%\usetikzlibrary{arrows}


\newtheoremstyle{gen-style}{\topsep}{\topsep}%
{}%         Body font
{}%         Indent amount (empty = no indent, \parindent = para indent)
{\sffamily\bfseries}% Thm head font
{.}%        Punctuation after thm head
{ }%     Space after thm head (\newline = linebreak)
{\thmname{#1}\thmnumber{~#2}\thmnote{~#3}}%         Thm head spec


\newtheoremstyle{no-num-style}{\topsep}{\topsep}%
{}%         Body font
{}%         Indent amount (empty = no indent, \parindent = para indent)
{\sffamily\bfseries}% Thm head font
{.}%        Punctuation after thm head
{ }%     Space after thm head (\newline = linebreak)
{\thmname{#1}}%         Thm head spec


\usepackage[]{mdframed}

\newcommand{\mytheorem}[4]{%
	\newmdtheoremenv[
	hidealllines=true,
	leftline=true,
	skipabove=0pt,
	innertopmargin=-5pt,
	innerbottommargin=2pt,
	linewidth=4pt,
	innerrightmargin=0pt,
	linecolor=#3,
	]{#1}{#2}[#4]%
}


\newcommand{\mytheoremnocount}[3]{%
	\newmdtheoremenv[
	hidealllines=true,
	leftline=true,
	skipabove=0pt,
	innertopmargin=-5pt,
	innerbottommargin=2pt,
	linewidth=4pt,
	innerrightmargin=0pt,
	linecolor=#3,
	]{#1}{#2}%
}
\newcommand{\myoctheorem}[4]{%
	\newmdtheoremenv[
	hidealllines=true,
	leftline=true,
	skipabove=0pt,
	innertopmargin=-5pt,
	innerbottommargin=2pt,
	linewidth=4pt,
	innerrightmargin=0pt,
	linecolor=#3,
	]{#1}[#4]{#2}%
}

\theoremstyle{gen-style}
\mytheorem{proposition}{Proposition}{divergent-denim}{section}
\mytheorem{propdef}{Proposition - Définition}{divergent-denim}{section}
\mytheorem{theorem}{Théorème}{quadratic-quartz}{section}
\mytheorem{lemme}{Lemme}{quadratic-quartz}{section}
\mytheorem{example}{Exemple}{quadratic-quartz}{section}
\mytheorem{remark}{Remarque}{matrix-mist}{section}
\mytheorem{notation}{Notation}{matrix-mist}{section}
\mytheorem{exercise}{Exercice}{calculus-coral}{section}
\mytheorem{exercice}{Exercice}{calculus-coral}{section}
\mytheorem{definition}{Definition}{algebraic-amber}{section}

\newcounter{oc-counter}
\myoctheorem{oc-proposition}{Proposition}{divergent-denim}{oc-counter}
\myoctheorem{oc-propdef}{Proposition - Définition}{divergent-denim}{oc-counter}
\myoctheorem{oc-theorem}{Théorème}{divergent-denim}{oc-counter}
\myoctheorem{oc-lemme}{Lemme}{quadratic-quartz}{oc-counter}
\myoctheorem{oc-example}{Exemple}{quadratic-quartz}{oc-counter}
\myoctheorem{oc-remark}{Remarque}{matrix-mist}{oc-counter}
\myoctheorem{oc-exercise}{Exercice}{calculus-coral}{oc-counter}
\myoctheorem{oc-definition}{Definition}{algebraic-amber}{oc-counter}

\theoremstyle{no-num-style}
\mytheoremnocount{td-sol}{Solution}{verdant}
\mytheoremnocount{no-num-definition}{Definition}{algebraic-amber}
\mytheoremnocount{no-num-theorem}{Théorème}{algebraic-amber}
\mytheoremnocount{oc-intro}{Introduction}{quadratic-quartz}
\mytheoremnocount{oc-proof}{Preuve}{verdant}
\mytheoremnocount{oc-young}{Formule de Taylor à l'ordre 2}{verdant}
\mytheoremnocount{oc-notation}{Notation}{matrix-mist}
\mytheorem{rappel}{Rappel}{matrix-mist}{section}
\mytheoremnocount{myproof}{Preuve}{verdant}
\mytheoremnocount{td-exo}{Exercice}{calculus-coral}
\numberwithin{oc-counter}{subsection}

%---------------
% Mise en page
%--------------

\setlength{\parindent}{0pt}

\providecommand{\defemph}[1]{{\sffamily\bfseries\color{astral}#1}}


\usepackage{sectsty}
\allsectionsfont{\color{astral}\normalfont\sffamily\bfseries}

\usepackage{mathrsfs}

%----- Easy way to redeclare math operators -----
\makeatletter
\newcommand\RedeclareMathOperator{%
	\@ifstar{\def\rmo@s{m}\rmo@redeclare}{\def\rmo@s{o}\rmo@redeclare}%
}
\newcommand\rmo@redeclare[2]{%
	\begingroup \escapechar\m@ne\xdef\@gtempa{{\string#1}}\endgroup
	\expandafter\@ifundefined\@gtempa
	{\@latex@error{\noexpand#1undefined}\@ehc}%
	\relax
	\expandafter\rmo@declmathop\rmo@s{#1}{#2}}
\newcommand\rmo@declmathop[3]{%
	\DeclareRobustCommand{#2}{\qopname\newmcodes@#1{#3}}%
}
\@onlypreamble\RedeclareMathOperator
\makeatother

\newcommand{\skipline}{\vspace{\baselineskip}}
\newcommand{\noi}{\noindent}
%------------------------------------------------


\newcommand{\adh}[1]{\mathring{#1}} %adherence
\newcommand{\badh}[1]{\mathring{\overbrace{#1}}} % big adherence
\newcommand{\norm}{\mathcal{N}} % norme
\newcommand{\ol}[1]{\overline{#1}} % overline
\newcommand{\ul}[1]{\underline{#1}} % underline
\newcommand{\sub}{\subset} % subset
\newcommand{\scr}[1]{\mathscr{#1}} % scr rapide
\newcommand{\bb}[1]{\mathbb{#1}} % bb rapide
\newcommand{\bolo}[1]{B({#1}\mathopen{}[\mathclose{}} % boule ouverte
\newcommand{\bolf}[1]{B({#1}\mathopen{}]\mathclose{}} % boule fermee
\newcommand{\act}{\circlearrowleft} % agit sur
\newcommand{\glx}[1]{\text{GL}_{#1}} % GL_x
\newcommand{\cequiv}[1]{\mathopen{}[#1\mathclose{}]} % classe d'equivalence
\newcommand{\restr}[2]{#1\mathop{}\!|_{#2}} % restriction


%----- Intervalles -----
\newcommand{\oo}[1]{\mathopen{]}#1\mathclose{[}}
\newcommand{\of}[1]{\mathopen{]}#1\mathclose{]}}
\newcommand{\fo}[1]{\mathopen{[}#1\mathclose{[}}
\newcommand{\ff}[1]{\mathopen{[}#1\mathclose{]}}



\providecommand{\1}{\mathds{1}}
\DeclareMathOperator{\im}{\mathsf{Im}}
\DeclareRobustCommand{\re}{\mathsf{Re}}
\RedeclareMathOperator{\ker}{\mathsf{Ker}}
\RedeclareMathOperator{\det}{\mathsf{det}}
\DeclareMathOperator{\vect}{\mathsf{Vect}}
\DeclareMathOperator{\orb}{\mathsf{orb}}
\DeclareMathOperator{\st}{\mathsf{st}}
\DeclareMathOperator{\aut}{\mathsf{Aut}}
\DeclareMathOperator{\bij}{\mathsf{Bij}}
\DeclareMathOperator{\rank}{\mathsf{rank}}
\DeclareMathOperator{\tr}{\mathsf{tr}}
\DeclareMathOperator{\id}{\mathsf{Id}}
\providecommand{\B}{\mathsf{B}}


\providecommand{\dpar}[2]{\frac{\partial #1}{\partial #2}}
\makeatother

\usepackage[a4paper,hmargin=30mm,vmargin=30mm]{geometry}
\title{\color{astral} \sffamily \bfseries Planche TD 2}
\author{Ivan Lejeune}
\date{\today}

\begin{document}
	
	\maketitle

    % exo 1 à rajouter

    % exo 2 à rajouter

    \begin{td-exo}[2]
        Soit $X$ un espace topologique et $A,B$ deux parties de $X$.
        \begin{enumerate}
            \item Vérifier que $A\sub B$ implique $\ol A\sub\ol B$ et $\adh A\sub \adh B$.

            \item Etablir les égalités
                \[\begin{gathered}
                    C_X(\ol A)=\badh{C_X(A)},\\
                    C_X(\adh A)=\ol{C_X(A)},\\
                    \ol{A\cup B}=\ol A\cup\ol B,\\
                    \badh{A\cap B}=\adh A\cap\adh B.
                \end{gathered}\]

            \item Etablir les inclusions
                \[\begin{gathered}
                    \ol{A\cap B}\sub\ol A\cap \ol B,\\
                    \adh A\cup \adh B\sub\badh{A\cup B}.
                \end{gathered}\]
                Puis construire des exemples où ces inclusions sont strictes.

        \end{enumerate}
    \end{td-exo}

    \begin{td-exo}[3]
        Dans un espace métrique $X$, notons $B$ la boule ouverte de centre $a$ et de rayon $r$
        et $B'$ la boule fermée correspondante.

        \begin{enumerate}
            \item Rappeler pourquoi on a toujours $\adh B=B$ et $\ol{B'}=B'$.

            \item Montrer que $B\sub \adh{B'}$ et $\ol B\sub B'$. Trouver
            un espace métrique où ces inclusions sont strictes.

            \item Montrer que si $X$ est un espace normé, les inclusions ci-dessus
            sont toujours des égalités.

        \end{enumerate}


    \end{td-exo}

    \begin{td-sol}\,
        \begin{enumerate}
            \item La première égalité est vraie car les boules ouvertes sont des ouverts.

            La deuxieme est vraie care les boules fermées sont fermées
            
            \item La première égalité est vraie car $B$ est un ouvert contenu dans $B'$ donc par définition de 
            $\adh{B'}$ on a $B\sub\adh{B'}$

            La deuxième est vraie car $\ol B\sub B'$, $B'$ est un fermé et $B\sub B'$
            donc par définition de $\ol B$, on a $\ol B\sub B'$.

            Un espace topologique où les inclusions sont strictes serait le suivant:

            Soit $X=\{a,b\}$ et $d$ la distance discrète. Alors
            \[\begin{aligned}
                B(a,1[&=\{a\}\\
                \ol{B(a,1[}&=\ol{\{a\}}=\{a\}\\
                B(a,1]&=X=\{a,b\}\\
                \badh{B(a,1]}&=X=\{a,b\}
            \end{aligned}\]

            \item Soit $x\in B(a,r]$.

            Soit $V$ un voisinage de $x$ et $\rho >0$ tel que 
                \[
                    x\in B(x,\rho[\sub V
                \]
                On peut supposer $p<r$. Alors en posant
                \[
                    \lambda\coloneq 1-\frac\rho{2r}
                    \]
                    on a
                    \[
                        a+\lambda(x-a)\in B(a,r[\cap B(x,\rho[\]
                Donc $V\cap B(a,r[\ne \varnothing$.

            % insertion schema
            
            \definecolor{zzttqq}{rgb}{0.6,0.2,0}
            \definecolor{xdxdff}{rgb}{0.49019607843137253,0.49019607843137253,1}
            \definecolor{ududff}{rgb}{0.30196078431372547,0.30196078431372547,1}
            \definecolor{uuuuuu}{rgb}{0.26666666666666666,0.26666666666666666,0.26666666666666666}
            \begin{tikzpicture}[line cap=round,line join=round,>=triangle 45,x=1cm,y=1cm]
            \begin{axis}[
            x=1cm,y=1cm,
            axis lines=middle,
            ymajorgrids=true,
            xmajorgrids=true,
            xmin=-5.5,
            xmax=5.5,
            ymin=-5.5,
            ymax=5.5,
            xtick={-5,-4,...,5},
            ytick={-5,-4,...,5},]
            \fill[line width=2pt,color=zzttqq,fill=zzttqq,fill opacity=0.10000000149011612] (0.8,2.18) -- (0.96,3.36) -- (1.78,4.02) -- (2.5,3.94) -- (3.68,2.42) -- (4.16,1.86) -- (3.9,1.42) -- (4,1) -- (2.739026131847485,1.223820186570153) -- (1.78,0.98) -- (1.06,1.42) -- cycle;
            \draw [line width=2pt] (0,0) circle (3cm);
            \draw [line width=2pt] (2.121320343559643,2.1213203435596424) circle (0.8595503872414193cm);
            \draw [line width=2pt,color=zzttqq] (0.8,2.18)-- (0.96,3.36);
            \draw [line width=2pt,color=zzttqq] (0.96,3.36)-- (1.78,4.02);
            \draw [line width=2pt,color=zzttqq] (1.78,4.02)-- (2.5,3.94);
            \draw [line width=2pt,color=zzttqq] (2.5,3.94)-- (3.68,2.42);
            \draw [line width=2pt,color=zzttqq] (3.68,2.42)-- (4.16,1.86);
            \draw [line width=2pt,color=zzttqq] (4.16,1.86)-- (3.9,1.42);
            \draw [line width=2pt,color=zzttqq] (3.9,1.42)-- (4,1);
            \draw [line width=2pt,color=zzttqq] (4,1)-- (2.739026131847485,1.223820186570153);
            \draw [line width=2pt,color=zzttqq] (2.739026131847485,1.223820186570153)-- (1.78,0.98);
            \draw [line width=2pt,color=zzttqq] (1.78,0.98)-- (1.06,1.42);
            \draw [line width=2pt,color=zzttqq] (1.06,1.42)-- (0.8,2.18);
            \draw [<->,line width=1pt] (0,0) -- (-2.1213203435596424,2.121320343559643);
            \draw [<->,line width=1pt] (2.121320343559643,2.1213203435596424) -- (1.3881139400798583,1.6727278357052275);
            \draw [<->,line width=1pt] (2.121320343559643,2.1213203435596424) -- (1.06,1.42);
            \end{axis}
            \end{tikzpicture}
            % unites a conserver :
            % a droite, r
            % fleche petite rho
            % fleche grande a+\lambda(x-a)
            % \lambda = 1-(r-p)/2
            % petit cercle = B(x,\rho[
            % gros polygone = V
            % centre petit cercle = x


            On veut choisir $\lambda<1$ tel que 
                \[
                    \frac{||x-(a+\lambda(x-a))||}{(1-\lambda)(x-a)}<\rho
                \]

            Alors $x\in \ol{B(a,r[}$ car
            \[
                x=\lim_{n\to+\infty}\left(a+\left(1-\frac1n\right)(x-a)\right)
            \]

            et cet element appartient à $B(a,r]$ puisque il est $<r$.
        \end{enumerate}
    \end{td-sol}

    \begin{td-sol}[4]
        Il faut montrer que 
        \[
            |d_A(x)-d_A(y)|\le k\cdot d(x,y)
        \]
        Cela ressemble fort à la seconde inégalité triangulaire qui nous dit:
        \[
            |d(x,a)-d(a,y)|\le d(x,y)
        \]

        Si $A=\{a\}$, ceci dit que $d_A$ est $1$-lipschitzienne. 
        Aurait-on dans le cas général
        \[
            |d_A(x)-d_A(y)|\le d(x,y)\quad\text ?
        \]

        Cela revient à prouver
        \[
            -d(x,y)\le d_A(x)-d_A(y)\le d(x,y)
        \]

        Soit $a\in A$, alors
        \[
            d(x,a)-d(y,a)\le d(x,y)
        \]
        et donc
        \[
            d(x,a)\le d(x,y)+d(y,a)
        \]
        d'où
        \[
            d_A(x)-d(x,y)\le d(y,a)
        \]
        pour tout $a\in A$ et donc la fonction est $1$-lipschitzienne.

        Dans des espaces métriques, $f$ est continue en $x\in X$ si et seulement si
        pour toute suite $(x_n)$ de points de $X$ qui convergent vers $x\in X$,
        on a $f(x_n)\to f(x)$.

    \end{td-sol}
\end{document}



