\documentclass[french,a4paper,10pt]{article}
\makeatletter
%--------------------------------------------------------------------------------
\usepackage[T1]{fontenc} % font type
\usepackage[french]{babel} % language
\usepackage{lmodern} % font type
\usepackage[shortlabels]{enumitem}
\setlist[itemize,1]{label={\color{gray}\small \textbullet}} % customises itemize default -
\usepackage{fancyhdr} % customises head and foot-notes
\usepackage{centernot} % allows centering \not with \centernot
\usepackage{stmaryrd} % allows \llbracket
\usepackage[overload]{abraces} % allows \aoverbrace

\usepackage{xcolor} % colour customisation, extends to tables with {colortbl}
\definecolor{astral}{RGB}{46,116,181}
\definecolor{verdant}{RGB}{96,172,128}
\definecolor{algebraic-amber}{RGB}{255,179,102} % definition colour
\definecolor{calculus-coral}{RGB}{255,191,191} % exercice colour
\definecolor{divergent-denim}{RGB}{130,172,211} % proposition colour 
\definecolor{matrix-mist}{RGB}{204,204,204} % remark colour
\definecolor{numeric-navy}{RGB}{204,204,204} % theorem colour 
\definecolor{quadratic-quartz}{RGB}{204,153,153} % example colour 


\usepackage{latexsym}
\usepackage{amsmath}
\usepackage{amsfonts}
\usepackage{amssymb}
\usepackage{amsthm}
\usepackage{mathtools}
\usepackage{mathrsfs}
\usepackage{MnSymbol}
\usepackage{etoolbox}% http://ctan.org/pkg/etoolbox

%\usepackage{tikz}
%\usepackage{pgfplots}
%\pgfplotsset{compat=1.18}
%\usetikzlibrary{arrows}


\newtheoremstyle{gen-style}{\topsep}{\topsep}%
{}%         Body font
{}%         Indent amount (empty = no indent, \parindent = para indent)
{\sffamily\bfseries}% Thm head font
{.}%        Punctuation after thm head
{ }%     Space after thm head (\newline = linebreak)
{\thmname{#1}\thmnumber{~#2}\thmnote{~#3}}%         Thm head spec


\newtheoremstyle{no-num-style}{\topsep}{\topsep}%
{}%         Body font
{}%         Indent amount (empty = no indent, \parindent = para indent)
{\sffamily\bfseries}% Thm head font
{.}%        Punctuation after thm head
{ }%     Space after thm head (\newline = linebreak)
{\thmname{#1}}%         Thm head spec


\usepackage[]{mdframed}

\newcommand{\mytheorem}[4]{%
	\newmdtheoremenv[
	hidealllines=true,
	leftline=true,
	skipabove=0pt,
	innertopmargin=-5pt,
	innerbottommargin=2pt,
	linewidth=4pt,
	innerrightmargin=0pt,
	linecolor=#3,
	]{#1}{#2}[#4]%
}


\newcommand{\mytheoremnocount}[3]{%
	\newmdtheoremenv[
	hidealllines=true,
	leftline=true,
	skipabove=0pt,
	innertopmargin=-5pt,
	innerbottommargin=2pt,
	linewidth=4pt,
	innerrightmargin=0pt,
	linecolor=#3,
	]{#1}{#2}%
}
\newcommand{\myoctheorem}[4]{%
	\newmdtheoremenv[
	hidealllines=true,
	leftline=true,
	skipabove=0pt,
	innertopmargin=-5pt,
	innerbottommargin=2pt,
	linewidth=4pt,
	innerrightmargin=0pt,
	linecolor=#3,
	]{#1}[#4]{#2}%
}

\theoremstyle{gen-style}
\mytheorem{proposition}{Proposition}{divergent-denim}{section}
\mytheorem{propdef}{Proposition - Définition}{divergent-denim}{section}
\mytheorem{theorem}{Théorème}{quadratic-quartz}{section}
\mytheorem{lemme}{Lemme}{quadratic-quartz}{section}
\mytheorem{example}{Exemple}{quadratic-quartz}{section}
\mytheorem{remark}{Remarque}{matrix-mist}{section}
\mytheorem{notation}{Notation}{matrix-mist}{section}
\mytheorem{exercise}{Exercice}{calculus-coral}{section}
\mytheorem{exercice}{Exercice}{calculus-coral}{section}
\mytheorem{definition}{Definition}{algebraic-amber}{section}

\newcounter{oc-counter}
\myoctheorem{oc-proposition}{Proposition}{divergent-denim}{oc-counter}
\myoctheorem{oc-propdef}{Proposition - Définition}{divergent-denim}{oc-counter}
\myoctheorem{oc-theorem}{Théorème}{divergent-denim}{oc-counter}
\myoctheorem{oc-lemme}{Lemme}{quadratic-quartz}{oc-counter}
\myoctheorem{oc-example}{Exemple}{quadratic-quartz}{oc-counter}
\myoctheorem{oc-remark}{Remarque}{matrix-mist}{oc-counter}
\myoctheorem{oc-exercise}{Exercice}{calculus-coral}{oc-counter}
\myoctheorem{oc-definition}{Definition}{algebraic-amber}{oc-counter}

\theoremstyle{no-num-style}
\mytheoremnocount{td-sol}{Solution}{verdant}
\mytheoremnocount{no-num-definition}{Definition}{algebraic-amber}
\mytheoremnocount{no-num-theorem}{Théorème}{algebraic-amber}
\mytheoremnocount{oc-intro}{Introduction}{quadratic-quartz}
\mytheoremnocount{oc-proof}{Preuve}{verdant}
\mytheoremnocount{oc-young}{Formule de Taylor à l'ordre 2}{verdant}
\mytheoremnocount{oc-notation}{Notation}{matrix-mist}
\mytheorem{rappel}{Rappel}{matrix-mist}{section}
\mytheoremnocount{myproof}{Preuve}{verdant}
\mytheoremnocount{td-exo}{Exercice}{calculus-coral}
\numberwithin{oc-counter}{subsection}

%---------------
% Mise en page
%--------------

\setlength{\parindent}{0pt}

\providecommand{\defemph}[1]{{\sffamily\bfseries\color{astral}#1}}


\usepackage{sectsty}
\allsectionsfont{\color{astral}\normalfont\sffamily\bfseries}

\usepackage{mathrsfs}

%----- Easy way to redeclare math operators -----
\makeatletter
\newcommand\RedeclareMathOperator{%
	\@ifstar{\def\rmo@s{m}\rmo@redeclare}{\def\rmo@s{o}\rmo@redeclare}%
}
\newcommand\rmo@redeclare[2]{%
	\begingroup \escapechar\m@ne\xdef\@gtempa{{\string#1}}\endgroup
	\expandafter\@ifundefined\@gtempa
	{\@latex@error{\noexpand#1undefined}\@ehc}%
	\relax
	\expandafter\rmo@declmathop\rmo@s{#1}{#2}}
\newcommand\rmo@declmathop[3]{%
	\DeclareRobustCommand{#2}{\qopname\newmcodes@#1{#3}}%
}
\@onlypreamble\RedeclareMathOperator
\makeatother

\newcommand{\skipline}{\vspace{\baselineskip}}
\newcommand{\noi}{\noindent}
%------------------------------------------------


\newcommand{\adh}[1]{\mathring{#1}} %adherence
\newcommand{\badh}[1]{\mathring{\overbrace{#1}}} % big adherence
\newcommand{\norm}{\mathcal{N}} % norme
\newcommand{\ol}[1]{\overline{#1}} % overline
\newcommand{\ul}[1]{\underline{#1}} % underline
\newcommand{\sub}{\subset} % subset
\newcommand{\scr}[1]{\mathscr{#1}} % scr rapide
\newcommand{\bb}[1]{\mathbb{#1}} % bb rapide
\newcommand{\bolo}[1]{B({#1}\mathopen{}[\mathclose{}} % boule ouverte
\newcommand{\bolf}[1]{B({#1}\mathopen{}]\mathclose{}} % boule fermee
\newcommand{\act}{\circlearrowleft} % agit sur
\newcommand{\glx}[1]{\text{GL}_{#1}} % GL_x
\newcommand{\cequiv}[1]{\mathopen{}[#1\mathclose{}]} % classe d'equivalence
\newcommand{\restr}[2]{#1\mathop{}\!|_{#2}} % restriction


%----- Intervalles -----
\newcommand{\oo}[1]{\mathopen{]}#1\mathclose{[}}
\newcommand{\of}[1]{\mathopen{]}#1\mathclose{]}}
\newcommand{\fo}[1]{\mathopen{[}#1\mathclose{[}}
\newcommand{\ff}[1]{\mathopen{[}#1\mathclose{]}}



\providecommand{\1}{\mathds{1}}
\DeclareMathOperator{\im}{\mathsf{Im}}
\DeclareRobustCommand{\re}{\mathsf{Re}}
\RedeclareMathOperator{\ker}{\mathsf{Ker}}
\RedeclareMathOperator{\det}{\mathsf{det}}
\DeclareMathOperator{\vect}{\mathsf{Vect}}
\DeclareMathOperator{\orb}{\mathsf{orb}}
\DeclareMathOperator{\st}{\mathsf{st}}
\DeclareMathOperator{\aut}{\mathsf{Aut}}
\DeclareMathOperator{\bij}{\mathsf{Bij}}
\DeclareMathOperator{\rank}{\mathsf{rank}}
\DeclareMathOperator{\tr}{\mathsf{tr}}
\DeclareMathOperator{\id}{\mathsf{Id}}
\providecommand{\B}{\mathsf{B}}


\providecommand{\dpar}[2]{\frac{\partial #1}{\partial #2}}
\makeatother

\usepackage[a4paper,hmargin=30mm,vmargin=30mm]{geometry}
\title{\color{astral} \sffamily \bfseries Planche TD 0}
\author{Ivan Lejeune\thanks{Cours inspiré de M. Charlier et M. Akrout}}
\date{\today}

\begin{document}
	
	\maketitle
	\begin{td-exo}[(Distance discrète)]
		Soit $X$ un ensemble et $\delta$ la distance discrète sur cet ensemble.
		\begin{enumerate}
			\item Vérifier que $\delta$ est une distance sur $X$.
			\item Déterminer les boules ouvertes et fermées de $(X, \delta)$. Puis déterminer la topologie $\mathcal{T}_\delta$ associée à $\delta$.
		\end{enumerate}
	\end{td-exo}
	\begin{td-sol}
		Soient $x,y,z\in X$
		\begin{enumerate}
			\item On rappelle
				\[\begin{aligned}
					\delta(x,y) = 
					\begin{cases}
						0&\text{si }x=y\\
						1&\text{si }x\ne y
					\end{cases}
				\end{aligned}\]
				\begin{enumerate}[label=$(\roman*)$]
					\item $\delta(x, y)\ge 0$ par définition
					
					\item $\delta(x,y) = 0 \Longleftrightarrow x=y$ par définition
					
					\item Si $x\ne y$ alors $y\ne x$ donc $\delta(x,y) = \delta(y,x)$ par définition
					
					\item Si $\delta(x,y) + \delta(y,z) = 0$ alors $x=y=z$ et  $\delta(x,y) + \delta(y,z) \ge \delta(x,z)$. \\Sinon $\delta(x,y) + \delta(y,z) \ge 1\ge \delta(x,z)$.
				\end{enumerate}
				$\delta$ vérifie les quatre propriétés donc c'est une distance sur $X$.
				
			\item Pour $\varepsilon=0$ on a
				\[\begin{aligned}
					\bolo{x, \varepsilon}&=\varnothing\\
					\bolf{x, \varepsilon}&=\{x\}
				\end{aligned}\]
				Pour $\varepsilon\in\oo{0, 1}$ on a
				\[\begin{aligned}
					\bolo{x, \varepsilon}&=\{x\}\\
					\bolf{x, \varepsilon}&=\{x\}
				\end{aligned}\]
				Pour $\varepsilon= 1$ on a
				\[\begin{aligned}
					\bolo{x, \varepsilon}&=\{x\}\\
					\bolf{x, \varepsilon}&=X
				\end{aligned}\]
				Pour $\varepsilon> 1$ on a
				\[\begin{aligned}
					\bolo{x, \varepsilon}&=X\\
					\bolf{x, \varepsilon}&=X
				\end{aligned}\]
				La topologie $\mathcal{T}_\delta$ associée à $\delta$ est engendrée par l'ensemble des boules ouvertes de $X$. Comme on vient de le voir, il existe une boule ouverte associée à chaque élément $x\in X$. Ainsi on a
				\[\begin{aligned}
					\mathcal{T}_\delta = \scr P(X)
				\end{aligned}\]
		\end{enumerate}
	\end{td-sol}
	\medspace
	\begin{td-exo}[(Distance et normes)]
		Soit $E$ un espace vectoriel et $\mathcal{N}$ une norme sur $E$, montrer que $d(x,y)=\mathcal{N}(y-x)$ est une distance sur $E$.
	\end{td-exo}
	\begin{td-sol}
		Soient $x,y\in E$
		\begin{enumerate}[label=$(\roman*)$]
			\item $d(x, y) = \mathcal{N}(y-x)\ge 0$ par définition d'une norme.
			
			\item $d(x,y) = 0 \Longleftrightarrow \mathcal{N}(y-x) = 0 \Longleftrightarrow x=y$ par définition d'une norme.
			
			\item $d(x,y) =\mathcal{N}(y-x)=|-1|\, \mathcal{N}(x-y)= d(y,x)$
			
			\item Soit $z\in X$, on a
				\[\begin{aligned}
					d(x,z) &= \norm(z-x)\\
					&=\norm(z-x+y-y)\\
					&=\norm((y-x)+(z-y))\\
					&\le \norm(y-x)+\norm(z-y)\\
					&\le d(x,y)+d(y,z)
				\end{aligned}\]
		\end{enumerate}
		$\delta$ vérifie les quatre propriétés donc c'est une distance sur $X$.
	\end{td-sol}
	\medspace
	\begin{td-exo}[(Normes sur $\bb R^n$)]
		On considère les normes suivantes:
		\[\begin{aligned}
			\mathcal{N}_1(x_1, \dots,x_n)&=\sum_{i=1}^n|x_i|\\
			\mathcal{N}_\infty(x_1, \dots, x_n)&=\max_{i\in\llbracket1,n\rrbracket}(|x_i|)
		\end{aligned}\]
		
		Montrer qu'elles sont des normes sur $\bb R^n$ et dessiner leurs boules unités lorsque $n=2$
	\end{td-exo}
	\begin{td-sol}
		Pour $\norm_1$ on a
		\begin{enumerate}[label=$(\roman*)$]
			\item On notera $x\coloneq(x_1,\dots,x_n)$ et $I=\{1,\dots,n\}$. On a alors
				\[\begin{aligned}
					\norm_1(x)=0&\Longrightarrow\sum_{i=1}^n|x_i| = 0\\
					&\Longrightarrow\forall i\in I,x_i=0
				\end{aligned}\] 
				Dans l'autre sens on a aussi
				\[\begin{aligned}
					\forall i\in I,x_i=0&\Longrightarrow \sum_{i=1}^n|x_i| = 0\\
					&\Longrightarrow\norm_1(x)=0
				\end{aligned}\] 
			\item On considère $\lambda x \coloneq(\lambda x_1,\dots,\lambda x_n)$ avec $\lambda>0$. On a alors
				\[\begin{aligned}
					\norm_1 (\lambda x)&=\sum_{i=1}^n|\lambda x_i|\\
					&=\sum_{i=1}^n|\lambda| |x_i|\\
					&=|\lambda| \sum_{i=1}^n|x_i|\\
					&=|\lambda| \norm_1 (x)
				\end{aligned}\]
			\item On considère $y\coloneq(y_1,\dots,y_n)$. On a alors
				\[\begin{aligned}
					\norm_1 (x+y)&=\sum_{i=1}^n|x_i+y_i|\\
					&\le \sum_{i=1}^n|x_i|+|y_i|\\
					&\le \norm_1 (x)+\norm_1(y)
				\end{aligned}\]
		\end{enumerate}
		Ainsi $\norm_1$ est bien une norme.
		
		
		Pour $\norm_\infty$ on a
		\begin{enumerate}[label=$(\roman*)$]
			\item On notera $x\coloneq(x_1,\dots,x_n)$ et $I=\{1,\dots,n\}$. On a alors
			\[\begin{aligned}
				\norm_\infty(x)=0&\Longrightarrow\max(|x_i|) = 0\\
				&\Longrightarrow\forall i\in I,x_i=0
			\end{aligned}\] 
			Dans l'autre sens on a aussi
			\[\begin{aligned}
				\forall i\in I,x_i=0&\Longrightarrow \max(|x_i|) = 0\\
				&\Longrightarrow\norm_\infty(x)=0
			\end{aligned}\] 
			\item On considère $\lambda x \coloneq(\lambda x_1,\dots,\lambda x_n)$ avec $\lambda>0$. On a alors
			\[\begin{aligned}
				\norm_\infty (\lambda x)&=\max(|\lambda x_i|)\\
				&=|\lambda| \max(|x_i|)\\
				&=|\lambda| \norm_\infty (x)
			\end{aligned}\]
			\item On considère $y\coloneq(y_1,\dots,y_n)$. On a alors
			\[\begin{aligned}
				\norm_\infty (x+y)&=\max(|x_i + y_i|)\\
				&\le\max(|x_i| + |y_i|)\\
				&\le \max(|x_i|) + \max(|y_i|)\\
				&\le \norm_\infty (x)+\norm_\infty(y)
			\end{aligned}\]
		\end{enumerate}
		Ainsi $\norm_\infty$ est bien une norme.
		
		Les boules unités associées sont : \\
		
		\begin{center}
			\begin{tikzpicture}[line cap=round,line join=round,>=triangle 45,x=1cm,y=1cm]
			\begin{axis}[
				x=1cm,y=1cm,
				axis lines=middle,
				grid style=dashed,
				ymajorgrids=true,
				xmajorgrids=true,
				xmin=-2.5,
				xmax=2.5,
				ymin=-2.5,
				ymax=2.5,
				xtick={-2,-1,...,2},
				ytick={-2,-1,...,2},]
				
				\draw [line width=1pt,color=astral] (-1,0)-- (0,1);
				\draw [line width=1pt,color=astral] (0,1)-- (1,0);
				\draw [line width=1pt,color=astral] (1,0)-- (0,-1);
				\draw [line width=1pt,color=astral] (0,-1)-- (-1,0);
				\draw [line width=1pt,color=verdant] (-1,1)-- (1,1);
				\draw [line width=1pt,color=verdant] (1,1)-- (1,-1);
				\draw [line width=1pt,color=verdant] (1,-1)-- (-1,-1);
				\draw [line width=1pt,color=verdant] (-1,-1)-- (-1,1);
				
				\draw[color=astral] (0.7, -0.7) node {$\norm_1$};
				\draw[color=verdant] (1.3, -1.2) node {$\norm_\infty$};
			\end{axis}
		\end{tikzpicture}
		\end{center}
	\end{td-sol}
	\medspace
	\begin{td-exo}[(Distance Fly Emirate)]
		Soit $(X, d)$ un espace métrique et $\text{Dubai}=D$ un point de $X$. On considère
		\[\begin{aligned}
			d_{FE}(x, y)\coloneq\begin{cases}
				0&\text{si }x=y\\
				d(x, D)+d(y, D)&\text{sinon}
			\end{cases}
		\end{aligned}\]
		
		\begin{enumerate}
			\item Montrer que $d_{FE}$ est une distance sur $X$
			\item On suppose que $(X, d)=(\bb R^2, \text{ euclidien})$ et $D=0$. Pour $x\in X$, dessiner les boules ouvertes centrées en $x$.
			\item Montrer que pour $x\ne D$, le singleton $\{x\}$ est ouvert.
		\end{enumerate}
	\end{td-exo}
	\newpage
	\begin{td-sol} \,
		\begin{enumerate}
			\item Il faut vérifier les 4 conditions d'une distance :
			\begin{enumerate}[label=$(\roman*)$]
				\item $d_{FE}(x, y)\ge 0$ car $d(x,y)\ge 0$
				\item On a $x=y\Longrightarrow d_{FE}(x, y)=0$ par définition. 
				
				Si $d_{FE}(x, y) = 0$ alors on a $x=y$ ou $d(x, D)+d(y, D) = 0$. 
				
				Comme une distance est positive, on a $d(x, D)=d(y, D)=0$ donc $x=y$.
				\item Si $x=y$, on a $d_{FE}(x, y) = d_{FE}(y, x)$. Dans le cas contraire on a
				\[\begin{aligned}
					d_{FE}(x, y) &= d_{FE}(x, D)+d_{FE}(y, D)\\
					&=d_{FE}(y, D)+d_{FE}(x, D)\\
					&= d_{FE}(y, x)
				\end{aligned}\]
				\item Soit $z\in X$. Si $x=z$, cela est évident. Dans le cas contraire on a
				\[\begin{aligned}
					d(x,z) &= \norm(z-x)\\
					&=\norm(z-x+y-y)\\
					&=\norm((y-x)+(z-y))\\
					&\le \norm(y-x)+\norm(z-y)\\
					&\le d(x,y)+d(y,z)
				\end{aligned}\]
			\end{enumerate}
			\item On fixe $x=(0,1.5)$ et on fait varier $\varepsilon\in\{1, 2, 4\}$.
			
			Pour $\varepsilon = 1$ on a $\bolo{x, \varepsilon} = \{x\}$\\
			\begin{center}
				\begin{tikzpicture}[line cap=round,line join=round,>=triangle 45,x=1cm,y=1cm]
					\begin{axis}[
						x=1cm,y=1cm,
						axis lines=middle,
						grid style=dashed,
						ymajorgrids=true,
						xmajorgrids=true,
						xmin=-3.5,
						xmax=3.5,
						ymin=-3.5,
						ymax=3.5,
						xtick={-3,-2,...,3},
						ytick={-3,-2,...,3},
						legend style={at={(1,1)}, nodes={scale=0.8}},
						]
						\draw [line width=1pt,color=astral] (0,0) circle (0.5cm);
						\addlegendimage{line width=1pt, color=astral};
						\addlegendentry{$r_1 = 2-d(x, D)$}
						
						\draw [line width=1pt,color=verdant] (0,0) circle (2.5cm);
						\addlegendimage{line width=1pt, color=verdant};
						\addlegendentry{$r_2 = 4-d(x, D)$}
						
						\draw [line width=1pt,color=verdant] (0,0)-- (2.5,0);
						\draw [line width=1pt,color=astral] (0,0)-- (0.5,0);
						\draw[color=astral] (0.2, -0.2) node {$r_1$};
						\draw[color=verdant] (1.3, 0.2) node {$r_2$};
						\addplot [mark=*, color=astral] coordinates {(0, 1.5)};
						\draw[color=astral] (0.25, 1.5) node {$x$};
						
					\end{axis}
				
				\end{tikzpicture}
			\end{center}
		\item Comme on vient de le voir, pour $x\ne D$ on peut choisir $0\le \varepsilon< d(x,D)$ et on a alors $\bolo{x, \varepsilon}=\{x\}\sub \{x\}$. Soit, que $\{x\}$ est ouvert.
		\end{enumerate}
	
	\end{td-sol}
	
	
\end{document}



