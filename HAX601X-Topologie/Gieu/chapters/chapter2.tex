\documentclass[french,a4paper,10pt]{article}
\makeatletter
%--------------------------------------------------------------------------------
\usepackage[T1]{fontenc} % font type
\usepackage[french]{babel} % language
\usepackage{lmodern} % font type
\usepackage[shortlabels]{enumitem}
\setlist[itemize,1]{label={\color{gray}\small \textbullet}} % customises itemize default -
\usepackage{fancyhdr} % customises head and foot-notes
\usepackage{centernot} % allows centering \not with \centernot
\usepackage{stmaryrd} % allows \llbracket
\usepackage[overload]{abraces} % allows \aoverbrace

\usepackage{xcolor} % colour customisation, extends to tables with {colortbl}
\definecolor{astral}{RGB}{46,116,181}
\definecolor{verdant}{RGB}{96,172,128}
\definecolor{algebraic-amber}{RGB}{255,179,102} % definition colour
\definecolor{calculus-coral}{RGB}{255,191,191} % exercice colour
\definecolor{divergent-denim}{RGB}{130,172,211} % proposition colour 
\definecolor{matrix-mist}{RGB}{204,204,204} % remark colour
\definecolor{numeric-navy}{RGB}{204,204,204} % theorem colour 
\definecolor{quadratic-quartz}{RGB}{204,153,153} % example colour 


\usepackage{latexsym}
\usepackage{amsmath}
\usepackage{amsfonts}
\usepackage{amssymb}
\usepackage{amsthm}
\usepackage{mathtools}
\usepackage{mathrsfs}
\usepackage{mnsymbol}
\usepackage{etoolbox}% http://ctan.org/pkg/etoolbox

\let\overfence\overbrace % \overfence is similar to \overbrace
\let\downfencefill\downbracefill % match components of \overbrace
\patchcmd{\overfence}{\downbracefill}{\downfencefill}{}{}% patch \overfence...
\patchcmd{\downfencefill}{\braceru \bracelu}{}{}{}%... and \downfencefill


\usepackage{tikz}
\usepackage{pgfplots}
\pgfplotsset{compat=1.18}
\usetikzlibrary{arrows}


\newtheoremstyle{gen-style}{\topsep}{\topsep}%
{}%         Body font
{}%         Indent amount (empty = no indent, \parindent = para indent)
{\sffamily\bfseries}% Thm head font
{.}%        Punctuation after thm head
{ }%     Space after thm head (\newline = linebreak)
{\thmname{#1}\thmnumber{~#2}\thmnote{~#3}}%         Thm head spec


\newtheoremstyle{no-num-style}{\topsep}{\topsep}%
{}%         Body font
{}%         Indent amount (empty = no indent, \parindent = para indent)
{\sffamily\bfseries}% Thm head font
{.}%        Punctuation after thm head
{ }%     Space after thm head (\newline = linebreak)
{\thmname{#1}}%         Thm head spec


\usepackage[]{mdframed}

\newcommand{\mytheorem}[4]{%
	\newmdtheoremenv[
	hidealllines=true,
	leftline=true,
	skipabove=0pt,
	innertopmargin=-5pt,
	innerbottommargin=2pt,
	linewidth=4pt,
	innerrightmargin=0pt,
	linecolor=#3,
	]{#1}{#2}[#4]%
}

\newcommand{\myoctheorem}[4]{%
	\newmdtheoremenv[
	hidealllines=true,
	leftline=true,
	skipabove=0pt,
	innertopmargin=-5pt,
	innerbottommargin=2pt,
	linewidth=4pt,
	innerrightmargin=0pt,
	linecolor=#3,
	]{#1}[#4]{#2}%
}

\newcommand{\mytheoremnocount}[3]{%
	\newmdtheoremenv[
	hidealllines=true,
	leftline=true,
	skipabove=0pt,
	innertopmargin=-5pt,
	innerbottommargin=2pt,
	linewidth=4pt,
	innerrightmargin=0pt,
	linecolor=#3,
	]{#1}{#2}%
}

\theoremstyle{gen-style}
\mytheorem{proposition}{Proposition}{divergent-denim}{section}
\mytheorem{propdef}{Proposition - Définition}{divergent-denim}{section}
\mytheorem{theorem}{Théorème}{quadratic-quartz}{section}
\mytheorem{lemme}{Lemme}{quadratic-quartz}{section}
\mytheorem{example}{Exemple}{quadratic-quartz}{section}
\mytheorem{remark}{Remarque}{matrix-mist}{section}
\mytheorem{notation}{Notation}{matrix-mist}{section}
\mytheorem{exercise}{Exercice}{calculus-coral}{section}
\mytheorem{exercice}{Exercice}{calculus-coral}{section}
\mytheorem{definition}{Definition}{algebraic-amber}{section}
\newcounter{oc-counter}
\myoctheorem{oc-proposition}{Proposition}{divergent-denim}{oc-counter}
\myoctheorem{oc-propdef}{Proposition - Définition}{divergent-denim}{oc-counter}
\myoctheorem{oc-theorem}{Théorème}{divergent-denim}{oc-counter}
\myoctheorem{oc-lemme}{Lemme}{quadratic-quartz}{oc-counter}
\myoctheorem{oc-example}{Exemple}{quadratic-quartz}{oc-counter}
\myoctheorem{oc-remark}{Remarque}{matrix-mist}{oc-counter}
\myoctheorem{oc-exercise}{Exercice}{calculus-coral}{oc-counter}
\myoctheorem{oc-definition}{Definition}{algebraic-amber}{oc-counter}
\theoremstyle{no-num-style}
\mytheoremnocount{td-sol}{Solution}{verdant}
\mytheoremnocount{no-num-definition}{Definition}{algebraic-amber}
\mytheoremnocount{no-num-theorem}{Théorème}{algebraic-amber}
\mytheoremnocount{oc-intro}{Introduction}{quadratic-quartz}
\mytheoremnocount{oc-proof}{Preuve}{verdant}
\mytheoremnocount{oc-young}{Formule de Taylor à l'ordre 2}{verdant}
\mytheoremnocount{oc-notation}{Notation}{matrix-mist}
\mytheorem{rappel}{Rappel}{matrix-mist}{section}
\mytheoremnocount{myproof}{Preuve}{verdant}
\mytheoremnocount{td-exo}{Exercice}{calculus-coral}
\numberwithin{oc-counter}{subsection}

%---------------
% Mise en page
%--------------

\setlength{\parindent}{0pt}

\providecommand{\defemph}[1]{{\sffamily\bfseries\color{astral}#1}}


\usepackage{sectsty}
\allsectionsfont{\color{astral}\normalfont\sffamily\bfseries}

\usepackage{mathrsfs}

%----- Easy way to redeclare math operators -----
\makeatletter
\newcommand\RedeclareMathOperator{%
	\@ifstar{\def\rmo@s{m}\rmo@redeclare}{\def\rmo@s{o}\rmo@redeclare}%
}
\newcommand\rmo@redeclare[2]{%
	\begingroup \escapechar\m@ne\xdef\@gtempa{{\string#1}}\endgroup
	\expandafter\@ifundefined\@gtempa
	{\@latex@error{\noexpand#1undefined}\@ehc}%
	\relax
	\expandafter\rmo@declmathop\rmo@s{#1}{#2}}
\newcommand\rmo@declmathop[3]{%
	\DeclareRobustCommand{#2}{\qopname\newmcodes@#1{#3}}%
}
\@onlypreamble\RedeclareMathOperator
\makeatother

\newcommand{\skipline}{\vspace{\baselineskip}}
\newcommand{\noi}{\noindent}
%------------------------------------------------


\newcommand{\adh}[1]{\mathring{#1}} %adherence
\newcommand{\badh}[1]{\mathring{\overfence{#1}}} % big adherence
\newcommand{\norm}{\mathcal{N}} % norme
\newcommand{\ol}[1]{\overline{#1}} % overline
\newcommand{\sub}{\subset} % subset
\newcommand{\scr}[1]{\mathscr{#1}} % scr rapide
\newcommand{\bb}[1]{\mathbb{#1}} % bb rapide
\newcommand{\bolo}[1]{B({#1}\mathopen{}[\mathclose{}} % boule ouverte
\newcommand{\bolf}[1]{B({#1}\mathopen{}]\mathclose{}} % boule fermee
\newcommand{\act}{\circlearrowleft} % agit sur
\newcommand{\glx}[1]{\text{GL}_{#1}} % GL_x
\newcommand{\cequiv}[1]{\mathopen{}[#1\mathclose{}]} % classe d'equivalence
\newcommand{\restr}[2]{#1\mathop{}\!|_{#2}} % restriction


%----- Intervalles -----
\newcommand{\oo}[1]{\mathopen{]}#1\mathclose{[}}
\newcommand{\of}[1]{\mathopen{]}#1\mathclose{]}}
\newcommand{\fo}[1]{\mathopen{[}#1\mathclose{[}}
\newcommand{\ff}[1]{\mathopen{[}#1\mathclose{]}}



\providecommand{\1}{\mathds{1}}
\DeclareMathOperator{\im}{\mathsf{Im}}
\DeclareRobustCommand{\re}{\mathsf{Re}}
\RedeclareMathOperator{\ker}{\mathsf{Ker}}
\RedeclareMathOperator{\det}{\mathsf{det}}
\DeclareMathOperator{\vect}{\mathsf{Vect}}
\DeclareMathOperator{\orb}{\mathsf{orb}}
\DeclareMathOperator{\st}{\mathsf{st}}
\DeclareMathOperator{\aut}{\mathsf{Aut}}
\DeclareMathOperator{\bij}{\mathsf{Bij}}
\DeclareMathOperator{\rank}{\mathsf{rank}}
\DeclareMathOperator{\tr}{\mathsf{tr}}
\DeclareMathOperator{\id}{\mathsf{Id}}
\providecommand{\B}{\mathsf{B}}


\providecommand{\dpar}[2]{\frac{\partial #1}{\partial #2}}
\makeatother

\usepackage[a4paper,hmargin=30mm,vmargin=30mm]{geometry}

\title{\color{astral} \sffamily \bfseries Chapitre 2: Espaces vectoriels normés}
\author{Ivan Lejeune\thanks{Cours inspiré de M. Charlier et M. Gieu}}
\date{\today}
% pdflatex -output-directory=output chapter2.tex && move /Y output\chapter2.pdf .\

\begin{document}
	\maketitle

	\section{Généralités}
    
    On considère $E$ un $\bb K$-espace vectoriel. $\bb K$ sera $\bb R$ ou $\bb C$.

    La dimension de $E$ sera quelconque, finie ou infinie.

    %def
    \begin{definition}
        Soit $E$ un espace vectoriel et $||\cdot ||$ de $E$ dans $\bb R$ une fonction.

        \begin{enumerate}\item ava \end{enumerate}

        On dit que $||\cdot ||$ est une \defemph{norme} sur $E$ si:

        \begin{enumerate}[label=$(\roman*)$]
            \item[(P)] $\forall x \in E, ||x|| \geq 0$
            \item[(H)] $\forall x \in E, \forall \lambda \in \bb K, ||\lambda x|| = |\lambda| \cdot ||x||$
            \item[(S)] $\forall x\in E, ||x|| = 0 \Leftrightarrow x = 0$
            \item[(T)] $\forall x, y \in E, ||x + y|| \leq ||x|| + ||y||$
        \end{enumerate}
    \end{definition}

    %prop
    \begin{proposition}[Immédiate] % seconde inég. triangulaire
        Soit $E$ un espace vectoriel normé. Alors:
        \[
            \forall x, y \in E, ||x - y|| \geq | ||x|| - ||y|| |
        \]
    \end{proposition}

    %proof
    \begin{myproof}
        \begin{align*}
            ||x|| &= ||(x - y) + y|| \\
            &\leq ||x - y|| + ||y|| \\
            \Rightarrow ||x|| - ||y|| &\leq ||x - y||
        \end{align*}
        En échangeant $x$ et $y$, on obtient l'autre inégalité:
        \[
            ||y|| - ||x|| \leq ||x - y||
        \]
    \end{myproof}
    
    Si on enlève $(S)$, on obtient une \defemph{semi-norme}.

    Cette proposition implique que $(S)$ est en fait une équivalence grâce à $(H)$:

    \[
        x=0 \Rightarrow ||x|| = ||\overrightarrow{0}|| = ||\overrightarrow{0} - \overrightarrow{0}|| \leq ||\overrightarrow{0}|| + ||\overrightarrow{0}|| = 0
    \]

    %not
    \begin{notation}
        On note $d(x,y)=||x-y||$ la \defemph{distance} sur $E$ associée à la norme $||\cdot||$.

        $(E, d)$ est un \defemph{espace métrique} et $\mathcal T_d$ est la topologie associée.

        Si $(E, d)$ est complet, on dit que $E$ est un \defemph{espace de Banach}.
    \end{notation}

    %def
    \begin{definition}
        Soit $n,n'$ deux normes sur le même espace vectoriel $E$. Elles sont dites
        \defemph{équivalentes} si:
        \[
            \exists\alpha, \beta > 0, \forall x \in E, \alpha n(x) \leq n'(x) \leq \beta n(x)
        \]

        On note alors $n \sim n'$, $n\sim n' \Rightarrow d \sim d'$ et $\mathcal T_d = \mathcal T_{d'}$.
    \end{definition}

    %ex
    \begin{example}
        \begin{enumerate}
            \item $(\bb R^n,||\cdot||)$ est un espace de Banach, de même pour $(\bb C^n,||\cdot||)$.
            \item $\left(\ell^\infty(X), ||\cdot||_\infty\right)$ est un espace de Banach.
            \item $\left(C^0(X), ||\cdot||_\infty\right)$ est un espace topologique compact, de Banach.
            \item $\left(C^0([a,b]), ||\cdot||_1\right)$ espace vectoriel normé, mais pas de Banach.
            \item $\ell^1=\{x\colon;\bb N\to\bb R \mid \sum x(n)\}$ est absolument convergent de Banach.
        \end{enumerate}
    \end{example}

    % insert reste julien

    % reste insert julien

    %proof
    \begin{myproof}
        \begin{enumerate}
            \item Cela est clair, $|\norm{x} - \norm{y}| \leq \norm{x-y}$.
            \item On a $z=(x,y)\in E\times E$, $z'=(x',y')\in E\times E$. Alors
            \[\begin{aligned}
                \norm{\sigma(z)+\sigma(z')}&=\norm{x+y-(x'+y')}\\
                &= \norm{x-x'+y-y'}\\
                &\leq \norm{x-x'}+\norm{y-y'}\\
                &\le 2 \norm{z-z'}_\infty
            \end{aligned}\]
            \item On a $z=(x,y)\in E\times E$, $z'=(x',y')\in E\times E$. Alors
            \item $t_x-t_0x_0=(t-t_0)(x-x_0)+(t-t_0)x_0+t_0(x-x_0)$.

\end{document}