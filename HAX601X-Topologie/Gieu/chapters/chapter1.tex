\documentclass[french,a4paper,10pt]{article}
\makeatletter
%--------------------------------------------------------------------------------
\usepackage[T1]{fontenc} % font type
\usepackage[french]{babel} % language
\usepackage{lmodern} % font type
\usepackage[shortlabels]{enumitem}
\setlist[itemize,1]{label={\color{gray}\small \textbullet}} % customises itemize default -
\usepackage{fancyhdr} % customises head and foot-notes
\usepackage{centernot} % allows centering \not with \centernot
\usepackage{stmaryrd} % allows \llbracket
\usepackage[overload]{abraces} % allows \aoverbrace

\usepackage{xcolor} % colour customisation, extends to tables with {colortbl}
\definecolor{astral}{RGB}{46,116,181}
\definecolor{verdant}{RGB}{96,172,128}
\definecolor{algebraic-amber}{RGB}{255,179,102} % definition colour
\definecolor{calculus-coral}{RGB}{255,191,191} % exercice colour
\definecolor{divergent-denim}{RGB}{130,172,211} % proposition colour 
\definecolor{matrix-mist}{RGB}{204,204,204} % remark colour
\definecolor{numeric-navy}{RGB}{204,204,204} % theorem colour 
\definecolor{quadratic-quartz}{RGB}{204,153,153} % example colour 


\usepackage{latexsym}
\usepackage{amsmath}
\usepackage{amsfonts}
\usepackage{amssymb}
\usepackage{amsthm}
\usepackage{mathtools}
\usepackage{mathrsfs}
\usepackage{mnsymbol}
\usepackage{etoolbox}% http://ctan.org/pkg/etoolbox

\let\overfence\overbrace % \overfence is similar to \overbrace
\let\downfencefill\downbracefill % match components of \overbrace
\patchcmd{\overfence}{\downbracefill}{\downfencefill}{}{}% patch \overfence...
\patchcmd{\downfencefill}{\braceru \bracelu}{}{}{}%... and \downfencefill


\usepackage{tikz}
\usepackage{pgfplots}
\pgfplotsset{compat=1.18}
\usetikzlibrary{arrows}


\newtheoremstyle{gen-style}{\topsep}{\topsep}%
{}%         Body font
{}%         Indent amount (empty = no indent, \parindent = para indent)
{\sffamily\bfseries}% Thm head font
{.}%        Punctuation after thm head
{ }%     Space after thm head (\newline = linebreak)
{\thmname{#1}\thmnumber{~#2}\thmnote{~#3}}%         Thm head spec


\newtheoremstyle{no-num-style}{\topsep}{\topsep}%
{}%         Body font
{}%         Indent amount (empty = no indent, \parindent = para indent)
{\sffamily\bfseries}% Thm head font
{.}%        Punctuation after thm head
{ }%     Space after thm head (\newline = linebreak)
{\thmname{#1}}%         Thm head spec


\usepackage[]{mdframed}

\newcommand{\mytheorem}[4]{%
	\newmdtheoremenv[
	hidealllines=true,
	leftline=true,
	skipabove=0pt,
	innertopmargin=-5pt,
	innerbottommargin=2pt,
	linewidth=4pt,
	innerrightmargin=0pt,
	linecolor=#3,
	]{#1}{#2}[#4]%
}

\newcommand{\myoctheorem}[4]{%
	\newmdtheoremenv[
	hidealllines=true,
	leftline=true,
	skipabove=0pt,
	innertopmargin=-5pt,
	innerbottommargin=2pt,
	linewidth=4pt,
	innerrightmargin=0pt,
	linecolor=#3,
	]{#1}[#4]{#2}%
}

\newcommand{\mytheoremnocount}[3]{%
	\newmdtheoremenv[
	hidealllines=true,
	leftline=true,
	skipabove=0pt,
	innertopmargin=-5pt,
	innerbottommargin=2pt,
	linewidth=4pt,
	innerrightmargin=0pt,
	linecolor=#3,
	]{#1}{#2}%
}

\theoremstyle{gen-style}
\mytheorem{proposition}{Proposition}{divergent-denim}{section}
\mytheorem{propdef}{Proposition - Définition}{divergent-denim}{section}
\mytheorem{theorem}{Théorème}{quadratic-quartz}{section}
\mytheorem{lemme}{Lemme}{quadratic-quartz}{section}
\mytheorem{example}{Exemple}{quadratic-quartz}{section}
\mytheorem{remark}{Remarque}{matrix-mist}{section}
\mytheorem{notation}{Notation}{matrix-mist}{section}
\mytheorem{exercise}{Exercice}{calculus-coral}{section}
\mytheorem{exercice}{Exercice}{calculus-coral}{section}
\mytheorem{definition}{Definition}{algebraic-amber}{section}
\newcounter{oc-counter}
\myoctheorem{oc-proposition}{Proposition}{divergent-denim}{oc-counter}
\myoctheorem{oc-propdef}{Proposition - Définition}{divergent-denim}{oc-counter}
\myoctheorem{oc-theorem}{Théorème}{divergent-denim}{oc-counter}
\myoctheorem{oc-lemme}{Lemme}{quadratic-quartz}{oc-counter}
\myoctheorem{oc-example}{Exemple}{quadratic-quartz}{oc-counter}
\myoctheorem{oc-remark}{Remarque}{matrix-mist}{oc-counter}
\myoctheorem{oc-exercise}{Exercice}{calculus-coral}{oc-counter}
\myoctheorem{oc-definition}{Definition}{algebraic-amber}{oc-counter}
\theoremstyle{no-num-style}
\mytheoremnocount{td-sol}{Solution}{verdant}
\mytheoremnocount{no-num-definition}{Definition}{algebraic-amber}
\mytheoremnocount{no-num-theorem}{Théorème}{algebraic-amber}
\mytheoremnocount{oc-intro}{Introduction}{quadratic-quartz}
\mytheoremnocount{oc-proof}{Preuve}{verdant}
\mytheoremnocount{oc-young}{Formule de Taylor à l'ordre 2}{verdant}
\mytheoremnocount{oc-notation}{Notation}{matrix-mist}
\mytheorem{rappel}{Rappel}{matrix-mist}{section}
\mytheoremnocount{myproof}{Preuve}{verdant}
\mytheoremnocount{td-exo}{Exercice}{calculus-coral}
\numberwithin{oc-counter}{subsection}

%---------------
% Mise en page
%--------------

\setlength{\parindent}{0pt}

\providecommand{\defemph}[1]{{\sffamily\bfseries\color{astral}#1}}


\usepackage{sectsty}
\allsectionsfont{\color{astral}\normalfont\sffamily\bfseries}

\usepackage{mathrsfs}

%----- Easy way to redeclare math operators -----
\makeatletter
\newcommand\RedeclareMathOperator{%
	\@ifstar{\def\rmo@s{m}\rmo@redeclare}{\def\rmo@s{o}\rmo@redeclare}%
}
\newcommand\rmo@redeclare[2]{%
	\begingroup \escapechar\m@ne\xdef\@gtempa{{\string#1}}\endgroup
	\expandafter\@ifundefined\@gtempa
	{\@latex@error{\noexpand#1undefined}\@ehc}%
	\relax
	\expandafter\rmo@declmathop\rmo@s{#1}{#2}}
\newcommand\rmo@declmathop[3]{%
	\DeclareRobustCommand{#2}{\qopname\newmcodes@#1{#3}}%
}
\@onlypreamble\RedeclareMathOperator
\makeatother

\newcommand{\skipline}{\vspace{\baselineskip}}
\newcommand{\noi}{\noindent}
%------------------------------------------------


\newcommand{\adh}[1]{\mathring{#1}} %adherence
\newcommand{\badh}[1]{\mathring{\overfence{#1}}} % big adherence
\newcommand{\norm}{\mathcal{N}} % norme
\newcommand{\ol}[1]{\overline{#1}} % overline
\newcommand{\sub}{\subset} % subset
\newcommand{\scr}[1]{\mathscr{#1}} % scr rapide
\newcommand{\bb}[1]{\mathbb{#1}} % bb rapide
\newcommand{\bolo}[1]{B({#1}\mathopen{}[\mathclose{}} % boule ouverte
\newcommand{\bolf}[1]{B({#1}\mathopen{}]\mathclose{}} % boule fermee
\newcommand{\act}{\circlearrowleft} % agit sur
\newcommand{\glx}[1]{\text{GL}_{#1}} % GL_x
\newcommand{\cequiv}[1]{\mathopen{}[#1\mathclose{}]} % classe d'equivalence
\newcommand{\restr}[2]{#1\mathop{}\!|_{#2}} % restriction


%----- Intervalles -----
\newcommand{\oo}[1]{\mathopen{]}#1\mathclose{[}}
\newcommand{\of}[1]{\mathopen{]}#1\mathclose{]}}
\newcommand{\fo}[1]{\mathopen{[}#1\mathclose{[}}
\newcommand{\ff}[1]{\mathopen{[}#1\mathclose{]}}



\providecommand{\1}{\mathds{1}}
\DeclareMathOperator{\im}{\mathsf{Im}}
\DeclareRobustCommand{\re}{\mathsf{Re}}
\RedeclareMathOperator{\ker}{\mathsf{Ker}}
\RedeclareMathOperator{\det}{\mathsf{det}}
\DeclareMathOperator{\vect}{\mathsf{Vect}}
\DeclareMathOperator{\orb}{\mathsf{orb}}
\DeclareMathOperator{\st}{\mathsf{st}}
\DeclareMathOperator{\aut}{\mathsf{Aut}}
\DeclareMathOperator{\bij}{\mathsf{Bij}}
\DeclareMathOperator{\rank}{\mathsf{rank}}
\DeclareMathOperator{\tr}{\mathsf{tr}}
\DeclareMathOperator{\id}{\mathsf{Id}}
\providecommand{\B}{\mathsf{B}}


\providecommand{\dpar}[2]{\frac{\partial #1}{\partial #2}}
\makeatother

\usepackage[a4paper,hmargin=30mm,vmargin=30mm]{geometry}

\title{\color{astral} \sffamily \bfseries Chapitre 1: Espaces Métriques}
\author{Ivan Lejeune\thanks{Cours inspiré de M. Charlier et M. Gieu}}
\date{\today}
% pdflatex -output-directory=output chapter1.tex && move /Y output\chapter1.pdf .\

\begin{document}
	\maketitle

	\section{Espaces métriques}
	
    %def
	\begin{definition}
        Soit $(X,d)$ un espace métrique.
        Une suite $(x_n)\subset X$ sera dite de Cauchy si:
        \[
            \forall \varepsilon>0,\ \exists N\in\bb N,\ \forall n,m\geq N,\ d(x_n,x_m)<\varepsilon.
        \]
    \end{definition}

    %rem
    \begin{remark}\,
        \begin{itemize}
            \item C'est une notion purement métrique.
            \item intuitivement: ``$x_n$ et $x_m$ se rapprochent de plus en plus
            à mesure que $n$ et $m$ deviennent grands''.
        \end{itemize}
    \end{remark}

    %sec
    \section{Un premier lot de résultats utiles}

    %prop
    \begin{proposition}\,
        \begin{enumerate}
            \item Si $d$ et $\delta$ sont deux distances fortement équivalentes sur le même ensemble $X$,
            alors $(X,d)$ et $(X,\delta)$ ont les \defemph{mêmes} suites de Cauchy.
            \item Si $f\colon (X,d)\to (Y,\delta)$ est une application uniformément continue, alors
            $f$ envoie les suites de Cauchy de $X$ sur les suites de Cauchy de $Y$.
            \item Toute suite de Cauchy est bornée.
            \item Toute suite convergente est de Cauchy.
            \item Si une suite de Cauchy admet une valeur d'adhérence, alors elle converge.
        \end{enumerate}
    \end{proposition}

    %proof
    \begin{myproof}\,
        \begin{enumerate}
            \item Il existe $\alpha,\beta>0$ tels que $\alpha d(x,y)\leq \delta(x,y)\leq \beta d(x,y)$.

            Soit $(x_n)$ une suite de Cauchy dans $(X,d)$.
            \[
                \delta(x_n,x_m)\leq \beta d(x_n,x_m)
            \]

            Soit $\varepsilon>0$. Il existe $N\in\bb N$ tel que pour $n,m\geq N$, $d(x_n,x_m)<\frac{\varepsilon}{\beta}$.

            Donc $\forall n,m\geq N$, $\delta(x_n,x_m)\leq \beta d(x_n,x_m)<\varepsilon$.

            La réciproque est analogue.

            \item Soit $(x_n)$ une suite de Cauchy dans $(X,d)$ et $f\colon (X,d)\to (Y,\delta)$ uniformément continue.

            On pose $y_n\coloneqq f(x_n)$ et on veut montrer que $(y_n)$ est de Cauchy dans $(Y,\delta)$.

            Comme $f$ est uniformément continue, on a:
            \[
                \forall \varepsilon>0,\ \exists \alpha>0,\ \forall x,x'\in X,\ d(x,x')<\alpha\Rightarrow \delta(f(x),f(x'))<\varepsilon.
            \]

            Comme $(x_n)$ est de Cauchy, on a:
            \[
                \forall \alpha>0,\ \exists N\in\bb N,\ \forall n,m\geq N,\ d(x_n,x_m)<\alpha.
            \]

            Soit $\varepsilon>0$. On veut montrer
            \[
                \delta(y_n,y_m)=\delta(f(x_n),f(x_m))<\varepsilon.
            \]

            Or $\delta(f(x_n),f(x_m))<\varepsilon$ vrai dès que $d(x_n,x_m)<\alpha$ et donc dès que $n,m\geq N$.

            \item Soit $(x_n)$ une suite de Cauchy dans $(X,d)$.
            Prenons $\varepsilon=1$. 
            Alors, il existe $N\in\bb N$ tel que pour $n,m\geq N$, $d(x_n,x_m)<1$.

            En particulier, pour $n\geq N$, $d(x_n,\underset{a}{x_N})<1$.

            Prenons $R\coloneqq \max\{d(x_0,a),\ldots,d(x_{N-1},a),1\}$.

            Alors, pour tout $n\in\bb N$, $d(x_n,a)\leq R$.

            Donc $(x_n)\subset B(a,R)$, ce qui montre que $(x_n)$ est bornée.

            \item Soit $(x_n)$ une suite convergente dans $(X,d)$.
            Alors
            \[
                d(x_n,a)\xrightarrow[n\to+\infty]{}0.
            \]

            Montrons que $(x_n)$ est de Cauchy:

            Soit $\varepsilon>0$, alors
            \[
            d(x_n,x_m)\leq \underset{<\frac\varepsilon2}{d(x_n,a)}+\underset{<\frac\varepsilon2}{d(a,x_m)}<\varepsilon
            \]
            pour $n,m$ assez grands.

            \item %lem
            \begin{lemme}[cf prochain TD]
                Soit $(x_n)$ une suite dans $(X,d)$. Les assertions suivantes sont équivalentes:
                \begin{enumerate}[label=$(\roman*)$]
                    \item $(x_n)$ admet une sous-suite convergente vers $a\in X$.
                    \item $a$ est valeur d'adhérence de $(x_n)$.
                    \item $\forall V\in \mathcal V_a,\ \forall m\in\bb N,\ \exists n\geq m,\ x_n\in V$.
                    \item $\forall \varepsilon>0,\ \forall m\in\bb N,\ \exists n\geq m,\ d(a, x_n)<\varepsilon$.
                \end{enumerate}
            \end{lemme}
            
            % retour à la preuve
            Soit $(x_n)$ une suite de Cauchy dans $(X,d)$ qui admet une valeur d'adhérence $a\in X$.
            Alors, 
            \[
                \forall \varepsilon>0,\ \forall m,p\in\bb N,\ d(x_m,x_p)<\varepsilon.
            \]

            Soit $\varepsilon>0$, on a les assertions suivantes:
            \begin{itemize}
                \item $(x_n)$ de Cauchy donc $\forall m,p\ge N,\ d(x_m,x_p)<\frac\varepsilon2$.
                \item $a$ est une valeur d'adhérence donc si on fixe $m\in\bb N,\ \exists n\geq m,\ d(a,x_n)<\frac\varepsilon2$.
            \end{itemize}

            Donc, pour $m\in\bb N$, on a
            \[
                d(a,x_m)\leq d(a, x_n) d(x_n,x_m)<\varepsilon
            \]

            Donc $\lim_{n\to+\infty}x_n=a$.
        \end{enumerate}
    \end{myproof}

    %rem
    \begin{remark}
        Il existe des suites de Cauchy qui ne convergent pas:

        On considère Héron d'Alexandrie qui a donné une suite de Cauchy qui ne converge pas:
        \[\begin{cases}
            x_0=2\\
            x_{n+1}=\frac{x_n}2+\frac1{x_n}
        \end{cases}\]

        Suite dans $\bb Q$. 

        $(x_n)$ est de Cauchy dans $\bb Q$ mais ne converge pas.

        $(x_n)$ est de Cauchy dans $\bb R$ et converge vers $\sqrt 2$.
    \end{remark}

    %def
    \begin{definition}
        Un espace métrique $(X,d)$ sera dit complet si chacune de ses suites de Cauchy converge (dans $X$).
    \end{definition}

    %ex
    \begin{example}
        $\bb Q$ n'est pas complet.
    \end{example}

    %prop
    \begin{proposition}
        $(\bb R,|\cdot|)$ est complet.
    \end{proposition}

    %proof
    \begin{myproof}
        Soit $(x_n)$ une suite de Cauchy dans $\bb R$.

        Elle est donc bornée.

        Par le théorème de Bolzano-Weierstrass, elle admet une valeur d'adhérence $a\in\bb R$.

        Donc, $(x_n)$ admet une sous-suite convergente vers $a$.

        Alors $(x_n) \subset \ff{\alpha, \beta}$ compact.
    \end{myproof}

    %sec
    \section{Complétude et Produit}

    Soient $(X,d)$ et $(Y,\delta)$ deux espaces métriques.
    On considère
    \[\begin{aligned}
        z &= (x,y)\in Z = X\times Y\\
        z' &= (x',y')\in Z
    \end{aligned}\]

    %not
    \begin{notation}
        On notera
        \[
            \mathcal D_\infty(z,z')\coloneqq \max\{d(x,x'),\delta(y,y')\}
        \]

        %exo
        \begin{exercise}
            Montrer que $\mathcal D_\infty$ est une distance sur $Z$.
        \end{exercise}

        ``la'' distance produit. 
        On note aussi $(Z, \mathcal D_\infty)$ ``l'' espace métrique produit.

        \defemph{Pourquoi les ``$\cdot$''?} On aurait pu prendre
        \[
            \mathcal D_p(z,z')\coloneqq {\left({d(x,x')}^p+{\delta(y,y')}^p\right)}^{\frac1p}
        \]
        pour $p\geq 1$. $\mathcal D_p$ est une distance équivalente à $\mathcal D_\infty$.

        Elles donnent la \defemph{même} topologie: la topologie produit.
    \end{notation}

    %prop
    \begin{proposition}
        Si $(x_n)$ une suite dans $X$,  $(y_n)$ une suite dans $Y$ et $(z_n) = (x_n,y_n)$ une suite dans $Z$.

        Alors, $(z_n)$ est de Cauchy dans $Z$ si et seulement si $(x_n)$ et $(y_n)$ sont de Cauchy dans $X$ et $Y$ respectivement.
    \end{proposition}

    %proof
    \begin{myproof}
        \begin{itemize}
            \item[$\Rightarrow$] $\mathcal D_\infty(z_n,z_m) \le \max\{d(x_n,x_m)+\delta(y_n,y_m)\}$.

            \item[$\Leftarrow$] 
            \[\begin{cases}
                \d(x_n,x_m)<\mathcal D_\infty(z_n,z_m)\\
                \delta(y_n,y_m)<\mathcal D_\infty(z_n,z_m)
            \end{cases}\]
        \end{itemize}
    \end{myproof}

    %cor
    \begin{corollaire}
        $(Z, \mathcal D_\infty)$ est complet si et seulement si $(X,d)$ et $(Y,\delta)$ sont complets.
    \end{corollaire}

    %cor
    \begin{corollaire}
        $(\bb R^n, ||\cdot||_\infty)$ est complet, et
        \[
            \left(\bb R^n,||\cdot||_\infty\right) = {\left(\bb R,|\cdot|\right)}^n
        \]
    \end{corollaire}

    %sec
    \section{Complétude et Sous-espaces}

    %not
    \begin{notation}
        On notera $(A,d_A)$ un sous-espace de $(X,d)$.
    \end{notation}

    %rem
    \begin{remark}
        $(a_n)\subset A$ est une suite de Cauchy dans $(A,d_A)$ si et seulement si $(a_n)$ est de Cauchy dans $(X,d)$.
    \end{remark}

    %prop
    \begin{proposition}
        Soit $A\subset (X,d)$ muni de la distance induite $d_A$. Alors:
        \begin{enumerate}
            \item $(A,d_A)$ complet $\Rightarrow A$ fermé dans $(X,d)$.
            \item Si $A$ est fermé dans $(X,d)$ et $(X,d)$ complet, alors $(A,d_A)$ est complet.
            \item Si $A$ est compact dans $(X,d)$, alors $(A,d_A)$ est complet.
        \end{enumerate}
    \end{proposition}

    %proof
    \begin{myproof}\,
        \begin{enumerate}
            \item Soit $x\in \overline A$. Il existe $(a_n)\subset A$ telle que $a_n\xrightarrow[n\to+\infty]{}x$.

            Comme $(a_n)$ est convergente dans $(A,d_A)$, elle est de Cauchy dans $(A,d_A)$.

            Donc, $(a_n)$ converge dans $(A,d_A)$.

            \item On suppose que $A$ est fermé dans $(X,d)$ et $(X,d)$ complet.

            Soit $(a_n)$ une suite de Cauchy dans $(A,d_A)$. Comme $(a_n)$ est de Cauchy dans $(X,d)$, elle converge dans $(X,d)$.

            $X$ complet $\Rightarrow (a_n)$ converge vers $x\in X$.

            Comme $A$ est fermé, $x\in A$.

            $(a_n)$ converge dans $(A,d_A)$. $(A,d_A)$ complet.

            \item On suppose que $A$ est compact dans $(X,d)$.

            Soit $(a_n)$ une suite de Cauchy dans $(A,d_A)$.

            Comme $A$ est compact, $(a_n)$ admet une valeur d'adhérence $x\in A$.

            Comme $A$ compact dans un séparé, $A$ est fermé. Donc $x\in A$.

            Donc $(a_n)$ converge dans $(A,d_A)$. $(A,d_A)$ complet.
        \end{enumerate}
    \end{myproof}

    %sec
    \section{Exemples fondamentaux}

    \subsection{f puissance infinie de X}

    On considère $X$ un ensemble et $f^\infty(X)\subset \mathcal F(X,\bb R)$ l'ensemble des fonctions bornées de $X$ dans $\bb R$.

    On a $||f||_\infty = \sup_{x\in X}|f(x)|$.

    %prop
    \begin{proposition}
        $(f^\infty(X), ||\cdot||_\infty)$ est complet.
    \end{proposition}

    %rem
    \begin{remark}
        Si $X=\bb N$, alors $f^\infty(X)$ est l'ensemble des suites bornées.

        Alors $|| x ||_\infty = \sup_{n\in\bb N}|x_n|$.
    \end{remark}

    Maintenant je prends $X$ un espace topologique compact.

    Si $C^0(X)=\{f\colon X\to\bb R\text{ continue}\}$, alors $C^0(X)\subset f^\infty(X)$.

    %prop
    \begin{proposition}
        $(C^0(X), ||\cdot||_\infty)$ est fermé.
    \end{proposition}

    %cor
    \begin{corollaire}
        $(C^0(X), ||\cdot||_\infty)$ est complet.
    \end{corollaire}

    Prenons maintenant $X$=$I$=$\ff{0, 2}$ compact.

    Alors $C^0(I)$ est complet.

    Prenons maintenant $||f||_1 = \int_I |f(t)| dt$.

    C'est une norme.

    %prop
    \begin{proposition}
        $(C^0(I), ||\cdot||_1)$ n'est \defemph{pas} complet.
    \end{proposition}

    On considère $C^0(I)$ avec $I=\ff{0,2}$ muni de $||\cdot||_1$.

    On considère la suite $(f_n)$ définie par
    \[
        f_n\colon\ I\to\bb R,\ t\mapsto \begin{cases}
            1&\text{si }t\in\ff{0,1}\\
            -nt+n+1 &\text{si }t\in\ff{1,1+\frac1n}\\
            0&\text{si }t\in\ff{1+\frac1n,2}
        \end{cases}
    \]

    $(f_n)$ est de Cauchy.

    \begin{tikzpicture}[line cap=round,line join=round,>=triangle 45,x=1cm,y=1cm]
        \centering
        \begin{axis}[
        x=1cm,y=1cm,
        axis lines=middle,
        ymajorgrids=true,
        xmajorgrids=true,
        xmin=-0.5,
        xmax=3.,
        ymin=-0.5,
        ymax=1.5,
        xtick={0, 1, 2, 3},
        ytick={-1, 0, 1, 2},
        ]
        \fill[line width=1pt,color=red,fill=yellow,fill opacity=0.4] (1,1) -- (1.4,0) -- (1.6,0) -- cycle;
        \draw [line width=1pt] (0,1)-- (1,1);
        \draw [line width=1pt] (1,0)-- (2.5,0);
        \draw [line width=1pt, dashed] (1,1)-- (1,0);
        \draw [line width=1pt] (1,1)-- (1.4,0);
        \draw [line width=1pt] (1,1)-- (1.6,0);
        \draw [line width=1pt,color=red] (1,1)-- (1.4,0);
        \draw [line width=1pt,color=red] (1.4,0)-- (1.6,0);
        \draw [line width=1pt,color=blue] (1.6,0)-- (1,1);
        
        \end{axis}
    \end{tikzpicture}

    \[\begin{aligned}
        || f_n - f_m ||_1 &= \int_0^2 |f_n(t)-f_m(t)|dt\\
        &= \int_0^1 |1-1|dt + \int_1^{1+\frac1n} |nt-n+1|dt + \int_{1+\frac1n}^2 |0-0|dt\\
        &= \text{ Aire sous le graphe bleu - Aire sous le graphe rouge}
        &= \text{ Aire de }T\\
        &=\frac12\left(\frac1n-\frac1m\right)
    \end{aligned}\]

    On considère $u_n = \frac1n$ de Cauchy. Alors
    \[
        ||f_n - f_m||_1 = \frac12\left(\frac1n-\frac1m\right)=\frac12|u_n-u_m|
    \]

    $(f_n)$ n'admet pas de limite dans $(C^0(I), ||\cdot||_1)$.

    Supposons le contraire : $f_n\xrightarrow[||\cdot||_1]{}f$ avec $f$ continue.

    Si $f_n\to f$ alors $d_1(f_n,f)\to 0$. Alors

    \[\begin{aligned}
        d_1(f_n,f) &= \int_0^2 |f_n(t)-f(t)|dt\\
        &= \int_0^1 |f_n(t)-f(t)|dt + \int_1^{1+\frac1n} |f_n(t)-f(t)|dt + \int_{1+\frac1n}^2 |f_n(t)-f(t)|dt\\
        &= \underbrace{\int_0^1 |1 - f(t)|dt}_{\alpha} + \beta_n + \gamma_n
    \end{aligned}\]

    Avec 
    \[
        0\le \beta_n \le \left(1+\frac1n-1\right)\left(\sup_I|f_n|+\sup_I|f|\right)\\
        0\le \beta_n \le \frac{1+\sigma}{n}\to 0
    \]
    et
    \[
        \gamma_n= F(2)-F(1+\frac1n)\to \int_1^2 |f(t)|dt
    \]

    En revenant au calcul, on a:
    \[\begin{aligned}
        d_1(f_n,f)&=\alpha+\beta_n+\gamma_n\to 0\\
        0&=\alpha+0+\int_1^2 |f(t)|dt\to 0
    \end{aligned}\]

    % insert here when merging with the other file <--- ref voir poly ivan raf ---> "ligne 440"

    % end insert

    Donc $(x_n)$ converge (puisque $X$ est complet).

    On a donc $\exists l\in X$ tel que $x_n\xrightarrow[n\to+\infty]{}l$.	

    Comme de plus, $f$ est continue (puisque contractante) alors on peut passer à la limite
    dans $x_{n+1}=f(x_n)$ et on obtient $l=f(l)$.

    Donc $l$ est un point fixe de $f$.

    Vérifions maintenant l'unicité de $l$.

    Soit $l'$ un autre point fixe de $X$. On a
    \[
        d(l,l')=d(f(l),f(l'))\leq k d(l,l')
    \]
    Alors $\underbrace{(1-k)}_{>0}d(l,l')\leq 0$ et donc $d(l,l')=0$.



\end{document}