\documentclass[french,a4paper,10pt]{article}
\makeatletter
%--------------------------------------------------------------------------------
\usepackage[T1]{fontenc} % font type
\usepackage[french]{babel} % language
\usepackage{lmodern} % font type
\usepackage[shortlabels]{enumitem}
\setlist[itemize,1]{label={\color{gray}\small \textbullet}} % customises itemize default -
\usepackage{fancyhdr} % customises head and foot-notes
\usepackage{centernot} % allows centering \not with \centernot
\usepackage{stmaryrd} % allows \llbracket
\usepackage[overload]{abraces} % allows \aoverbrace

\usepackage{xcolor} % colour customisation, extends to tables with {colortbl}
\definecolor{astral}{RGB}{46,116,181}
\definecolor{verdant}{RGB}{96,172,128}
\definecolor{algebraic-amber}{RGB}{255,179,102} % definition colour
\definecolor{calculus-coral}{RGB}{255,191,191} % exercice colour
\definecolor{divergent-denim}{RGB}{130,172,211} % proposition colour 
\definecolor{matrix-mist}{RGB}{204,204,204} % remark colour
\definecolor{numeric-navy}{RGB}{204,204,204} % theorem colour 
\definecolor{quadratic-quartz}{RGB}{204,153,153} % example colour 


\usepackage{latexsym}
\usepackage{amsmath}
\usepackage{amsfonts}
\usepackage{amssymb}
\usepackage{amsthm}
\usepackage{mathtools}
\usepackage{mathrsfs}
\usepackage{MnSymbol}
\usepackage{etoolbox}% http://ctan.org/pkg/etoolbox

%\usepackage{tikz}
%\usepackage{pgfplots}
%\pgfplotsset{compat=1.18}
%\usetikzlibrary{arrows}


\newtheoremstyle{gen-style}{\topsep}{\topsep}%
{}%         Body font
{}%         Indent amount (empty = no indent, \parindent = para indent)
{\sffamily\bfseries}% Thm head font
{.}%        Punctuation after thm head
{ }%     Space after thm head (\newline = linebreak)
{\thmname{#1}\thmnumber{~#2}\thmnote{~#3}}%         Thm head spec


\newtheoremstyle{no-num-style}{\topsep}{\topsep}%
{}%         Body font
{}%         Indent amount (empty = no indent, \parindent = para indent)
{\sffamily\bfseries}% Thm head font
{.}%        Punctuation after thm head
{ }%     Space after thm head (\newline = linebreak)
{\thmname{#1}}%         Thm head spec


\usepackage[]{mdframed}

\newcommand{\mytheorem}[4]{%
	\newmdtheoremenv[
	hidealllines=true,
	leftline=true,
	skipabove=0pt,
	innertopmargin=-5pt,
	innerbottommargin=2pt,
	linewidth=4pt,
	innerrightmargin=0pt,
	linecolor=#3,
	]{#1}{#2}[#4]%
}


\newcommand{\mytheoremnocount}[3]{%
	\newmdtheoremenv[
	hidealllines=true,
	leftline=true,
	skipabove=0pt,
	innertopmargin=-5pt,
	innerbottommargin=2pt,
	linewidth=4pt,
	innerrightmargin=0pt,
	linecolor=#3,
	]{#1}{#2}%
}
\newcommand{\myoctheorem}[4]{%
	\newmdtheoremenv[
	hidealllines=true,
	leftline=true,
	skipabove=0pt,
	innertopmargin=-5pt,
	innerbottommargin=2pt,
	linewidth=4pt,
	innerrightmargin=0pt,
	linecolor=#3,
	]{#1}[#4]{#2}%
}

\theoremstyle{gen-style}
\mytheorem{proposition}{Proposition}{divergent-denim}{section}
\mytheorem{propdef}{Proposition - Définition}{divergent-denim}{section}
\mytheorem{theorem}{Théorème}{quadratic-quartz}{section}
\mytheorem{lemme}{Lemme}{quadratic-quartz}{section}
\mytheorem{example}{Exemple}{quadratic-quartz}{section}
\mytheorem{remark}{Remarque}{matrix-mist}{section}
\mytheorem{notation}{Notation}{matrix-mist}{section}
\mytheorem{exercise}{Exercice}{calculus-coral}{section}
\mytheorem{exercice}{Exercice}{calculus-coral}{section}
\mytheorem{definition}{Definition}{algebraic-amber}{section}

\newcounter{oc-counter}
\myoctheorem{oc-proposition}{Proposition}{divergent-denim}{oc-counter}
\myoctheorem{oc-propdef}{Proposition - Définition}{divergent-denim}{oc-counter}
\myoctheorem{oc-theorem}{Théorème}{divergent-denim}{oc-counter}
\myoctheorem{oc-lemme}{Lemme}{quadratic-quartz}{oc-counter}
\myoctheorem{oc-example}{Exemple}{quadratic-quartz}{oc-counter}
\myoctheorem{oc-remark}{Remarque}{matrix-mist}{oc-counter}
\myoctheorem{oc-exercise}{Exercice}{calculus-coral}{oc-counter}
\myoctheorem{oc-definition}{Definition}{algebraic-amber}{oc-counter}

\theoremstyle{no-num-style}
\mytheoremnocount{td-sol}{Solution}{verdant}
\mytheoremnocount{no-num-definition}{Definition}{algebraic-amber}
\mytheoremnocount{no-num-theorem}{Théorème}{algebraic-amber}
\mytheoremnocount{oc-intro}{Introduction}{quadratic-quartz}
\mytheoremnocount{oc-proof}{Preuve}{verdant}
\mytheoremnocount{oc-young}{Formule de Taylor à l'ordre 2}{verdant}
\mytheoremnocount{oc-notation}{Notation}{matrix-mist}
\mytheorem{rappel}{Rappel}{matrix-mist}{section}
\mytheoremnocount{myproof}{Preuve}{verdant}
\mytheoremnocount{td-exo}{Exercice}{calculus-coral}
\numberwithin{oc-counter}{subsection}

%---------------
% Mise en page
%--------------

\setlength{\parindent}{0pt}

\providecommand{\defemph}[1]{{\sffamily\bfseries\color{astral}#1}}


\usepackage{sectsty}
\allsectionsfont{\color{astral}\normalfont\sffamily\bfseries}

\usepackage{mathrsfs}

%----- Easy way to redeclare math operators -----
\makeatletter
\newcommand\RedeclareMathOperator{%
	\@ifstar{\def\rmo@s{m}\rmo@redeclare}{\def\rmo@s{o}\rmo@redeclare}%
}
\newcommand\rmo@redeclare[2]{%
	\begingroup \escapechar\m@ne\xdef\@gtempa{{\string#1}}\endgroup
	\expandafter\@ifundefined\@gtempa
	{\@latex@error{\noexpand#1undefined}\@ehc}%
	\relax
	\expandafter\rmo@declmathop\rmo@s{#1}{#2}}
\newcommand\rmo@declmathop[3]{%
	\DeclareRobustCommand{#2}{\qopname\newmcodes@#1{#3}}%
}
\@onlypreamble\RedeclareMathOperator
\makeatother

\newcommand{\skipline}{\vspace{\baselineskip}}
\newcommand{\noi}{\noindent}
%------------------------------------------------


\newcommand{\adh}[1]{\mathring{#1}} %adherence
\newcommand{\badh}[1]{\mathring{\overbrace{#1}}} % big adherence
\newcommand{\norm}{\mathcal{N}} % norme
\newcommand{\ol}[1]{\overline{#1}} % overline
\newcommand{\ul}[1]{\underline{#1}} % underline
\newcommand{\sub}{\subset} % subset
\newcommand{\scr}[1]{\mathscr{#1}} % scr rapide
\newcommand{\bb}[1]{\mathbb{#1}} % bb rapide
\newcommand{\bolo}[1]{B({#1}\mathopen{}[\mathclose{}} % boule ouverte
\newcommand{\bolf}[1]{B({#1}\mathopen{}]\mathclose{}} % boule fermee
\newcommand{\act}{\circlearrowleft} % agit sur
\newcommand{\glx}[1]{\text{GL}_{#1}} % GL_x
\newcommand{\cequiv}[1]{\mathopen{}[#1\mathclose{}]} % classe d'equivalence
\newcommand{\restr}[2]{#1\mathop{}\!|_{#2}} % restriction


%----- Intervalles -----
\newcommand{\oo}[1]{\mathopen{]}#1\mathclose{[}}
\newcommand{\of}[1]{\mathopen{]}#1\mathclose{]}}
\newcommand{\fo}[1]{\mathopen{[}#1\mathclose{[}}
\newcommand{\ff}[1]{\mathopen{[}#1\mathclose{]}}



\providecommand{\1}{\mathds{1}}
\DeclareMathOperator{\im}{\mathsf{Im}}
\DeclareRobustCommand{\re}{\mathsf{Re}}
\RedeclareMathOperator{\ker}{\mathsf{Ker}}
\RedeclareMathOperator{\det}{\mathsf{det}}
\DeclareMathOperator{\vect}{\mathsf{Vect}}
\DeclareMathOperator{\orb}{\mathsf{orb}}
\DeclareMathOperator{\st}{\mathsf{st}}
\DeclareMathOperator{\aut}{\mathsf{Aut}}
\DeclareMathOperator{\bij}{\mathsf{Bij}}
\DeclareMathOperator{\rank}{\mathsf{rank}}
\DeclareMathOperator{\tr}{\mathsf{tr}}
\DeclareMathOperator{\id}{\mathsf{Id}}
\providecommand{\B}{\mathsf{B}}


\providecommand{\dpar}[2]{\frac{\partial #1}{\partial #2}}
\makeatother

\usepackage[a4paper,hmargin=30mm,vmargin=30mm]{geometry}

\title{\color{astral} \sffamily \bfseries Espaces métriques, notions de topologie}
\author{Ivan Lejeune\thanks{Cours inspiré de M. Charlier et M. Akrout}}
\date{\today}
% pdflatex -output-directory=output chapter2.tex && move /Y output\chapter2.pdf .\
% pdflatex -output-directory=output chapter2.tex && mv output/chapter2.pdf ./

\begin{document}
	\maketitle
	
	\section{Connexité}
	
	\begin{remark}[Idée]
		Un espace topologique est \defemph{convexe} s'il est ``en un seul morceau''
	\end{remark}
	
	\begin{example}
		$\bb R$ usuel est connexe,

		$\bb R^\ast$ n'est pas connexe, en ``deux morceaux'': $\bb R^\ast= \bb R^+ \cup \bb R^-$

		$X = \ff{0,1} \cup \ff{2,3}$ n'est pas connexe, en ``deux morceaux''

		$X = \{0\} \cup \of{0,1}$ est connexe, en ``un seul morceau'': $X=\ff{0,1}$
	\end{example}
	
	\medskip

	\subsection{Introduction}

	On a envie de dire que $X$ (espace topologique) est constitué de ``deux morceaux'' $A$ et $B$
	si 
	\[\begin{gathered}
		X=A\cup B,\\
		A\neq\emptyset, B\neq\emptyset,\\
		A\cap B=\emptyset,\\
		\overline{A}\cap B=\emptyset, A\cap\overline{B}=\emptyset
	\end{gathered}\]
	
	Si c'est la cas, alors nécessairement $A$ et $B$ sont des \defemph{fermés} de $X$:
	\[
		A\sub \ol A\sub X\setminus B=A
	\]
	Donc $A=\ol A$ et $A$ est un fermé de $X$.

	De même:
	\[
		B\sub \ol B\sub X\setminus A=B
	\]
	Donc $B=\ol B$ et $B$ est un fermé de $X$.

	$X=A\sqcup B$ est une union disjointe avec $A$ et $B$ fermés.

	Alors, 
	\[\begin{cases}
		A = X\setminus B\\
		B = X\setminus A
	\end{cases}\]
	ouverts de $X$

	\medskip

	Donc une décomposition topologique de $X$ peut se voir comme:
	\[
		X = A\sqcup (X\setminus A)
	\]
	avec $ A\ne\varnothing, A\ne{} X, A$ ouvert et fermé de $X$.

	\subsection{Espaces topologiques connexes}

	\begin{definition}
		Un espace topologique $X$ est \defemph{connexe} si $X$ et $\varnothing$ sont les seuls ouverts et fermés de $X$.
		Une partie de $ (X,\cal{T} )$ est dite \defemph{connexe} si elle est connexe pour la topologie induite par $\cal T$.
	\end{definition}

	\begin{example}
		$X$ muni de la topologie grossière est connexe.
	\end{example}

	\begin{example}
		$X$ muni de la topologie discrète est connexe si et seulement si $X$ est vide ou à un seul élément.
	\end{example}

	\begin{example}
		$\bb N, \bb Z$ muni de la topologie usuelle.
	\end{example}

	\begin{remark}
		Les propositions suivantes sont équivalentes:
		\begin{enumerate}[label=$(\roman*)$]
			\item $(X,\cal T)$ est connexe
			\item Il n'existe pas d'ouverts disjoints $A$ et $B$ de $X$ tels que $X=A\cup B$
			\item Il n'existe pas de fermés disjoints $A$ et $B$ de $X$ tels que $X=A\sqcup B$
		\end{enumerate}
	\end{remark}

	\begin{theorem}
		$\bb R$ usuel ainsi que tous ses intervalles, est connexe.
	\end{theorem}

	\begin{myproof}
		Soit $I$ un intervalle de $\bb R$.

		Si $I$ est vide, c'est évident.

		Si $I$ est non vide, on suppose que $I$ est la réunion de deux ouverts disjoints $A$ et $B$ de $I$
		avec $A\ne\varnothing, B\ne\varnothing$.

		On cherche une contradiction.

		Soit $a\in A, b\in B$.

		Sans perte de généralité, on suppose que $a<b$.

		Comme $I$ est un intervalle, on a $[a,b]\sub I$.

		% draw tikzpicture representing R with a and b and [a, b] in I
		\begin{center}
			\begin{tikzpicture}
				\draw[->] (0,0) -- (4,0) node[right] {$\bb R$};
				\draw (1,.1) -- (1,-0.1) node[below] {$a\in A$};
				\draw (3,0.1) -- (3,-0.1) node[below] {$b\in B$};
				\draw[<->] (1,0.5) -- (3,0.5) node[midway, above] {$[a,b]\sub I$};
			\end{tikzpicture}
		\end{center}

		Considérons $\{X\in \ff{a,b}\mid x\in A\}$.

		C'est une partie non vide de $\bb R$ (car $a\in A$) et majorée par $b$.

		Donc elle admet une borne supérieure $m=\sup\{x\in\ff{a,b}\mid x\in A\}$.

		Alors $m\in\ff{a,b}\sub I$ donc $m\in A\cup B$.

		\begin{itemize}
			\item Si $m\in A$, alors $A$ est voisinage de $m$ donc il existe $\varepsilon>0$ tel que $\oo{m-\varepsilon,m+\varepsilon}\sub A$.

			Ceci contredit la définition de $m$ comme borne supérieure de $\{x\in\ff{a,b}\mid x\in A\}$.
			\item Si $m\in B$, alors $B$ est voisinage de $m$ donc il existe $\varepsilon>0$ tel que $\oo{m-\varepsilon,m+\varepsilon}\sub B$.

			Ceci contredit la définition de $m$ comme borne supérieure de $\{x\in\ff{a,b}\mid x\in A\}$.
		\end{itemize}

		% draw same tikzpicture with m and epsilon in the middle
		\begin{center}
			\begin{tikzpicture}
				\draw[->] (0,0) -- (4,0) node[right] {$\bb R$};
				\draw (1,.1) -- (1,-0.1) node[below] {$a\in A$};
				\draw (3,0.1) -- (3,-0.1) node[below] {$b\in B$};
				\draw (2,0.1) -- (2,-0.1) node[below] {$m$};
				\draw[<->] (1,-0.7) -- (3,-0.7) node[midway, below] {$[a,b]\sub I$};
				\draw[<->] (1.5,0.2) -- (2.5,0.2) node[midway, above] {\tiny $\oo{m-\varepsilon,m+\varepsilon}$};
			\end{tikzpicture}
		\end{center}

		Ceci contredit l'hypothèse que $A$ et $B$ sont disjoints.

		Soit $A$ une partie de $\bb R$ qui n'est pas un intervalle.

		Alors il existe $a,a'\in A$ et $b\in\bb R\setminus A$ tels que $a<b<a'$.

		Mais alors
		\[
			A= \underbrace{A\cap\ff{-\infty,b}}_{\text{ouvert}\ne\varnothing} \sqcup \underbrace{A\cap\ff{b,+\infty}}_{\text{ouvert}\ne\varnothing}
		\]

		Donc $A$ n'est pas connexe.

	\end{myproof}

	% theorem
	\begin{theorem}
		Les parties connexes de $\bb R$ sont exactement les intervalles de $\bb R$.
	\end{theorem}

	\begin{exercise}
		$\bb Q$ usuel, est-il connexe? Non, ce n'est pas un intervalle de $\bb R$.

		Il suffit de vérifier que $\sqrt{2}\in\bb R\setminus\bb Q$ est un point de séparation.
	\end{exercise}

	%th
	\begin{theorem}
		Soit $(X,\cal T)$ un espace topologique. Alors $X$ est connexe si et seulement si toute application continue de $X$ dans $\{0,1\}$ est constante.
	\end{theorem}

	%rem
	\begin{remark}
		On considère $f\colon X\to\{0,1\}$. A quelle condition $f$ est-elle continue?

		Les ouverts de $\{0,1\}$ discret sont $\varnothing, \{0\}, \{1\}, \{0,1\}$.

		Alors 
		%gath
		\[\begin{gathered}
			f^{-1}(\varnothing) = \varnothing\\
			f^{-1}(\{0,1\}) = X
		\end{gathered}\]

		Une telle fonction est continue si et seulement si $f^{-1}(\{0\})$ et $f^{-1}(\{1\})$ sont ouverts de $X$.
	\end{remark}

	%myproof
	\begin{myproof}\,
		\begin{itemize}
			\item[$(\Rightarrow)$] Supposons que $X$ est connexe.

			Soit $f\colon X\to\{0,1\}$ continue.

			Alors $f^{-1}(\{0\})$ et $f^{-1}(\{1\})$ sont des ouverts de $X$.

			Comme $f^{-1}(\{0\})\sqcup f^{-1}(\{1\})=X$ et $X$ connexe, alors soit:
			\begin{itemize}
				\item $f^{-1}(\{0\})=\varnothing$, donc $f$ est constante égale à $1$.
				\item $f^{-1}(\{1\})=\varnothing$, donc $f$ est constante égale à $0$.
			\end{itemize}

			Donc $f$ est constante.

			\item[$(\Leftarrow)$] Supposons que toute application continue de $X$ dans $\{0,1\}$ est constante.

			Supposons que $X$ n'est pas connexe.

			Alors il existe $A$ et $B$ ouverts de $X$ tels que $X=A\sqcup B$ avec $A\ne\varnothing, B\ne\varnothing$.

			Considérons l'application caractéristique
			\[\begin{aligned}
				\chi\colon X &\to \{0,1\}\\
				x &\mapsto \begin{cases}
					0 & \text{si } x\in A\\
					1 & \text{si } x\in B
				\end{cases}
			\end{aligned}\]
			
			$\chi$ est continue et non constante, donc contradiction.
		\end{itemize}
	\end{myproof}

	%prop
	\begin{proposition}
		Soit $(X,\cal T)$ un espace topologique. Alors:
		\begin{enumerate}[label= (\arabic*)]
			\item Si $A,B$ sont deux parties connexes de $X$ et $A\cap B\ne\varnothing$, alors $A\cup B$ est connexe.
			\item Si $A,B$ sont deux parties de $X$ telles que
			%2 line display a\sub b\sub ol a and a connex
			\[\begin{aligned}
				A\sub B\sub \ol A\\
				A\text{ connexe}
			\end{aligned}\]

			Alors $B$ est connexe.

			\item En particulier:

			$A$ connexe $\Rightarrow \ol A$ connexe pour toute partie $A$ de $X$.
		\end{enumerate}

	\end{proposition}

	%myproof
	\begin{myproof}\,
		\begin{enumerate}[label= (\arabic*)]
			\item On suppose $A, B$ connexes et $A\cap B\ne\varnothing$.

			Soit $f\colon A\cup B\to\{0,1\}$ continue.

			Alors $f_{|A}$ et $f_{|B}$ sont continues.

			Or $A$ et $B$ sont connexes, donc $f_{|A}$ et $f_{|B}$ sont constantes.

			Comme $A\cap B\ne\varnothing$, alors $f_{|A}(A\cap B)=f_{|B}(A\cap B)$.

			Donc $f_{|A\cup B}$ est constante.

			Donc $A\cup B$ est connexe.

		\item On suppose $A\sub B\sub \ol A$ et $A$ connexe.

			Soit $f\colon B\to\{0,1\}$ continue.

			Alors $f_{|A}$ est continue donc constante car $A$ est connexe.

			Donc $f_{|B}$ est constante:

			Soit $x\in B$. Soit ${(x_n)}_n$ une suite de $A$ qui converge vers $x$.

			Alors $\underbrace{f(x_n)}_{\text{constante}}\to f(x)$ par continuité de $f$.

			Donc $f(x)$ est constante.

		\item On applique $(2)$ à $A$ et $\ol{A}$.
		\end{enumerate}
	\end{myproof}


\end{document}
