\documentclass[french,a4paper,10pt]{article}
\makeatletter
%--------------------------------------------------------------------------------
\usepackage[T1]{fontenc} % font type
\usepackage[french]{babel} % language
\usepackage{lmodern} % font type
\usepackage[shortlabels]{enumitem}
\setlist[itemize,1]{label={\color{gray}\small \textbullet}} % customises itemize default -
\usepackage{fancyhdr} % customises head and foot-notes
\usepackage{centernot} % allows centering \not with \centernot
\usepackage{stmaryrd} % allows \llbracket
\usepackage[overload]{abraces} % allows \aoverbrace

\usepackage{xcolor} % colour customisation, extends to tables with {colortbl}
\definecolor{astral}{RGB}{46,116,181}
\definecolor{verdant}{RGB}{96,172,128}
\definecolor{algebraic-amber}{RGB}{255,179,102} % definition colour
\definecolor{calculus-coral}{RGB}{255,191,191} % exercice colour
\definecolor{divergent-denim}{RGB}{130,172,211} % proposition colour 
\definecolor{matrix-mist}{RGB}{204,204,204} % remark colour
\definecolor{numeric-navy}{RGB}{204,204,204} % theorem colour 
\definecolor{quadratic-quartz}{RGB}{204,153,153} % example colour 


\usepackage{latexsym}
\usepackage{amsmath}
\usepackage{amsfonts}
\usepackage{amssymb}
\usepackage{amsthm}
\usepackage{mathtools}
\usepackage{mathrsfs}
\usepackage{mnsymbol}
\usepackage{etoolbox}% http://ctan.org/pkg/etoolbox

\let\overfence\overbrace % \overfence is similar to \overbrace
\let\downfencefill\downbracefill % match components of \overbrace
\patchcmd{\overfence}{\downbracefill}{\downfencefill}{}{}% patch \overfence...
\patchcmd{\downfencefill}{\braceru \bracelu}{}{}{}%... and \downfencefill


\usepackage{tikz}
\usepackage{pgfplots}
\pgfplotsset{compat=1.18}
\usetikzlibrary{arrows}


\newtheoremstyle{gen-style}{\topsep}{\topsep}%
{}%         Body font
{}%         Indent amount (empty = no indent, \parindent = para indent)
{\sffamily\bfseries}% Thm head font
{.}%        Punctuation after thm head
{ }%     Space after thm head (\newline = linebreak)
{\thmname{#1}\thmnumber{~#2}\thmnote{~#3}}%         Thm head spec


\newtheoremstyle{no-num-style}{\topsep}{\topsep}%
{}%         Body font
{}%         Indent amount (empty = no indent, \parindent = para indent)
{\sffamily\bfseries}% Thm head font
{.}%        Punctuation after thm head
{ }%     Space after thm head (\newline = linebreak)
{\thmname{#1}}%         Thm head spec


\usepackage[]{mdframed}

\newcommand{\mytheorem}[4]{%
	\newmdtheoremenv[
	hidealllines=true,
	leftline=true,
	skipabove=0pt,
	innertopmargin=-5pt,
	innerbottommargin=2pt,
	linewidth=4pt,
	innerrightmargin=0pt,
	linecolor=#3,
	]{#1}{#2}[#4]%
}

\newcommand{\myoctheorem}[4]{%
	\newmdtheoremenv[
	hidealllines=true,
	leftline=true,
	skipabove=0pt,
	innertopmargin=-5pt,
	innerbottommargin=2pt,
	linewidth=4pt,
	innerrightmargin=0pt,
	linecolor=#3,
	]{#1}[#4]{#2}%
}

\newcommand{\mytheoremnocount}[3]{%
	\newmdtheoremenv[
	hidealllines=true,
	leftline=true,
	skipabove=0pt,
	innertopmargin=-5pt,
	innerbottommargin=2pt,
	linewidth=4pt,
	innerrightmargin=0pt,
	linecolor=#3,
	]{#1}{#2}%
}

\theoremstyle{gen-style}
\mytheorem{proposition}{Proposition}{divergent-denim}{section}
\mytheorem{propdef}{Proposition - Définition}{divergent-denim}{section}
\mytheorem{theorem}{Théorème}{quadratic-quartz}{section}
\mytheorem{lemme}{Lemme}{quadratic-quartz}{section}
\mytheorem{example}{Exemple}{quadratic-quartz}{section}
\mytheorem{remark}{Remarque}{matrix-mist}{section}
\mytheorem{notation}{Notation}{matrix-mist}{section}
\mytheorem{exercise}{Exercice}{calculus-coral}{section}
\mytheorem{exercice}{Exercice}{calculus-coral}{section}
\mytheorem{definition}{Definition}{algebraic-amber}{section}
\newcounter{oc-counter}
\myoctheorem{oc-proposition}{Proposition}{divergent-denim}{oc-counter}
\myoctheorem{oc-propdef}{Proposition - Définition}{divergent-denim}{oc-counter}
\myoctheorem{oc-theorem}{Théorème}{divergent-denim}{oc-counter}
\myoctheorem{oc-lemme}{Lemme}{quadratic-quartz}{oc-counter}
\myoctheorem{oc-example}{Exemple}{quadratic-quartz}{oc-counter}
\myoctheorem{oc-remark}{Remarque}{matrix-mist}{oc-counter}
\myoctheorem{oc-exercise}{Exercice}{calculus-coral}{oc-counter}
\myoctheorem{oc-definition}{Definition}{algebraic-amber}{oc-counter}
\theoremstyle{no-num-style}
\mytheoremnocount{td-sol}{Solution}{verdant}
\mytheoremnocount{no-num-definition}{Definition}{algebraic-amber}
\mytheoremnocount{no-num-theorem}{Théorème}{algebraic-amber}
\mytheoremnocount{oc-intro}{Introduction}{quadratic-quartz}
\mytheoremnocount{oc-proof}{Preuve}{verdant}
\mytheoremnocount{oc-young}{Formule de Taylor à l'ordre 2}{verdant}
\mytheoremnocount{oc-notation}{Notation}{matrix-mist}
\mytheorem{rappel}{Rappel}{matrix-mist}{section}
\mytheoremnocount{myproof}{Preuve}{verdant}
\mytheoremnocount{td-exo}{Exercice}{calculus-coral}
\numberwithin{oc-counter}{subsection}

%---------------
% Mise en page
%--------------

\setlength{\parindent}{0pt}

\providecommand{\defemph}[1]{{\sffamily\bfseries\color{astral}#1}}


\usepackage{sectsty}
\allsectionsfont{\color{astral}\normalfont\sffamily\bfseries}

\usepackage{mathrsfs}

%----- Easy way to redeclare math operators -----
\makeatletter
\newcommand\RedeclareMathOperator{%
	\@ifstar{\def\rmo@s{m}\rmo@redeclare}{\def\rmo@s{o}\rmo@redeclare}%
}
\newcommand\rmo@redeclare[2]{%
	\begingroup \escapechar\m@ne\xdef\@gtempa{{\string#1}}\endgroup
	\expandafter\@ifundefined\@gtempa
	{\@latex@error{\noexpand#1undefined}\@ehc}%
	\relax
	\expandafter\rmo@declmathop\rmo@s{#1}{#2}}
\newcommand\rmo@declmathop[3]{%
	\DeclareRobustCommand{#2}{\qopname\newmcodes@#1{#3}}%
}
\@onlypreamble\RedeclareMathOperator
\makeatother

\newcommand{\skipline}{\vspace{\baselineskip}}
\newcommand{\noi}{\noindent}
%------------------------------------------------


\newcommand{\adh}[1]{\mathring{#1}} %adherence
\newcommand{\badh}[1]{\mathring{\overfence{#1}}} % big adherence
\newcommand{\norm}{\mathcal{N}} % norme
\newcommand{\ol}[1]{\overline{#1}} % overline
\newcommand{\sub}{\subset} % subset
\newcommand{\scr}[1]{\mathscr{#1}} % scr rapide
\newcommand{\bb}[1]{\mathbb{#1}} % bb rapide
\newcommand{\bolo}[1]{B({#1}\mathopen{}[\mathclose{}} % boule ouverte
\newcommand{\bolf}[1]{B({#1}\mathopen{}]\mathclose{}} % boule fermee
\newcommand{\act}{\circlearrowleft} % agit sur
\newcommand{\glx}[1]{\text{GL}_{#1}} % GL_x
\newcommand{\cequiv}[1]{\mathopen{}[#1\mathclose{}]} % classe d'equivalence
\newcommand{\restr}[2]{#1\mathop{}\!|_{#2}} % restriction


%----- Intervalles -----
\newcommand{\oo}[1]{\mathopen{]}#1\mathclose{[}}
\newcommand{\of}[1]{\mathopen{]}#1\mathclose{]}}
\newcommand{\fo}[1]{\mathopen{[}#1\mathclose{[}}
\newcommand{\ff}[1]{\mathopen{[}#1\mathclose{]}}



\providecommand{\1}{\mathds{1}}
\DeclareMathOperator{\im}{\mathsf{Im}}
\DeclareRobustCommand{\re}{\mathsf{Re}}
\RedeclareMathOperator{\ker}{\mathsf{Ker}}
\RedeclareMathOperator{\det}{\mathsf{det}}
\DeclareMathOperator{\vect}{\mathsf{Vect}}
\DeclareMathOperator{\orb}{\mathsf{orb}}
\DeclareMathOperator{\st}{\mathsf{st}}
\DeclareMathOperator{\aut}{\mathsf{Aut}}
\DeclareMathOperator{\bij}{\mathsf{Bij}}
\DeclareMathOperator{\rank}{\mathsf{rank}}
\DeclareMathOperator{\tr}{\mathsf{tr}}
\DeclareMathOperator{\id}{\mathsf{Id}}
\providecommand{\B}{\mathsf{B}}


\providecommand{\dpar}[2]{\frac{\partial #1}{\partial #2}}
\makeatother

\usepackage[a4paper,hmargin=30mm,vmargin=30mm]{geometry}

\title{\color{astral} \sffamily \bfseries Espaces métriques, notions de topologie}
\author{Ivan Lejeune\thanks{Cours inspiré de M. Charlier et M. Akrout}}
\date{\today}
% pdflatex -output-directory=output main.tex && move /Y output\main.pdf pdfs\

\begin{document}
	\maketitle

	\section{Espaces métriques}
	
	\begin{definition}
		Un \defemph{espace métrique} est un ensemble $X$ munit d'une application 
		\[\begin{aligned}
			d\coloneq X\times X\to \R
		\end{aligned}\]
		tel que pour tout $x,y\in X$ on a 
		\begin{enumerate}[label=$(\roman*)$]
			\item $d(x,y)\ge0$;
			\item $d(x,y) = 0 \Longleftrightarrow x=y$ (séparation);
			\item $d(x,y) = d(y,x)$ (symétrie);
			\item $\forall z\in X, d(x,y) + d(y,z) \ge d(x,z)$ (inégalité triangulaire);
		\end{enumerate}
		On appelle $d$ la \defemph{distance} (ou métrique) sur $X$.
		
	\end{definition}
	
	\medskip
	
	\begin{myexample}
		Soit $X$ un ensemble, on considère 
		\[\begin{aligned}
			\delta(x,y) = 
			\begin{cases}
				0&\text{si }x=y\\
				1&\text{si }x\ne y
			\end{cases}
		\end{aligned}\]
		C'est une distance, appelée la distance discrète, qu'on verra en TD.
	\end{myexample}
	
	\begin{definition}
		Une \defemph{norme} sur $E$ est une application 
		\[\begin{aligned}
			\mathcal N\coloneq E\to \R^+
		\end{aligned}\]
		telle que pour tout $x,y\in E, \lambda\in\R$, on a
		\begin{enumerate}[label=$(\roman*)$]
			\item $\mathcal{N}(x)=0\Longleftrightarrow x=0$;
			\item $\mathcal{N}(\lambda x)=|\lambda|\mathcal{N}(x)$;
			\item $\mathcal{N}(x +y) \le \mathcal{N}(x) + \mathcal{N}(y)$;
		\end{enumerate}
		
		Pour $E = \R\text{-ev}$, on a $(E, \mathcal{N})$ \defemph{espace vectoriel normé}.
	\end{definition}
	
	\begin{myexercice}
			Montrer que $d(x,y) = \mathcal{N}(y-x)$ est une distance sur $E$.
	\end{myexercice}
	
	\begin{myremark}
		Si $(E, \mathcal{N})$ est un evn, alors $(E, \mathcal{N})$ est un espace métrique
	\end{myremark}
	
	
	\begin{myexercice}
		Montrer que evn $\Longrightarrow$ espace métrique pour
		\begin{itemize}
			\item $(\R^n \text{ euclidiens})$
			\item $(\{\text{fonctions bornées sur }\ff{0,1}\})$
			\[\begin{aligned}
				&\rightsquigarrow||f||_\infty = \sup|f|\\
				&\rightsquigarrow||f||_p=\left(\int_0^1|f(t)|^p\det\right)^{\frac1p}\text{ pour }f\text{ continue bornée}
			\end{aligned}\]
		\end{itemize}
	\end{myexercice} 
	
	
	\begin{myexample}
		Soit $(X,d)$ un espace métrique et $A\subset X$. On a $(A, \rst{d}{A\times A})$ espace métrique.
	\end{myexample}
	
	\begin{myexercice} 
		Le montrer pour $S^2\subset\R^3$.
	\end{myexercice}
	\begin{myrappel}
		\[\begin{aligned}
			S^2=\left\{x^2+y^2+z^2 = 1\right\}
		\end{aligned}\]
	\end{myrappel}
	
	\begin{definition}
		Pour $x\in X$ et $\varepsilon\ge0$ :
		\begin{itemize}
			\item La \defemph{boule ouverte} de centre $x$ et de rayon $\varepsilon$ est
			\[\begin{aligned}
				B(x,\varepsilon\mathopen{}[\mathclose{}=\left\{y\in X, d(x,y)<\varepsilon\right\}
			\end{aligned}\]
			\item La \defemph{boule fermée} de centre $x$ et de rayon $\varepsilon$ est
			\[\begin{aligned}
				B(x,\varepsilon\mathopen{}]\mathclose{}=\left\{y\in X, d(x,y)\le\varepsilon\right\}
			\end{aligned}\]
		\end{itemize}
	\end{definition}
	
	\begin{myexample}
		Pour $X=\R$ et $d(x,y)=|x-y|$, on a
		\[\begin{aligned}
			B(x,\varepsilon\mathopen{}[\mathclose{}=\oo{x-\varepsilon, x+\varepsilon}\\
			B(x,\varepsilon\mathopen{}]\mathclose{}=\ff{x-\varepsilon, x+\varepsilon}
		\end{aligned}\]
	\end{myexample}
	
	\begin{definition}
		Soit $X$ un ensemble et $U$ une partie de $X$. Les assertions suivantes sont équivalentes :
		\begin{enumerate}[label=$(\roman*)$]
			\item $U$ est un ouvert de $X$
			\item Pour tout $x\in U$, il existe $\varepsilon>0$ tel que
			\[\begin{aligned}
				\bolo{x,\varepsilon}\sub U
			\end{aligned}\]
		\end{enumerate}
	\end{definition}
	
	
	\begin{myexample}
		Une boule ouverte est un ouvert.
	\end{myexample}
	\begin{myproof}
		Laissée en exercice.
	\end{myproof}
	
	\begin{myremark}
		Si $(X,d)$ est un espace métrique alors
		\begin{enumerate}
			\item $\varnothing$ et $X$ des ouverts;
			\item toute intersection finie d'ouverts de $X$ est un ouvert de $X$;
			\item toute union quelconque d'ouverts de $X$ est un ouvert de $X$.
		\end{enumerate}
	\end{myremark}
	
	\begin{myproof} Il suffit de vérifier les 3 propriétés:
		\begin{enumerate}
			\item On a 
			\[\begin{aligned}
				\forall x\in X, \bolo{x,1}\sub X\quad\text{et}
			\end{aligned}\]
			\[\begin{aligned}
				\forall x\in\varnothing, \text{ la propriété est toujours vrai}
			\end{aligned}\]
			Donc $(1)$ est vérifié.
			
			\item Soient $U_1,\dots,U_n$ ouverts. On pose 
			\[\begin{aligned}
				U=\bigcap_{i=1}^nU_i
			\end{aligned}\]
			Soit $x\in U$, pour tout $i\in\{1,\dots,n\}$ on a $x\in U_i$ ouvert donc il existe $\varepsilon_i>0$ tel que 
			\[\begin{aligned}
				\bolo{x,\varepsilon_i}\sub U_i
			\end{aligned}\]
			On pose $\varepsilon=\inf(\varepsilon_1,\dots,\varepsilon_n)>0$.
			
			Pour tout $i\in\{1,\dots,n\}$, on a alors
			\[\begin{aligned}
				\bolo{x,\varepsilon}\sub\bolo{x,\varepsilon_i}\sub U_i
			\end{aligned}\]
			Et donc
			\[\begin{aligned}
				\bolo{x,\varepsilon}\sub U
			\end{aligned}\]
			Soit que $U$ est ouvert, et donc $(2)$ est vérifié.
			
			\item Soient $U_1,\dots,U_n$ ouverts. On pose 
			\[\begin{aligned}
				U=\bigcup_{i=1}^nU_i
			\end{aligned}\]
			Soit $x\in U$, il existe $i\in\{1,\dots,n\}$ tel que pour $x\in U_i$ ouvert, il existe $\varepsilon_i>0$ tel que 
			\[\begin{aligned}
				\bolo{x,\varepsilon}\sub U_i\sub U
			\end{aligned}\]
			Soit que $U$ est ouvert, et donc $(3)$ est vérifié.
		\end{enumerate}
	\end{myproof}
	
	\begin{myremark}
		On note 
		\[\begin{aligned}
			\mathcal{T}_d=\left\{U\in\mathscr{P}(X), U\text{ est ouvert pour }d\right\}
		\end{aligned}\]
	\end{myremark}
	
	\section{Espaces topologiques}
	On considère $X$ un ensemble quelconque.
	
	\begin{definition}
		On dit que $\mathcal{T}\sub\scr P(X)$ est une \defemph{topologie} sur $X$ si :
		\begin{enumerate}[label=$(\roman*)$]
			\item $\varnothing\in \mathcal{T}$ et $X\in\mathcal{T}$
			
			\item $\mathcal{T}$ est stable par intersection finie
			
			\item $\mathcal{T}$ est stable par union quelconque
		\end{enumerate}
		Les éléments de $\mathcal{T}$ sont dit \defemph{ouverts}.
	\end{definition}
	
	\begin{myexample}
		Si $(X,d)$ est métrique, $\mathcal{T}_d$ est une topologie.
	\end{myexample}
	
	
	\begin{myexample}
		Pour $X$ est un ensemble, les ensembles suivants sont des topologies:
		\begin{itemize}
			\item $\mathcal{T} \coloneq \scr P(X)$ appelée topologie discrète;
			\item $\mathcal{T}\coloneq\{\varnothing, X\}$ appelée topologie grossière;
			\item $\mathcal{T}_d$, la topologie associée à la métrique $d$;
			\item Si $X=\{a,b\}$ on a aussi la topologie
			\[\begin{aligned}
				\mathcal{T}=\left\{\{a,b\}, \{a\}, \varnothing\right\}
			\end{aligned}\]
		\end{itemize}
	\end{myexample}
	
	
	
	A partir de maintenant, on considère $(X, T)$ un espace topologique avec $T$ l'ensemble des ouverts de $X$.
	
	\begin{definition}
		Une partie $F\sub X$ est dite \defemph{fermée} si $X\setminus F$ est \defemph{ouvert}
	\end{definition}
	
	\begin{myexample} 
		Pour $(X, d)$ un espace métrique, on a $\bolf{x, r}$ fermée
	\end{myexample}
	\begin{myproof}
	Laissée en exercice.
	\end{myproof}
	
	\begin{myremark}
		On n'a pas $F$ non ouvert $\Longrightarrow$ $F$ fermé.
	\end{myremark}
	\begin{myexample}
		Pour $I = \fo{0, 1}\sub\R$, on a
		\begin{enumerate}
			\item $I$ n'est pas ouvert (problème en 0)
			\item $I$ n'est pas fermé (problème en 1)
		\end{enumerate}
	\end{myexample}
	
	\begin{proposition}
		Les assertions suivantes sont vraies.
		\begin{enumerate}
			\item $\varnothing$ et $X$ sont fermés
			\item Une union finie de fermés est fermé
			\item Une intersection quelconque de fermés est fermé
		\end{enumerate}
	\end{proposition}
	\begin{myrappel}
		\[
		X\setminus\bigcap_{i\in I}A_i=\bigcup_{i\in I} X\setminus A_i
		\]
		\[
		X\setminus\bigcup_{i\in I}A_i=\bigcap_{i\in I} X\setminus A_i
		\]
	\end{myrappel}
	
	
	\begin{myremark}
		Une topologie peut-être définie à partir de ses fermés (au lieu de ses ouverts).
	\end{myremark}
	
	\subsection{Adhérence, intérieur (version topologique)}
	
	\begin{propdef}
		Soit $(X, \mathcal{T})$ un espace topologique et $A\sub X$.
		\begin{enumerate}
			\item Il existe un plus grand ouvert (au sens de l'inclusion) noté $ \overset{\circ}A$ tel que $\overset{\circ}A\sub A$
			
			$\adh A$ est appelé \defemph{intérieur} de $A$
			
			\item Il existe un plus petit fermé (au sens de l'inclusion) noté $ \ol A$ tel que $A\sub \ol A$
			
			$\ol A$ est appelé \defemph{adhérence} de $A$
		\end{enumerate}
	\end{propdef}
	
	\begin{proposition}
		On a
		\begin{enumerate}
			\item $x\in\adh A \Longleftrightarrow$ il existe $\varepsilon>0$ tel que $\bolo{x, \varepsilon}\sub A$
			
			\item $x\in\ol A\Longleftrightarrow$ pour tout $\varepsilon>0, \bolo{x,\varepsilon}\cap A\ne \varnothing$
		\end{enumerate}
	\end{proposition}
	\begin{myproof}
		Pour le $1.$, on commence par le sens direct :
		\begin{itemize}[$\Longrightarrow$]
			\item On a $x\in \adh A$ ouvert donc il existe $\varepsilon>0$ tel que
			\[\begin{aligned}
				\bolo{x,\varepsilon}\sub\adh A\underset{\text{def}}{\sub} A
			\end{aligned}\]
		\end{itemize}
		\begin{itemize}[$\Longleftarrow$]
			\item Par hypothèse, on a $\bolo{x, \varepsilon}\sub A$ un ouvert de $A$ pour $\varepsilon>0$ donc 
			\[\begin{aligned}
				x\in\bolo{x, \varepsilon}\sub\adh A
			\end{aligned}\]
		\end{itemize}
		
		Pour le $2.$ cela revient à montrer que 
		\[\begin{aligned}
			x\in\ol A &\Longleftrightarrow \exists\varepsilon>0\text{ tq }\bolo{x, \varepsilon}\sub X\setminus A\\
			x\in X\setminus\ol A&\Longleftrightarrow x\in \badh{X\setminus A}
		\end{aligned}\]
		Il suffit de montrer que 
		\[\begin{aligned}
			X\setminus \ol A = \badh{X\setminus A}
		\end{aligned}\]
		\begin{lemme}
			Soit $(X, T)$ espace topologique et $A\sub X$. Alors
			\[\begin{aligned}
				X\setminus \ol A = \badh{X\setminus A}
			\end{aligned}\]
		\end{lemme}
		\begin{myproof}
			On démontre le lemme pour démontrer la proposition précédente
			\begin{itemize}[$\subset$]
				\item $\ol A$ fermé, $X\setminus\ol A$ ouvert, $A\sub \ol A$ donc 
				\[\begin{aligned}
					X\setminus A \supset X\setminus \ol A
				\end{aligned}\]
				D'où
				\[\begin{aligned}
					X\setminus \ol A \sub \badh{X\setminus A}
				\end{aligned}\]
			\end{itemize}
			\begin{itemize}[$\supset$]
				\item $X\setminus \left(\badh{X\setminus A}\right)$ fermé donc 
				\[\begin{aligned}
					X\setminus \left(\badh{X\setminus A}\right) \supset X\setminus \left(X\setminus A\right) = A
				\end{aligned}\]
				donc 
				\[\begin{aligned}
					X\setminus \left(\badh{X\setminus A}\right) \supset \ol A = X\setminus\left(X\setminus\ol A\right)
				\end{aligned}\]
				donc 
				\[\begin{aligned}
					\badh{X\setminus A} \sub X\setminus \ol A
				\end{aligned}\]
			\end{itemize}
		\end{myproof}
		Donc $2.$ est vérifié.
	\end{myproof}

\end{document}