% txs:///compile | txs:///view-pdf-internal --embedded | cmd /c move /Y output\%.pdf pdfs\
% pdflatex -synctex=1 -interaction=nonstopmode -output-directory=output %.tex
% pdflatex -synctex=1 -output-directory=output chapter1.tex && move /Y output\chapter1.pdf .\
\documentclass[french,a4paper,10pt]{article}
\makeatletter
%--------------------------------------------------------------------------------
\usepackage[T1]{fontenc} % font type
\usepackage[french]{babel} % language
\usepackage{lmodern} % font type
\usepackage[shortlabels]{enumitem}
\setlist[itemize,1]{label={\color{gray}\small \textbullet}} % customises itemize default -
\usepackage{fancyhdr} % customises head and foot-notes
\usepackage{centernot} % allows centering \not with \centernot
\usepackage{stmaryrd} % allows \llbracket
\usepackage[overload]{abraces} % allows \aoverbrace

\usepackage{xcolor} % colour customisation, extends to tables with {colortbl}
\definecolor{astral}{RGB}{46,116,181}
\definecolor{verdant}{RGB}{96,172,128}
\definecolor{algebraic-amber}{RGB}{255,179,102} % definition colour
\definecolor{calculus-coral}{RGB}{255,191,191} % exercice colour
\definecolor{divergent-denim}{RGB}{130,172,211} % proposition colour 
\definecolor{matrix-mist}{RGB}{204,204,204} % remark colour
\definecolor{numeric-navy}{RGB}{204,204,204} % theorem colour 
\definecolor{quadratic-quartz}{RGB}{204,153,153} % example colour 


\usepackage{latexsym}
\usepackage{amsmath}
\usepackage{amsfonts}
\usepackage{amssymb}
\usepackage{amsthm}
\usepackage{mathtools}
\usepackage{mathrsfs}
\usepackage{MnSymbol}
\usepackage{etoolbox}% http://ctan.org/pkg/etoolbox

%\usepackage{tikz}
%\usepackage{pgfplots}
%\pgfplotsset{compat=1.18}
%\usetikzlibrary{arrows}


\newtheoremstyle{gen-style}{\topsep}{\topsep}%
{}%         Body font
{}%         Indent amount (empty = no indent, \parindent = para indent)
{\sffamily\bfseries}% Thm head font
{.}%        Punctuation after thm head
{ }%     Space after thm head (\newline = linebreak)
{\thmname{#1}\thmnumber{~#2}\thmnote{~#3}}%         Thm head spec


\newtheoremstyle{no-num-style}{\topsep}{\topsep}%
{}%         Body font
{}%         Indent amount (empty = no indent, \parindent = para indent)
{\sffamily\bfseries}% Thm head font
{.}%        Punctuation after thm head
{ }%     Space after thm head (\newline = linebreak)
{\thmname{#1}}%         Thm head spec


\usepackage[]{mdframed}

\newcommand{\mytheorem}[4]{%
	\newmdtheoremenv[
	hidealllines=true,
	leftline=true,
	skipabove=0pt,
	innertopmargin=-5pt,
	innerbottommargin=2pt,
	linewidth=4pt,
	innerrightmargin=0pt,
	linecolor=#3,
	]{#1}{#2}[#4]%
}


\newcommand{\mytheoremnocount}[3]{%
	\newmdtheoremenv[
	hidealllines=true,
	leftline=true,
	skipabove=0pt,
	innertopmargin=-5pt,
	innerbottommargin=2pt,
	linewidth=4pt,
	innerrightmargin=0pt,
	linecolor=#3,
	]{#1}{#2}%
}
\newcommand{\myoctheorem}[4]{%
	\newmdtheoremenv[
	hidealllines=true,
	leftline=true,
	skipabove=0pt,
	innertopmargin=-5pt,
	innerbottommargin=2pt,
	linewidth=4pt,
	innerrightmargin=0pt,
	linecolor=#3,
	]{#1}[#4]{#2}%
}

\theoremstyle{gen-style}
\mytheorem{proposition}{Proposition}{divergent-denim}{section}
\mytheorem{propdef}{Proposition - Définition}{divergent-denim}{section}
\mytheorem{theorem}{Théorème}{quadratic-quartz}{section}
\mytheorem{lemme}{Lemme}{quadratic-quartz}{section}
\mytheorem{example}{Exemple}{quadratic-quartz}{section}
\mytheorem{remark}{Remarque}{matrix-mist}{section}
\mytheorem{notation}{Notation}{matrix-mist}{section}
\mytheorem{exercise}{Exercice}{calculus-coral}{section}
\mytheorem{exercice}{Exercice}{calculus-coral}{section}
\mytheorem{definition}{Definition}{algebraic-amber}{section}

\newcounter{oc-counter}
\myoctheorem{oc-proposition}{Proposition}{divergent-denim}{oc-counter}
\myoctheorem{oc-propdef}{Proposition - Définition}{divergent-denim}{oc-counter}
\myoctheorem{oc-theorem}{Théorème}{divergent-denim}{oc-counter}
\myoctheorem{oc-lemme}{Lemme}{quadratic-quartz}{oc-counter}
\myoctheorem{oc-example}{Exemple}{quadratic-quartz}{oc-counter}
\myoctheorem{oc-remark}{Remarque}{matrix-mist}{oc-counter}
\myoctheorem{oc-exercise}{Exercice}{calculus-coral}{oc-counter}
\myoctheorem{oc-definition}{Definition}{algebraic-amber}{oc-counter}

\theoremstyle{no-num-style}
\mytheoremnocount{td-sol}{Solution}{verdant}
\mytheoremnocount{no-num-definition}{Definition}{algebraic-amber}
\mytheoremnocount{no-num-theorem}{Théorème}{algebraic-amber}
\mytheoremnocount{oc-intro}{Introduction}{quadratic-quartz}
\mytheoremnocount{oc-proof}{Preuve}{verdant}
\mytheoremnocount{oc-young}{Formule de Taylor à l'ordre 2}{verdant}
\mytheoremnocount{oc-notation}{Notation}{matrix-mist}
\mytheorem{rappel}{Rappel}{matrix-mist}{section}
\mytheoremnocount{myproof}{Preuve}{verdant}
\mytheoremnocount{td-exo}{Exercice}{calculus-coral}
\numberwithin{oc-counter}{subsection}

%---------------
% Mise en page
%--------------

\setlength{\parindent}{0pt}

\providecommand{\defemph}[1]{{\sffamily\bfseries\color{astral}#1}}


\usepackage{sectsty}
\allsectionsfont{\color{astral}\normalfont\sffamily\bfseries}

\usepackage{mathrsfs}

%----- Easy way to redeclare math operators -----
\makeatletter
\newcommand\RedeclareMathOperator{%
	\@ifstar{\def\rmo@s{m}\rmo@redeclare}{\def\rmo@s{o}\rmo@redeclare}%
}
\newcommand\rmo@redeclare[2]{%
	\begingroup \escapechar\m@ne\xdef\@gtempa{{\string#1}}\endgroup
	\expandafter\@ifundefined\@gtempa
	{\@latex@error{\noexpand#1undefined}\@ehc}%
	\relax
	\expandafter\rmo@declmathop\rmo@s{#1}{#2}}
\newcommand\rmo@declmathop[3]{%
	\DeclareRobustCommand{#2}{\qopname\newmcodes@#1{#3}}%
}
\@onlypreamble\RedeclareMathOperator
\makeatother

\newcommand{\skipline}{\vspace{\baselineskip}}
\newcommand{\noi}{\noindent}
%------------------------------------------------


\newcommand{\adh}[1]{\mathring{#1}} %adherence
\newcommand{\badh}[1]{\mathring{\overbrace{#1}}} % big adherence
\newcommand{\norm}{\mathcal{N}} % norme
\newcommand{\ol}[1]{\overline{#1}} % overline
\newcommand{\ul}[1]{\underline{#1}} % underline
\newcommand{\sub}{\subset} % subset
\newcommand{\scr}[1]{\mathscr{#1}} % scr rapide
\newcommand{\bb}[1]{\mathbb{#1}} % bb rapide
\newcommand{\bolo}[1]{B({#1}\mathopen{}[\mathclose{}} % boule ouverte
\newcommand{\bolf}[1]{B({#1}\mathopen{}]\mathclose{}} % boule fermee
\newcommand{\act}{\circlearrowleft} % agit sur
\newcommand{\glx}[1]{\text{GL}_{#1}} % GL_x
\newcommand{\cequiv}[1]{\mathopen{}[#1\mathclose{}]} % classe d'equivalence
\newcommand{\restr}[2]{#1\mathop{}\!|_{#2}} % restriction


%----- Intervalles -----
\newcommand{\oo}[1]{\mathopen{]}#1\mathclose{[}}
\newcommand{\of}[1]{\mathopen{]}#1\mathclose{]}}
\newcommand{\fo}[1]{\mathopen{[}#1\mathclose{[}}
\newcommand{\ff}[1]{\mathopen{[}#1\mathclose{]}}



\providecommand{\1}{\mathds{1}}
\DeclareMathOperator{\im}{\mathsf{Im}}
\DeclareRobustCommand{\re}{\mathsf{Re}}
\RedeclareMathOperator{\ker}{\mathsf{Ker}}
\RedeclareMathOperator{\det}{\mathsf{det}}
\DeclareMathOperator{\vect}{\mathsf{Vect}}
\DeclareMathOperator{\orb}{\mathsf{orb}}
\DeclareMathOperator{\st}{\mathsf{st}}
\DeclareMathOperator{\aut}{\mathsf{Aut}}
\DeclareMathOperator{\bij}{\mathsf{Bij}}
\DeclareMathOperator{\rank}{\mathsf{rank}}
\DeclareMathOperator{\tr}{\mathsf{tr}}
\DeclareMathOperator{\id}{\mathsf{Id}}
\providecommand{\B}{\mathsf{B}}


\providecommand{\dpar}[2]{\frac{\partial #1}{\partial #2}}
\makeatother

\usepackage[a4paper,hmargin=30mm,vmargin=30mm]{geometry}
\title{\color{astral} \sffamily \bfseries Compléments - chapitre 1}
\author{Ivan Lejeune\thanks{Cours inspiré de M. Marche}}
\date{\today}

\begin{document}
	
	\maketitle
	\section{Chapitre 1 - Optimisation convexe}
	\section{Compléments chapitre 1}
	On considère la fonction :
		\[\begin{aligned}
			f\colon &\bb R^n&\to&\bb R\\
			&x&\mapsto &\frac 12(Ax,x)-(b,x)=0
		\end{aligned}\]
	où
		\[\begin{aligned}
			A\in S_n(\bb R),b\in\bb R^n,c\in\bb R
		\end{aligned}\]
	On considère le problème :
		\[\begin{aligned}
			\inf_{x\in\bb R^n}f(x)
		\end{aligned}\]
		
	\begin{oc-proposition}
		On a 
		\begin{center}
			$f'(x)=Ax-b$ et $f''(x)=A$.
		\end{center}
	\end{oc-proposition}
	\begin{myproof}
		Il suffit de calculer
			\[\begin{aligned}
				f(x+h)-f(x)&=\frac12(A(x+h),x+h)-(b,x+h)-\frac12(Ax,x)+(b,x)\\
				&=\frac12(Ax,h)+\frac12(Ah,x)+\frac12(Ah,h)-(b,h)\\
				&=(Ax-b,h)+\frac12(Ah,h)
			\end{aligned}\]
			
		et d'identifier ces termes avec le développement de Taylor :
			\[\begin{aligned}
				f(x+h)-f(x)=f'(x)\cdot h+\frac12 f''(x)\cdot(h,h)
			\end{aligned}\] 
	\end{myproof}
	
	\begin{oc-proposition}
		Coercivité de $f$
		
		$f$ est coercive si et seulement si $A$ est définie positive
	\end{oc-proposition}
	
	\begin{myproof}
		$A$ est symétrique réelle donc $A$ est diagonalisable dans une base $\perp$.
		
		Ainsi, il existe $P\in O_n(\bb R)$ telle que
			\[\begin{aligned}
				A=~^tPDP
			\end{aligned}\]
			
		Soit $h\in\bb R^n$, et posons $u=Ph$ (changement de base).
		
		On a ainsi
		\[\begin{aligned}
			(Ah,h)=\sum_{i=1}^n\lambda_iu_i^2
		\end{aligned}\]
		où $\lambda_1\le\dots\le \lambda_n$ est le spectre de $A$.
		
		Ainsi 
			\[\begin{aligned}
				(Ah,h)\ge\lambda_1\sum_{i=1}^n u_i^2&=\lambda_1\nn u^2\\
				&=\lambda_1\nn h^2
			\end{aligned}\]
		car la base est orthonormée.
		
		Ainsi $f$ est coercive si et seulement si $A$ est définie positive.
	\end{myproof}
	
	\begin{oc-proposition}
		$f$ est convexe si et seulement si $A$ est positive
		
		$f$ est strictement convexe si et seulement si $A$ est définie positive
	\end{oc-proposition}
	\begin{myproof}
		C'est une conséquence directe de la proposition 1 et des résultats 
	\end{myproof}
	
	\begin{oc-proposition}
		On suppose que $A$ est positive. Alors $f$ admet un minimum global si et seulement si il existe $\ul x\in\bb R^n$ tel que 
			\[\begin{aligned}
				A\ul x=b
			\end{aligned}\]
		ou encore si et seulement si $b\in\im A$.
	\end{oc-proposition}
	
	\begin{myproof}
		C'est une conséquence de la proposition 1.4.2
	\end{myproof}
	
	\begin{oc-remark}
		En dimension finie, on a
			\[\begin{aligned}
				\im A=\left(\ker~^tA\right)^\perp=\left(\ker A\right)^\perp
			\end{aligned}\]
		Soit $x\in\ker~^tA$, on a $~^tAx=0$, donc $\forall y\in\bb R^n, \left(~^tAx,y\right)=0$.
		
		D'où $(x,Ay)=0$ c'est à dire $x\in(\im A)^\perp$ ainsi
			\[\begin{aligned}
				\ker~^tA\sub (\im A)^\perp
			\end{aligned}\]
		De plus, on a
			\[\begin{aligned}
				\dim(\ker)\cdots
			\end{aligned}\]
	\end{oc-remark}
	
	\begin{oc-remark}
		On peut distinguer plusieurs cas :
		\begin{itemize}
			\item si $\lambda_1<0$ alors $f$ n'est pas bornée inférieurement
				\[\begin{aligned}
					\forall z\in\bb R,f(ze_i)=\frac{\lambda_1}2z^2-z(b,e_i)+c\underset{z\to+\infty}\to-\infty
				\end{aligned}\]
				
			\item si $\lambda_1=0$ et si $b\notin (\ker A)^\perp$, alors l'équation $f'(x)=0$ n'a pas de solution
			
			$f$ est convexe, mais non bornée inférieurement. En effet il existe $e_i\in\ker A$ tel que $(b,e_i)\ne 0$. On en déduit
				\[\begin{aligned}
					\forall z\in\bb R,f(z,e_i)=-z(b,e_i)+c\underset{z\to\text{sign}(b,e_i)\times\infty}\to-\infty
				\end{aligned}\]
				
			\item si $\lambda_1=0$ et $b\in(\ker A)^\perp$ alors l'équation $f'(x)=0$ possède une infinité de solutions. En effet, puisque $A$ est positive, $f$ est convexe et tout minimum local est global.
			
			Si $x_0$ est une solution particulière de $f'(x)=0$, l'ensemble des solutions est l'espace affine
				\[\begin{aligned}
					x_0+\ker A
				\end{aligned}\]
				
			et 
				\[\begin{aligned}
					\min_{\bb R^n}f(x)=-\frac12(b,x_0)+c
				\end{aligned}\]
			\item Si $\lambda_1>0$, $A$ est définie positive, l'équation $f'(x)=0$ admet une solution donnée par $A^{-1}b$.
			
			$f$ est strictement convexe et
				\[\begin{aligned}
					\min_{\bb R^n}f(x)=-\frac12(b,A^{-1}b)+c
				\end{aligned}\]
		\end{itemize}
	\end{oc-remark}
		
\end{document}
