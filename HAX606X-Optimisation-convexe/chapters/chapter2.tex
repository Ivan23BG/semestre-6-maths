% txs:///compile | txs:///view-pdf-internal --embedded | cmd /c move /Y output\%.pdf pdfs\
% pdflatex -synctex=1 -interaction=nonstopmode -output-directory=output %.tex
% pdflatex -synctex=1 -output-directory=output chapter1.tex && move /Y output\chapter1.pdf .\
\documentclass[french,a4paper,10pt]{article}
\makeatletter
%--------------------------------------------------------------------------------
\usepackage[T1]{fontenc} % font type
\usepackage[french]{babel} % language
\usepackage{lmodern} % font type
\usepackage[shortlabels]{enumitem}
\setlist[itemize,1]{label={\color{gray}\small \textbullet}} % customises itemize default -
\usepackage{fancyhdr} % customises head and foot-notes
\usepackage{centernot} % allows centering \not with \centernot
\usepackage{stmaryrd} % allows \llbracket
\usepackage[overload]{abraces} % allows \aoverbrace

\usepackage{xcolor} % colour customisation, extends to tables with {colortbl}
\definecolor{astral}{RGB}{46,116,181}
\definecolor{verdant}{RGB}{96,172,128}
\definecolor{algebraic-amber}{RGB}{255,179,102} % definition colour
\definecolor{calculus-coral}{RGB}{255,191,191} % exercice colour
\definecolor{divergent-denim}{RGB}{130,172,211} % proposition colour 
\definecolor{matrix-mist}{RGB}{204,204,204} % remark colour
\definecolor{numeric-navy}{RGB}{204,204,204} % theorem colour 
\definecolor{quadratic-quartz}{RGB}{204,153,153} % example colour 


\usepackage{latexsym}
\usepackage{amsmath}
\usepackage{amsfonts}
\usepackage{amssymb}
\usepackage{amsthm}
\usepackage{mathtools}
\usepackage{mathrsfs}
\usepackage{mnsymbol}
\usepackage{etoolbox}% http://ctan.org/pkg/etoolbox

\let\overfence\overbrace % \overfence is similar to \overbrace
\let\downfencefill\downbracefill % match components of \overbrace
\patchcmd{\overfence}{\downbracefill}{\downfencefill}{}{}% patch \overfence...
\patchcmd{\downfencefill}{\braceru \bracelu}{}{}{}%... and \downfencefill


\usepackage{tikz}
\usepackage{pgfplots}
\pgfplotsset{compat=1.18}
\usetikzlibrary{arrows}


\newtheoremstyle{gen-style}{\topsep}{\topsep}%
{}%         Body font
{}%         Indent amount (empty = no indent, \parindent = para indent)
{\sffamily\bfseries}% Thm head font
{.}%        Punctuation after thm head
{ }%     Space after thm head (\newline = linebreak)
{\thmname{#1}\thmnumber{~#2}\thmnote{~#3}}%         Thm head spec


\newtheoremstyle{no-num-style}{\topsep}{\topsep}%
{}%         Body font
{}%         Indent amount (empty = no indent, \parindent = para indent)
{\sffamily\bfseries}% Thm head font
{.}%        Punctuation after thm head
{ }%     Space after thm head (\newline = linebreak)
{\thmname{#1}}%         Thm head spec


\usepackage[]{mdframed}

\newcommand{\mytheorem}[4]{%
	\newmdtheoremenv[
	hidealllines=true,
	leftline=true,
	skipabove=0pt,
	innertopmargin=-5pt,
	innerbottommargin=2pt,
	linewidth=4pt,
	innerrightmargin=0pt,
	linecolor=#3,
	]{#1}{#2}[#4]%
}

\newcommand{\myoctheorem}[4]{%
	\newmdtheoremenv[
	hidealllines=true,
	leftline=true,
	skipabove=0pt,
	innertopmargin=-5pt,
	innerbottommargin=2pt,
	linewidth=4pt,
	innerrightmargin=0pt,
	linecolor=#3,
	]{#1}[#4]{#2}%
}

\newcommand{\mytheoremnocount}[3]{%
	\newmdtheoremenv[
	hidealllines=true,
	leftline=true,
	skipabove=0pt,
	innertopmargin=-5pt,
	innerbottommargin=2pt,
	linewidth=4pt,
	innerrightmargin=0pt,
	linecolor=#3,
	]{#1}{#2}%
}

\theoremstyle{gen-style}
\mytheorem{proposition}{Proposition}{divergent-denim}{section}
\mytheorem{propdef}{Proposition - Définition}{divergent-denim}{section}
\mytheorem{theorem}{Théorème}{quadratic-quartz}{section}
\mytheorem{lemme}{Lemme}{quadratic-quartz}{section}
\mytheorem{example}{Exemple}{quadratic-quartz}{section}
\mytheorem{remark}{Remarque}{matrix-mist}{section}
\mytheorem{notation}{Notation}{matrix-mist}{section}
\mytheorem{exercise}{Exercice}{calculus-coral}{section}
\mytheorem{exercice}{Exercice}{calculus-coral}{section}
\mytheorem{definition}{Definition}{algebraic-amber}{section}
\newcounter{oc-counter}
\myoctheorem{oc-proposition}{Proposition}{divergent-denim}{oc-counter}
\myoctheorem{oc-propdef}{Proposition - Définition}{divergent-denim}{oc-counter}
\myoctheorem{oc-theorem}{Théorème}{divergent-denim}{oc-counter}
\myoctheorem{oc-lemme}{Lemme}{quadratic-quartz}{oc-counter}
\myoctheorem{oc-example}{Exemple}{quadratic-quartz}{oc-counter}
\myoctheorem{oc-remark}{Remarque}{matrix-mist}{oc-counter}
\myoctheorem{oc-exercise}{Exercice}{calculus-coral}{oc-counter}
\myoctheorem{oc-definition}{Definition}{algebraic-amber}{oc-counter}
\theoremstyle{no-num-style}
\mytheoremnocount{td-sol}{Solution}{verdant}
\mytheoremnocount{no-num-definition}{Definition}{algebraic-amber}
\mytheoremnocount{no-num-theorem}{Théorème}{algebraic-amber}
\mytheoremnocount{oc-intro}{Introduction}{quadratic-quartz}
\mytheoremnocount{oc-proof}{Preuve}{verdant}
\mytheoremnocount{oc-young}{Formule de Taylor à l'ordre 2}{verdant}
\mytheoremnocount{oc-notation}{Notation}{matrix-mist}
\mytheorem{rappel}{Rappel}{matrix-mist}{section}
\mytheoremnocount{myproof}{Preuve}{verdant}
\mytheoremnocount{td-exo}{Exercice}{calculus-coral}
\numberwithin{oc-counter}{subsection}

%---------------
% Mise en page
%--------------

\setlength{\parindent}{0pt}

\providecommand{\defemph}[1]{{\sffamily\bfseries\color{astral}#1}}


\usepackage{sectsty}
\allsectionsfont{\color{astral}\normalfont\sffamily\bfseries}

\usepackage{mathrsfs}

%----- Easy way to redeclare math operators -----
\makeatletter
\newcommand\RedeclareMathOperator{%
	\@ifstar{\def\rmo@s{m}\rmo@redeclare}{\def\rmo@s{o}\rmo@redeclare}%
}
\newcommand\rmo@redeclare[2]{%
	\begingroup \escapechar\m@ne\xdef\@gtempa{{\string#1}}\endgroup
	\expandafter\@ifundefined\@gtempa
	{\@latex@error{\noexpand#1undefined}\@ehc}%
	\relax
	\expandafter\rmo@declmathop\rmo@s{#1}{#2}}
\newcommand\rmo@declmathop[3]{%
	\DeclareRobustCommand{#2}{\qopname\newmcodes@#1{#3}}%
}
\@onlypreamble\RedeclareMathOperator
\makeatother

\newcommand{\skipline}{\vspace{\baselineskip}}
\newcommand{\noi}{\noindent}
%------------------------------------------------


\newcommand{\adh}[1]{\mathring{#1}} %adherence
\newcommand{\badh}[1]{\mathring{\overfence{#1}}} % big adherence
\newcommand{\norm}{\mathcal{N}} % norme
\newcommand{\ol}[1]{\overline{#1}} % overline
\newcommand{\sub}{\subset} % subset
\newcommand{\scr}[1]{\mathscr{#1}} % scr rapide
\newcommand{\bb}[1]{\mathbb{#1}} % bb rapide
\newcommand{\bolo}[1]{B({#1}\mathopen{}[\mathclose{}} % boule ouverte
\newcommand{\bolf}[1]{B({#1}\mathopen{}]\mathclose{}} % boule fermee
\newcommand{\act}{\circlearrowleft} % agit sur
\newcommand{\glx}[1]{\text{GL}_{#1}} % GL_x
\newcommand{\cequiv}[1]{\mathopen{}[#1\mathclose{}]} % classe d'equivalence
\newcommand{\restr}[2]{#1\mathop{}\!|_{#2}} % restriction


%----- Intervalles -----
\newcommand{\oo}[1]{\mathopen{]}#1\mathclose{[}}
\newcommand{\of}[1]{\mathopen{]}#1\mathclose{]}}
\newcommand{\fo}[1]{\mathopen{[}#1\mathclose{[}}
\newcommand{\ff}[1]{\mathopen{[}#1\mathclose{]}}



\providecommand{\1}{\mathds{1}}
\DeclareMathOperator{\im}{\mathsf{Im}}
\DeclareRobustCommand{\re}{\mathsf{Re}}
\RedeclareMathOperator{\ker}{\mathsf{Ker}}
\RedeclareMathOperator{\det}{\mathsf{det}}
\DeclareMathOperator{\vect}{\mathsf{Vect}}
\DeclareMathOperator{\orb}{\mathsf{orb}}
\DeclareMathOperator{\st}{\mathsf{st}}
\DeclareMathOperator{\aut}{\mathsf{Aut}}
\DeclareMathOperator{\bij}{\mathsf{Bij}}
\DeclareMathOperator{\rank}{\mathsf{rank}}
\DeclareMathOperator{\tr}{\mathsf{tr}}
\DeclareMathOperator{\id}{\mathsf{Id}}
\providecommand{\B}{\mathsf{B}}


\providecommand{\dpar}[2]{\frac{\partial #1}{\partial #2}}
\makeatother

\usepackage[a4paper,hmargin=30mm,vmargin=30mm]{geometry}
\title{\color{astral} \sffamily \bfseries Chapitre 2: Premiers algorithmes}
\author{Ivan Lejeune\thanks{Cours inspiré de M. Marche}}
\date{\today}

\begin{document}
	
	\maketitle
	\section{Chapitre 1: Optimisation convexe}
	\section{Chapitre 2: Premiers algorithmes}

	On considère $U\sub\bb R^n$ ouvert et $f\colon U\to\bb R$.
	
	On suppose dans la suite que l'on sait montrer l'existence du minimum

	On cherche un processus de construction d'une suite de points ${(x_k)}_{k\in\bb N}$ 
	telle que $(x_k)$ converge vers le minimum de $f$.

	\begin{oc-definition}
		Un algorithme \defemph{converge} si la suite de points converge vers le
		minimum recherché.

	\end{oc-definition}

	On suppose également qu'on connait des algorithmes qui convergent pour $n=1$
	$($cas des fonctions $f\colon I\sub \bb R\to\bb R)$.

	\subsection{Méthode de relaxation}

	La principe de la méthode de relaxation consiste à se ramener à la
	minimisation de fonctions à une variable, de la manière suivante:

	\begin{enumerate}[label=$(\roman*)$]
		\item On fixe une suite de directions de descente ${(d_k)}_{k\in\bb N}$,
		\item On fixe un point de départ $u_1$.
		\item Connaissant $u_k$, on calcule $\rho_{k+1}\in\bb R$ qui réalise le minimum de la fonction
			\[
				\phi_k\colon \rho\mapsto f(u_k+\rho\cdot d_{k+1})
			\]
		\item On pose $u_{k+1}=u_k+\rho_{k+1}d_{k+1}$.
	\end{enumerate}

	\begin{oc-theorem}
		Si $f$ est $\scr C^1$ et elliptique, alors la méthode de relaxation converge.
	\end{oc-theorem}

	\begin{myproof}[IMPORTANTE A COMPRENDRE]\,\\

		Nous avons vu au chapitre précédent que si $f$ est elliptique, alors
		\[
			\forall (x,y)\in\bb R^n, \underbrace{f(y)-f(x)-f'(x)(y-x)\geq \frac{\alpha}{2}{||y-x||}^2}_{(E)}
		\]
		et que
		\begin{center}
			$\ul x$ est un minimum de $f$ $\iff$ $f'(\ul x)=0$.
		\end{center}

		Considérons la fonction
		\[\begin{aligned}
			\phi_k\colon \bb R&\to\bb R\\
			t&\mapsto f(u_k+t d_{k+1})
		\end{aligned}\]
		qui atteint son minimum en $\rho_{k+1}$, donc
		\[
			\phi_k'(\rho_{k+1})=f'(x_{k+1})\cdot d_{k+1}=0
		\]

		et puisque $x_{k+1}-x_k$ est parallèle à $d_{k+1}$, on a
		\[
			f'(x_{k+1})\cdot(x_{k+1}-x_k)=0
		\]

		En outre, puisque $x_{k+1}$ minimise $f$ sur la droite passant par $x_k$
		et dirigée par $d_{k+1}$, on a
		\[
			f(x_{k+1})\leq f(x_k)
		\]

		Ainsi, la suite des $f(x_k)$ est décroissante, et minorée, donc convergente.

		Considérons maintenant l'inégalité $(E)$ avec $x=x_{k+1}$ et $y=x_k$.

		\[
			f(x_k)-f(x_{k+1})-f'(x_{k+1})(x_k-x_{k+1})\geq \frac{\alpha}{2}||x_k-x_{k+1}||^2\\
		\]

		Comme $f'(x_{k+1})\cdot(x_k-x_{k+1})=0$, on a
		\[
			f(x_k)-f(x_{k+1})\geq \frac{\alpha}{2}||x_k-x_{k+1}||^2
		\]

		Puisque la suite $(f(x_k))$ est convergente, on a
		\[
			\lim_{k\to\infty}||x_k-x_{k+1}||=0
		\]

		On a également, pour tout $p\in\{1,\dots,n\}$,
		\[
			\lim_{k\to\infty}x_{k+p}-x_k=0
		\]

		Par ailleurs, on a par ellipticité de $f$:
		\[
			\forall (x,y)\in\bb R^n, (f'(y)-f'(x))(y-x)\geq \alpha||y-x||^2
		\]

		Soit $\ul x$ le minimum de $f$. Puisque $f'(\ul x)=0$, on a, en appliquant l'inégalité précédente à $y=x_k$ et $x=\ul x$:
		\[
			\alpha||x_k-\ul x||^2\leq (f'(x_k)\cdot (x_k-\ul x))
		\]

		d'où
		\[
			\alpha||x_k-\ul x||^2\leq || f'(x_k)||\cdot||x_k-\ul x||
		\]

		et donc
		\[
			||x_k-\ul x||\leq \frac{||f'(x_k)||}{\alpha}
		\]

		Il reste donc à montrer que $f'(x_k)$ tend vers 0.

		Il suffit de montrer que:
		\[
			\forall p\in\{1,\dots,n\}, \lim_{k\to\infty}f'(x_k)\cdot e_p=0
		\]


		Soit donc $p\in\{1,\dots,n\}$, on a pour $j\in\bb N$:
		\[
			f'(x_{p+j_n})\cdot d_{p+j_n}=0
		\]
		or $d_{p+j_n}=e_p$ donc
		\[
			f'(x_{p+j_n})\cdot e_p=0
		\]

		On effectue la division euclidienne de $k$ par $n$:
		\begin{center}
			$k=\alpha n + i$ avec $i\in\{1,\dots,n\}$
		\end{center}

		On a alors
		\[\begin{aligned}
			f'(x_k)\cdot e_p&=f'(x_{\alpha n + i})\cdot e_p\\
			&=(f'(x_{\alpha n+1})-f'(x_{\alpha n+p}))\cdot e_p
		\end{aligned}\]
		puisque $f'(x_{\alpha n+p})\cdot e_p=0$.

		Ainsi, 
		\[
			|f'(x_k)\cdot e_p|\leq ||f'(x_{\alpha n+i})-f'(x_{\alpha n+p})||\cdot||e_p||
		\]

		Or, lorsque $k$ tend vers l'infini, $\alpha$ tend vers l'infini, et $i$ est borné par définition.

		Ainsi,
		\[
			\lim_{\alpha\to\infty}x_{\alpha n+i}-x_{\alpha n}=0
		\]
		et
		\[
			\lim_{\alpha\to\infty}x_{\alpha n}- x_{\alpha n+p}=0
		\]

		Ainsi, pour $\alpha\to\infty$, on a
		\[
			||x_{\alpha n+i}-x_{\alpha n+p}||\to 0
		\]

		Enfin, par uniforme continuité de $f'$ sur une boule $\ol B(0, M)$, on a
		\[
			||f'(x_{\alpha n+i})-f'(x_{\alpha n+p})||\to 0
		\]

		et donc finalement
		\[
			\lim_{k\to\infty}f'(x_k)\cdot e_p=0
		\]

		\begin{oc-remark}[sur le choix de $M$]
			On veut qu'à partir d'un certain rang, tous les termes
			de la suite soient dans la boule $\ol B(0,M)$.

			Or, la suite $(f(x_k))$ est bornée et $f$ est coercive, ce qui
			implique que la suite $(x_k)$ est nécessairement bornée.
		\end{oc-remark}

	\end{myproof}

	\subsection{Méthode de gradient à pas optimal}

	On considère $f\colon \bb R^n\to\bb R$.

	L'idée de la méthode est de choisir des directions de descente ($\dots$ vers le minimum) privilégiées.

	Connaissant $x_k$, on considère la fonction
	\[\begin{aligned}
		\phi_k\colon \bb R&\to\bb R\\
		t&\mapsto f(x_k-t\underbrace{\nabla f(x_k)}_{d_{k+1}})
	\end{aligned}\]
	et $t_{k+1}$ qui minimise $\phi_k$ sur $\bb R$.

	On pose alors $x_{k+1}=x_k-t_{k+1}\nabla f(x_k)$.

	Pourquoi un tel choix de descente selon la direction $-\nabla f(x_k)$?

	Parce que c'est localement le meilleur choix possible:
	\[
		f(x+t\xi)=f(x)+t\nabla f(x)\cdot \xi + o(t)
	\]
	Ainsi, on observe qu'en négligeant le reste, on `descend' le plus possible
	en rendant le terme $t\nabla f(x)\cdot \xi$ le plus négatif possible.
	Il faut donc choisir $\xi$ parallèle à $\nabla f(x)$ et de direction opposée.

	\begin{oc-theorem}
		Si $f$ est $\scr C^1$ et elliptique, alors la méthode de gradient à pas optimal converge.
	\end{oc-theorem}

	\begin{myproof}
		La preuve est similaire à celle de la méthode de relaxation, en utilisant
		la propriété de coercivité de $f$:

		La suite des $f(x_k)$ est décroissante et minorée, donc convergente.

		Ainsi, la suite $(f(x_k))$ est convergente, et la suite $(x_k)$ est bornée.

		Par ailleurs, on a $\nabla f(x_{k+1})\perp \nabla f(x_k)$ car
		\[\begin{aligned}
			\phi_k'(t_{k+1})=0&=f'(x_{k+1}\nabla f(x_k))\\
			&=(\nabla f(x_{k+1}), \nabla f(x_k))\\
		\end{aligned}\]

		On a également $(\nabla f(x_{k+1}), x_{k+1}-x_k)= 0$

		En outre, on a
		\[
			f(x_k)-f(x_{k+1})\geq \frac{\alpha}{2}||x_k-x_{k+1}||^2
		\]
		d'où $||x_k-x_{k+1}||\to 0$.

		On peut de manière similaire à la preuve précédente,
		montrer que $(\nabla f(x_k))$ tend vers 0.

		Pour conclure, pour $x=\ul x$ et $y=x_k$, on a
		\[
			(\nabla f(\ul x)-\nabla f(x_k)),(x_k-\ul x)\geq \alpha||x_k-\ul x||^2
		\]
		d'où
		\[
			||\ul x-x_k||\leq \frac{||\nabla f(\ul x) - \nabla f(x_k)||}{\alpha}
		\]
		Comme $\nabla f(\ul x)=0$, on a
		\[
			||\ul x-x_k||\le \frac{||\nabla f(x_k)||}{\alpha}\to 0
		\]
		et donc $x_k\to \ul x$.
	\end{myproof}

	\subsection{Méthode de gradient à pas fixe ou variable}

	Pour ne pas avoir à résoudre le problème d'optimisation 1d à chaque itération,
	on peut fixer à priori une suite de réels ${(\rho_k)}_{k\in\bb N}$ et poser
	\[
		x_{k+1}=x_k-\rho_{k+1}\nabla f(x_k)
	\]
	Il est même possible de choisir $\rho_k=\rho$ constant.

	\begin{oc-definition}
		Soit $f\colon \bb R^n\to\bb R$ de classe $\scr C^1$. On dit que
		$\nabla f$ est \defemph{Lipschitzienne} de constante $M>0$ si
		\[
			\forall (x,y)\in\bb R^n, ||\nabla f(x)-\nabla f(y)||\leq M||x-y||
		\]
	\end{oc-definition}

	\begin{oc-theorem}
		On suppose que $f$ est $\scr C^1$ elliptique (de constante $\alpha$) et que
		$\nabla f$ est Lipschitzienne de constante $M$.

		Si il existe deux constantes $a,b\in \bb R$ telles que
		\[
			\forall K\in\bb N,0<a\leq \rho_k\leq b<\frac{2\alpha}{M^2}
		\]
		alors la méthode de gradient à pas fixe converge et il existe $\beta<1$ tel que
		tel que
		\[
			\forall K\in\bb N,||x_k-\ul x||\leq \beta^k||x_0-\ul x||
		\]
	\end{oc-theorem}

\end{document}