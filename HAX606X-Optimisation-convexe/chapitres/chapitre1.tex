% txs:///compile | txs:///view-pdf-internal --embedded | cmd /c move /Y output\%.pdf pdfs\
% pdflatex -synctex=1 -interaction=nonstopmode -output-directory=output %.tex

\documentclass[french,a4paper,10pt]{article}
\makeatletter
%--------------------------------------------------------------------------------
\usepackage[T1]{fontenc} % font type
\usepackage[french]{babel} % language
\usepackage{lmodern} % font type
\usepackage[shortlabels]{enumitem}
\setlist[itemize,1]{label={\color{gray}\small \textbullet}} % customises itemize default -
\usepackage{fancyhdr} % customises head and foot-notes
\usepackage{centernot} % allows centering \not with \centernot
\usepackage{stmaryrd} % allows \llbracket
\usepackage[overload]{abraces} % allows \aoverbrace

\usepackage{xcolor} % colour customisation, extends to tables with {colortbl}
\definecolor{astral}{RGB}{46,116,181}
\definecolor{verdant}{RGB}{96,172,128}
\definecolor{algebraic-amber}{RGB}{255,179,102} % definition colour
\definecolor{calculus-coral}{RGB}{255,191,191} % exercice colour
\definecolor{divergent-denim}{RGB}{130,172,211} % proposition colour 
\definecolor{matrix-mist}{RGB}{204,204,204} % remark colour
\definecolor{numeric-navy}{RGB}{204,204,204} % theorem colour 
\definecolor{quadratic-quartz}{RGB}{204,153,153} % example colour 


\usepackage{latexsym}
\usepackage{amsmath}
\usepackage{amsfonts}
\usepackage{amssymb}
\usepackage{amsthm}
\usepackage{mathtools}
\usepackage{mathrsfs}
\usepackage{MnSymbol}
\usepackage{etoolbox}% http://ctan.org/pkg/etoolbox

%\usepackage{tikz}
%\usepackage{pgfplots}
%\pgfplotsset{compat=1.18}
%\usetikzlibrary{arrows}


\newtheoremstyle{gen-style}{\topsep}{\topsep}%
{}%         Body font
{}%         Indent amount (empty = no indent, \parindent = para indent)
{\sffamily\bfseries}% Thm head font
{.}%        Punctuation after thm head
{ }%     Space after thm head (\newline = linebreak)
{\thmname{#1}\thmnumber{~#2}\thmnote{~#3}}%         Thm head spec


\newtheoremstyle{no-num-style}{\topsep}{\topsep}%
{}%         Body font
{}%         Indent amount (empty = no indent, \parindent = para indent)
{\sffamily\bfseries}% Thm head font
{.}%        Punctuation after thm head
{ }%     Space after thm head (\newline = linebreak)
{\thmname{#1}}%         Thm head spec


\usepackage[]{mdframed}

\newcommand{\mytheorem}[4]{%
	\newmdtheoremenv[
	hidealllines=true,
	leftline=true,
	skipabove=0pt,
	innertopmargin=-5pt,
	innerbottommargin=2pt,
	linewidth=4pt,
	innerrightmargin=0pt,
	linecolor=#3,
	]{#1}{#2}[#4]%
}


\newcommand{\mytheoremnocount}[3]{%
	\newmdtheoremenv[
	hidealllines=true,
	leftline=true,
	skipabove=0pt,
	innertopmargin=-5pt,
	innerbottommargin=2pt,
	linewidth=4pt,
	innerrightmargin=0pt,
	linecolor=#3,
	]{#1}{#2}%
}
\newcommand{\myoctheorem}[4]{%
	\newmdtheoremenv[
	hidealllines=true,
	leftline=true,
	skipabove=0pt,
	innertopmargin=-5pt,
	innerbottommargin=2pt,
	linewidth=4pt,
	innerrightmargin=0pt,
	linecolor=#3,
	]{#1}[#4]{#2}%
}

\theoremstyle{gen-style}
\mytheorem{proposition}{Proposition}{divergent-denim}{section}
\mytheorem{propdef}{Proposition - Définition}{divergent-denim}{section}
\mytheorem{theorem}{Théorème}{quadratic-quartz}{section}
\mytheorem{lemme}{Lemme}{quadratic-quartz}{section}
\mytheorem{example}{Exemple}{quadratic-quartz}{section}
\mytheorem{remark}{Remarque}{matrix-mist}{section}
\mytheorem{notation}{Notation}{matrix-mist}{section}
\mytheorem{exercise}{Exercice}{calculus-coral}{section}
\mytheorem{exercice}{Exercice}{calculus-coral}{section}
\mytheorem{definition}{Definition}{algebraic-amber}{section}

\newcounter{oc-counter}
\myoctheorem{oc-proposition}{Proposition}{divergent-denim}{oc-counter}
\myoctheorem{oc-propdef}{Proposition - Définition}{divergent-denim}{oc-counter}
\myoctheorem{oc-theorem}{Théorème}{divergent-denim}{oc-counter}
\myoctheorem{oc-lemme}{Lemme}{quadratic-quartz}{oc-counter}
\myoctheorem{oc-example}{Exemple}{quadratic-quartz}{oc-counter}
\myoctheorem{oc-remark}{Remarque}{matrix-mist}{oc-counter}
\myoctheorem{oc-exercise}{Exercice}{calculus-coral}{oc-counter}
\myoctheorem{oc-definition}{Definition}{algebraic-amber}{oc-counter}

\theoremstyle{no-num-style}
\mytheoremnocount{td-sol}{Solution}{verdant}
\mytheoremnocount{no-num-definition}{Definition}{algebraic-amber}
\mytheoremnocount{no-num-theorem}{Théorème}{algebraic-amber}
\mytheoremnocount{oc-intro}{Introduction}{quadratic-quartz}
\mytheoremnocount{oc-proof}{Preuve}{verdant}
\mytheoremnocount{oc-young}{Formule de Taylor à l'ordre 2}{verdant}
\mytheoremnocount{oc-notation}{Notation}{matrix-mist}
\mytheorem{rappel}{Rappel}{matrix-mist}{section}
\mytheoremnocount{myproof}{Preuve}{verdant}
\mytheoremnocount{td-exo}{Exercice}{calculus-coral}
\numberwithin{oc-counter}{subsection}

%---------------
% Mise en page
%--------------

\setlength{\parindent}{0pt}

\providecommand{\defemph}[1]{{\sffamily\bfseries\color{astral}#1}}


\usepackage{sectsty}
\allsectionsfont{\color{astral}\normalfont\sffamily\bfseries}

\usepackage{mathrsfs}

%----- Easy way to redeclare math operators -----
\makeatletter
\newcommand\RedeclareMathOperator{%
	\@ifstar{\def\rmo@s{m}\rmo@redeclare}{\def\rmo@s{o}\rmo@redeclare}%
}
\newcommand\rmo@redeclare[2]{%
	\begingroup \escapechar\m@ne\xdef\@gtempa{{\string#1}}\endgroup
	\expandafter\@ifundefined\@gtempa
	{\@latex@error{\noexpand#1undefined}\@ehc}%
	\relax
	\expandafter\rmo@declmathop\rmo@s{#1}{#2}}
\newcommand\rmo@declmathop[3]{%
	\DeclareRobustCommand{#2}{\qopname\newmcodes@#1{#3}}%
}
\@onlypreamble\RedeclareMathOperator
\makeatother

\newcommand{\skipline}{\vspace{\baselineskip}}
\newcommand{\noi}{\noindent}
%------------------------------------------------


\newcommand{\adh}[1]{\mathring{#1}} %adherence
\newcommand{\badh}[1]{\mathring{\overbrace{#1}}} % big adherence
\newcommand{\norm}{\mathcal{N}} % norme
\newcommand{\ol}[1]{\overline{#1}} % overline
\newcommand{\ul}[1]{\underline{#1}} % underline
\newcommand{\sub}{\subset} % subset
\newcommand{\scr}[1]{\mathscr{#1}} % scr rapide
\newcommand{\bb}[1]{\mathbb{#1}} % bb rapide
\newcommand{\bolo}[1]{B({#1}\mathopen{}[\mathclose{}} % boule ouverte
\newcommand{\bolf}[1]{B({#1}\mathopen{}]\mathclose{}} % boule fermee
\newcommand{\act}{\circlearrowleft} % agit sur
\newcommand{\glx}[1]{\text{GL}_{#1}} % GL_x
\newcommand{\cequiv}[1]{\mathopen{}[#1\mathclose{}]} % classe d'equivalence
\newcommand{\restr}[2]{#1\mathop{}\!|_{#2}} % restriction


%----- Intervalles -----
\newcommand{\oo}[1]{\mathopen{]}#1\mathclose{[}}
\newcommand{\of}[1]{\mathopen{]}#1\mathclose{]}}
\newcommand{\fo}[1]{\mathopen{[}#1\mathclose{[}}
\newcommand{\ff}[1]{\mathopen{[}#1\mathclose{]}}



\providecommand{\1}{\mathds{1}}
\DeclareMathOperator{\im}{\mathsf{Im}}
\DeclareRobustCommand{\re}{\mathsf{Re}}
\RedeclareMathOperator{\ker}{\mathsf{Ker}}
\RedeclareMathOperator{\det}{\mathsf{det}}
\DeclareMathOperator{\vect}{\mathsf{Vect}}
\DeclareMathOperator{\orb}{\mathsf{orb}}
\DeclareMathOperator{\st}{\mathsf{st}}
\DeclareMathOperator{\aut}{\mathsf{Aut}}
\DeclareMathOperator{\bij}{\mathsf{Bij}}
\DeclareMathOperator{\rank}{\mathsf{rank}}
\DeclareMathOperator{\tr}{\mathsf{tr}}
\DeclareMathOperator{\id}{\mathsf{Id}}
\providecommand{\B}{\mathsf{B}}


\providecommand{\dpar}[2]{\frac{\partial #1}{\partial #2}}
\makeatother

\usepackage[a4paper,hmargin=30mm,vmargin=30mm]{geometry}
\title{\color{astral} \sffamily \bfseries Optimisation Convexe}
\author{Ivan Lejeune\thanks{Cours inspiré de M. Charlier et M. Marche}}
\date{\today}

\begin{document}
	
	\maketitle
	\section{Optimisation en dimension finie}
	\subsection{Introduction}
	Méthode d'évaluation :
	\begin{itemize}
		\item CC noté en CM
		\item TP noté
		\item Examen terminal
	\end{itemize}
	% Notation
	On considère un espace vectoriel normé de dimension $n$ noté $(E, ||\cdot||)$ et $U$ ouvert de $E$.
	
	On considère une fonction $f\colon U\to \R$. Dans la pratique, $E$ sera égal à $\R^n$.
	
	Soit $x\in U$, on note $f'(x)$ la différentielle (qu'on appelera plus simplement "dérivée") de $f$ en $x$. 
	
	On a donc, pour tout $h\in E$ tel que $||h||$ soit assez petit, 
		\[\begin{aligned}
			f(x+h)=f(x)+f'(x)\cdot h+||h||\varepsilon(x, h)
		\end{aligned}\]
		avec $\varepsilon(x, h)\underset{h\to 0}\to 0$
		et $f'(x)\in\scr L(E, \R)$.
		
	Avec cette notation, si $f$ est dérivable en $x$, alors $f$ admet des dérivées partielles en $x$ dans toutes les directions, et si $(e_1,\dots,e_n)$ est une base de $E$, on note "$\partial f(x)$" ou encore $\frac{\partial f}{\partial x}(x)$".
	
	La dérivée partielle de $f$ par rapport à la ieme variable. On a alors
	
		\[\begin{aligned}
			\partial f(x)=f'(x)\cdot e_i\qquad i=1,\dots n
		\end{aligned}\]
	
	Ainsi, pour $h\in E$ tel que $h=\sum_{i=1}^nh_ie_i$; on a
	
		\[\begin{aligned}
			f'(x)\cdot h &= f'(x)\cdot\left(\sum_{i=1}^nh_ie_i\right)\\
			&=\sum_{i=1}^n h_i f'(x) e_i\\
			&=\sum_{i=1}^{n} h_i\partial f(x)
		\end{aligned}\]
	
	De même, si $x\mapsto f(x)$ est dérivable en $x$, on note $f''(x)\in\scr L(E;\scr L(E, \bb R))$ cette dérivée seconde et on considère $f''(x)$ comme une forme bilinéaire 
	
		\[\begin{aligned}
			f''(x)\in\scr L(E\times E,\R)
		\end{aligned}\]
	
	Avec ces notations, le théorème fondamental de l'analyse (TTA) peut s'énoncer ainsi :
	
	\begin{theorem}
		Soit $f\in\scr C'(U, \R)$. Alors pour tout $(x, y)\in U$ tel que $\forall t\in\ff{0, 1}, x+t(y-x)\in U$, on a
			\[\begin{aligned}
				f(y)=f(x) + \int_0^1f'(x+t(y-x))\cdot(y-x)dt
			\end{aligned}\]	
	\end{theorem}

	% Formule de Taylor à l'ordre 2
	
	Soit $f\in \scr C^2(U, \R), x\in U$. Alors il existe un voisinage $v$ de $x$ tel que pour tout $h\in v$
	
		\[\begin{aligned}
			f(x+h)=f(x)+f'(x)\cdot h+\frac12 f''(x)\cdot(h,h)+ o(||h||^2)
		\end{aligned}\]
	
	Bien entendu, cette expression peut aussi se formuler ainsi :
		\[\begin{aligned}
			f(x+h)=f(x)&+\sum_{i=1}^n\partial f(x)h_i\\&+\frac12\sum_{i,j=1}^n\partial^2f(x)h_ih_j\\&+||h||^2\varepsilon (h)\\
		\end{aligned}\]
	avec $\varepsilon(h)\underset{||h||\to 0}\to 0$.
	
	% Gradient d'une fonction
	
	Soit $f\colon U\to \R$ de classe $\scr C^1$. Alors pour tout $a\in U$, il existe un unique vecteur, noté $\nabla f(a)$ tel que pour tout $h\in E$
		\[\begin{aligned}
			f'(a)\cdot h=<\nabla f(a), h>
		\end{aligned}\]
	
	où $<\cdot, \cdot>$ désigne le produit scalaire canonique sur $\R^n$
	
	\subsection{Résultats d'existence}
	
	Un outil fondamental à la compacité
	
	\begin{theorem}
		Soit $K$ un compact de $\R^n$ et $f\colon K\to \bb R$ une fonction continue. Alors $f$ est bornée et atteint ses bornes :
		
			\[\begin{aligned}
				\sup_{x\in K}|f(x)|<+\infty
			\end{aligned}\]
		
		et il existe $\underbar x\in K$ et $\ol x\in K$ tels que 
			\[\begin{aligned}
				f(\underbar x)&= \inf_{x\in K}f(x)=\min_{x\in K}f(x)\\
				f(\ol x)&= \sup_{x\in K}f(x)=\max_{x\in K}f(x)\\
			\end{aligned}\]
	\end{theorem}

	\begin{myproof}
		Ce résultat à été démontré dans le cours de topologie / analyse fonctionnelle. Puisque $f$ est continue, $f(K)$ est une partie compacte de $\R$, c'est à dire une partie fermée et bornée de $\R$. Ainsi on a
		\[\begin{aligned}
			-\infty<\inf f(K)\le \sup f(k)<+\infty
		\end{aligned}\]
		et puisque $f(K)$ est fermé et que $\inf f(k)$ et $\sup f(k)$ sont adhérents à $f(K)$, on a nécessairement
			\[\begin{aligned}
				&\inf f(K)=\min f(K)\in f(E)\\
				\text{ et}&\sup f(K)=\max f(K)\in f(E)
			\end{aligned}\]
	\end{myproof}
	\begin{definition}
		Soit $f\colon\R^n\to \R$. On dit que $f$ est \defemph{coercive} si $f(x)\to+\infty$ lorsque $||x||\to+\infty$.
		
	\end{definition}
	\begin{theorem}
		$f\colon\R^n\to\R$ continue et coercive. Alors $f$ est minorée et atteint son minimum.
		
	\end{theorem}

	\begin{myproof}
		Posons $A=f(0)+1$.
		
		Puisque $f$ est coercive, il existe $\alpha>0$ tel que
			\[\begin{aligned}
				\forall x\in\R^n,||x||\ge\alpha\Longrightarrow f(x)\ge f(0)+1
			\end{aligned}\]
		
		La boule $\ol B(0, \alpha)$ est un fermé borné de $\R^n$ donc un compact de $\R^n$ et $\rst f{\ol B(0, \alpha)}$ est continue.
		
		D'après le Théorème 1.1.1, $f$ est minorée sur $\ol B(0, \alpha)$ et atteint son minimum en un certain $x_0\quad(\forall x\in \ol B(0, \alpha), f(x)\ge f(x_0))$
		
		Ainsi, soit $x\in \R^n$
		\begin{enumerate}[label=$a\\$]
			\item si $x\in \ol B(0, \alpha)$, alors $f(x)\ge f(x_0)$
			\item $x\notin \ol B(0, \alpha)$, alors $||x||>\alpha$ et donc $f(x)\ge f(0)+1$ et $f(x)\ge f(x_0)+1>$...
		\end{enumerate}
	
		puisque $0\in \ol B(0, \alpha)$
		
		Ainsi...
		
	\end{myproof}

	\begin{myremark}
		Ce dernier résultat peut être généralisé, sous les même hypothèses, au cas d'une fonction $f\colon K\to\R$ avec $K$ fermé de $\R^n$.
		
	\end{myremark}

	\subsection{Caractérisation des extremas sans contraintes}
	
	Un outil fondamental au calcul différentiel.
	
	\begin{theorem}
		Soit $U\sub \R^n$ ouvert et $f\colon U\to\R$ de classe $\scr C^1$. Si $x_0\in U$ est extremum local de $f$ sur $U$ alors on a $f'(x_0)=0$ $($ou $\nabla f(x_0)=0)$
		
	\end{theorem}

	\begin{myproof}
		Rappelons ce qu'il se passe pour une fonction $\varphi\colon I\sub \R\to\R$ qui admet par exemple un maximum local en $O\in I$.
		
		On a d'une part $\varphi'(0)=\lim\lim\limits_{x\to0^+}\frac{\varphi(x)-\varphi(0)}x\le 0$
		car $x>0$ et $\varphi(x)-\varphi(0)\le 0$
		
		et d'autre part $\varphi'(0)=\lim\limits_{x\to0^-}\frac{\varphi(x)-\varphi(0)}x\ge 0$ car $x<0$ et $\varphi(x)-\varphi(0)\le 0$
		
		Dans le cas $E=\R^n$, supposons que $f$ admet un maximum local en $x_0\in U$. Soit $e_i$ un vecteur de base.
		
		On sait que $\partial_i f(x_0)=f'(x_0)\cdot e_i=\varphi_{e_i}'(0)$ avec $\varphi_{e_i}(t)=f(x_0+te_i)$
		
		(où on remarque que $t\in\oo{-\delta,\delta})$ puisque $U$ est ouvert et $x_0\in U$)
		
		Puisque $f$ admet un maximum local en $x_0$ il existe $r>0$ tel que
		\[\begin{aligned}
			\forall x\in B(x_0, r),\quad f(x)\le f(x_0)
		\end{aligned}\]
	
		Soit $h\in\R^n$ tel que $||h||\le r$ alors $f(x_0+h)\le f(x_0)$.
		
		et en particulier
		
		\[\begin{aligned}
			|t|\le r&\Longrightarrow f(x_0+t e_i)\le f(x_0)\\
			&\Longrightarrow\varphi_{e_i}'(0)=0
		\end{aligned}\]
	
		Ainsi, toutes les dérivées partielles de $f$ sont nulles en $x_0$ et donc $f'(x_0)=0$
		
	\end{myproof}
\end{document}