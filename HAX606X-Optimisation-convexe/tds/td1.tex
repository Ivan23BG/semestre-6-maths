\documentclass[french,a4paper,10pt]{article}
\makeatletter
%--------------------------------------------------------------------------------
\usepackage[T1]{fontenc} % font type
\usepackage[french]{babel} % language
\usepackage{lmodern} % font type
\usepackage[shortlabels]{enumitem}
\setlist[itemize,1]{label={\color{gray}\small \textbullet}} % customises itemize default -
\usepackage{fancyhdr} % customises head and foot-notes
\usepackage{centernot} % allows centering \not with \centernot
\usepackage{stmaryrd} % allows \llbracket
\usepackage[overload]{abraces} % allows \aoverbrace

\usepackage{xcolor} % colour customisation, extends to tables with {colortbl}
\definecolor{astral}{RGB}{46,116,181}
\definecolor{verdant}{RGB}{96,172,128}
\definecolor{algebraic-amber}{RGB}{255,179,102} % definition colour
\definecolor{calculus-coral}{RGB}{255,191,191} % exercice colour
\definecolor{divergent-denim}{RGB}{130,172,211} % proposition colour 
\definecolor{matrix-mist}{RGB}{204,204,204} % remark colour
\definecolor{numeric-navy}{RGB}{204,204,204} % theorem colour 
\definecolor{quadratic-quartz}{RGB}{204,153,153} % example colour 


\usepackage{latexsym}
\usepackage{amsmath}
\usepackage{amsfonts}
\usepackage{amssymb}
\usepackage{amsthm}
\usepackage{mathtools}
\usepackage{mathrsfs}
\usepackage{mnsymbol}
\usepackage{etoolbox}% http://ctan.org/pkg/etoolbox

\let\overfence\overbrace % \overfence is similar to \overbrace
\let\downfencefill\downbracefill % match components of \overbrace
\patchcmd{\overfence}{\downbracefill}{\downfencefill}{}{}% patch \overfence...
\patchcmd{\downfencefill}{\braceru \bracelu}{}{}{}%... and \downfencefill


\usepackage{tikz}
\usepackage{pgfplots}
\pgfplotsset{compat=1.18}
\usetikzlibrary{arrows}


\newtheoremstyle{gen-style}{\topsep}{\topsep}%
{}%         Body font
{}%         Indent amount (empty = no indent, \parindent = para indent)
{\sffamily\bfseries}% Thm head font
{.}%        Punctuation after thm head
{ }%     Space after thm head (\newline = linebreak)
{\thmname{#1}\thmnumber{~#2}\thmnote{~#3}}%         Thm head spec


\newtheoremstyle{no-num-style}{\topsep}{\topsep}%
{}%         Body font
{}%         Indent amount (empty = no indent, \parindent = para indent)
{\sffamily\bfseries}% Thm head font
{.}%        Punctuation after thm head
{ }%     Space after thm head (\newline = linebreak)
{\thmname{#1}}%         Thm head spec


\usepackage[]{mdframed}

\newcommand{\mytheorem}[4]{%
	\newmdtheoremenv[
	hidealllines=true,
	leftline=true,
	skipabove=0pt,
	innertopmargin=-5pt,
	innerbottommargin=2pt,
	linewidth=4pt,
	innerrightmargin=0pt,
	linecolor=#3,
	]{#1}{#2}[#4]%
}

\newcommand{\myoctheorem}[4]{%
	\newmdtheoremenv[
	hidealllines=true,
	leftline=true,
	skipabove=0pt,
	innertopmargin=-5pt,
	innerbottommargin=2pt,
	linewidth=4pt,
	innerrightmargin=0pt,
	linecolor=#3,
	]{#1}[#4]{#2}%
}

\newcommand{\mytheoremnocount}[3]{%
	\newmdtheoremenv[
	hidealllines=true,
	leftline=true,
	skipabove=0pt,
	innertopmargin=-5pt,
	innerbottommargin=2pt,
	linewidth=4pt,
	innerrightmargin=0pt,
	linecolor=#3,
	]{#1}{#2}%
}

\theoremstyle{gen-style}
\mytheorem{proposition}{Proposition}{divergent-denim}{section}
\mytheorem{propdef}{Proposition - Définition}{divergent-denim}{section}
\mytheorem{theorem}{Théorème}{quadratic-quartz}{section}
\mytheorem{lemme}{Lemme}{quadratic-quartz}{section}
\mytheorem{example}{Exemple}{quadratic-quartz}{section}
\mytheorem{remark}{Remarque}{matrix-mist}{section}
\mytheorem{notation}{Notation}{matrix-mist}{section}
\mytheorem{exercise}{Exercice}{calculus-coral}{section}
\mytheorem{exercice}{Exercice}{calculus-coral}{section}
\mytheorem{definition}{Definition}{algebraic-amber}{section}
\newcounter{oc-counter}
\myoctheorem{oc-proposition}{Proposition}{divergent-denim}{oc-counter}
\myoctheorem{oc-propdef}{Proposition - Définition}{divergent-denim}{oc-counter}
\myoctheorem{oc-theorem}{Théorème}{divergent-denim}{oc-counter}
\myoctheorem{oc-lemme}{Lemme}{quadratic-quartz}{oc-counter}
\myoctheorem{oc-example}{Exemple}{quadratic-quartz}{oc-counter}
\myoctheorem{oc-remark}{Remarque}{matrix-mist}{oc-counter}
\myoctheorem{oc-exercise}{Exercice}{calculus-coral}{oc-counter}
\myoctheorem{oc-definition}{Definition}{algebraic-amber}{oc-counter}
\theoremstyle{no-num-style}
\mytheoremnocount{td-sol}{Solution}{verdant}
\mytheoremnocount{no-num-definition}{Definition}{algebraic-amber}
\mytheoremnocount{no-num-theorem}{Théorème}{algebraic-amber}
\mytheoremnocount{oc-intro}{Introduction}{quadratic-quartz}
\mytheoremnocount{oc-proof}{Preuve}{verdant}
\mytheoremnocount{oc-young}{Formule de Taylor à l'ordre 2}{verdant}
\mytheoremnocount{oc-notation}{Notation}{matrix-mist}
\mytheorem{rappel}{Rappel}{matrix-mist}{section}
\mytheoremnocount{myproof}{Preuve}{verdant}
\mytheoremnocount{td-exo}{Exercice}{calculus-coral}
\numberwithin{oc-counter}{subsection}

%---------------
% Mise en page
%--------------

\setlength{\parindent}{0pt}

\providecommand{\defemph}[1]{{\sffamily\bfseries\color{astral}#1}}


\usepackage{sectsty}
\allsectionsfont{\color{astral}\normalfont\sffamily\bfseries}

\usepackage{mathrsfs}

%----- Easy way to redeclare math operators -----
\makeatletter
\newcommand\RedeclareMathOperator{%
	\@ifstar{\def\rmo@s{m}\rmo@redeclare}{\def\rmo@s{o}\rmo@redeclare}%
}
\newcommand\rmo@redeclare[2]{%
	\begingroup \escapechar\m@ne\xdef\@gtempa{{\string#1}}\endgroup
	\expandafter\@ifundefined\@gtempa
	{\@latex@error{\noexpand#1undefined}\@ehc}%
	\relax
	\expandafter\rmo@declmathop\rmo@s{#1}{#2}}
\newcommand\rmo@declmathop[3]{%
	\DeclareRobustCommand{#2}{\qopname\newmcodes@#1{#3}}%
}
\@onlypreamble\RedeclareMathOperator
\makeatother

\newcommand{\skipline}{\vspace{\baselineskip}}
\newcommand{\noi}{\noindent}
%------------------------------------------------


\newcommand{\adh}[1]{\mathring{#1}} %adherence
\newcommand{\badh}[1]{\mathring{\overfence{#1}}} % big adherence
\newcommand{\norm}{\mathcal{N}} % norme
\newcommand{\ol}[1]{\overline{#1}} % overline
\newcommand{\sub}{\subset} % subset
\newcommand{\scr}[1]{\mathscr{#1}} % scr rapide
\newcommand{\bb}[1]{\mathbb{#1}} % bb rapide
\newcommand{\bolo}[1]{B({#1}\mathopen{}[\mathclose{}} % boule ouverte
\newcommand{\bolf}[1]{B({#1}\mathopen{}]\mathclose{}} % boule fermee
\newcommand{\act}{\circlearrowleft} % agit sur
\newcommand{\glx}[1]{\text{GL}_{#1}} % GL_x
\newcommand{\cequiv}[1]{\mathopen{}[#1\mathclose{}]} % classe d'equivalence
\newcommand{\restr}[2]{#1\mathop{}\!|_{#2}} % restriction


%----- Intervalles -----
\newcommand{\oo}[1]{\mathopen{]}#1\mathclose{[}}
\newcommand{\of}[1]{\mathopen{]}#1\mathclose{]}}
\newcommand{\fo}[1]{\mathopen{[}#1\mathclose{[}}
\newcommand{\ff}[1]{\mathopen{[}#1\mathclose{]}}



\providecommand{\1}{\mathds{1}}
\DeclareMathOperator{\im}{\mathsf{Im}}
\DeclareRobustCommand{\re}{\mathsf{Re}}
\RedeclareMathOperator{\ker}{\mathsf{Ker}}
\RedeclareMathOperator{\det}{\mathsf{det}}
\DeclareMathOperator{\vect}{\mathsf{Vect}}
\DeclareMathOperator{\orb}{\mathsf{orb}}
\DeclareMathOperator{\st}{\mathsf{st}}
\DeclareMathOperator{\aut}{\mathsf{Aut}}
\DeclareMathOperator{\bij}{\mathsf{Bij}}
\DeclareMathOperator{\rank}{\mathsf{rank}}
\DeclareMathOperator{\tr}{\mathsf{tr}}
\DeclareMathOperator{\id}{\mathsf{Id}}
\providecommand{\B}{\mathsf{B}}


\providecommand{\dpar}[2]{\frac{\partial #1}{\partial #2}}
\makeatother

\usepackage[a4paper,hmargin=30mm,vmargin=30mm]{geometry}

\title{\color{astral} \sffamily \bfseries TD1 - Algorithmes unidimensionnels}
\author{Ivan Lejeune\thanks{Cours inspiré de M. Marche}}
\date{\today}
% pdflatex -output-directory=output chapter1.tex && move /Y output\chapter1.pdf .\

\begin{document}
	\maketitle
	\begin{td-exo}[1. Minimisation d'une fonction par dichotomie]
		Soit $f\in\scr C^0(\ff{a,b},\bb R)$. On dit que $f$ est \defemph{unimodale} sur l'intervalle $\ff{a,b}$ si il existe un point $\ol x\in\ff{a,b}$ tel que $f$ soit strictement décroissante sur $\fo{a,\ol x}$ et strictement croissante sur $\of{\ol x, b}$.\\
		
		Pour chercher $\ol x$, nous allons générer une suite strictement décroissante d'intervalles dont le diamètre tend vers 0 et qui encadrent le minimum recherché.\\
		
		Supposons connus cinq point $a=x_1<x_2<x_3<x_4<x_5=b$. Cinq situations se présentent :
			\begin{enumerate}[label=$(\roman*)$]
				\item $f(x_1)<f(x_2)<f(x_3)<f(x_4)<f(x_5)$ : $\ol x$ appartient alors à $\oo{x_1,x_2}$,
				
				\item $f(x_1)>f(x_2),\quad f(x_2)<f(x_3)<f(x_4)<f(x_5)$ : $\ol x$ appartient alors à $\oo{x_1,x_3}$,
				
				\item $f(x_1)>f(x_2)>f(x_3),\quad f(x_3)<f(x_4)<f(x_5)$ : $\ol x$ appartient alors à $\oo{x_2,x_4}$,
				
				\item $f(x_1)>f(x_2)>f(x_3)>f(x_4),\quad f(x_4)<f(x_5)$ : $\ol x$ appartient alors à $\oo{x_3,x_5}$,
				
				\item $f(x_1)>f(x_2)>f(x_3)>f(x_4)>f(x_5)$ : $\ol x$ appartient alors à $\oo{x_4,x_5}$.
			\end{enumerate}
		\begin{enumerate}
			\item Utiliser ces propriétés pour construire un algorithme permettant de générer une suite d'intervalles $(\ff{a_k,b_k})_{k\in\bb N}$ telle que
				\begin{itemize}
					\item $\ol x\in\ff{a_k,b_k}$
					
					\item $b_k-a_k=\frac{b_{k-1}-a_{k-1}}2$
					
					\item mis à part pour le premier pas, 2 évaluations de $f$ sont nécessaires à chaque itération.
				\end{itemize}
			\item Montrer que $a_k\to\ol x$ et $b_k\to \ol x$.
		\end{enumerate}
	\end{td-exo}
	
	\begin{td-sol}
		\begin{enumerate}
			\item On notera $\{x_1, x_2, x_3, x_4, x_5\}$ plutôt $\{a_i, y_i, z_i, t_i, b_i\}$ à la $i$-ème itération. 
				
				Pour n'évaluer $f$ que deux fois à chaque itération on regroupera également les cas $(i)$ et $(ii)$, soit
				\[\begin{aligned}
					\ul x \in\oo{a_k, z_k}
				\end{aligned}\]
				et les cas $(iv)$ et $(v)$, soit
				\[\begin{aligned}
					\ul x\in\oo{z_k,b_k}
				\end{aligned}\]
				le dernier cas est alors $(iii)$
				\[\begin{aligned}
					\ul x\in\oo{y_k,t_k}
				\end{aligned}\]
				
				Les valeurs deviennent alors dans le cas $(i)$ et $(ii)$,
				\[\begin{aligned}
					a_{k+1}&=a_k\\
					b_{k+1}&=z_k
				\end{aligned}\]
				dans le cas $(iii)$,
				\[\begin{aligned}
					a_{k+1}&=y_k\\
					b_{k+1}&=t_k
				\end{aligned}\]
				et dans le cas $(iv)$ et $(v)$,
				\[\begin{aligned}
					a_{k+1}&=z_k\\
					b_{k+1}&= b_k
				\end{aligned}\]
			
			\item On a $\frac{L_{k+1}}{L_k}=\frac12$ donc
				\[\begin{aligned}
					L_{k+1}&=\frac12L_k\\
					&=\left(\frac12\right)^{k+1}L_0
				\end{aligned}\]
			Soit que $L_k\underset{k\to\infty}\to 0$, et même que $(a_k)$ est strictement croissante, $(b_k)$ strictement décroissante et $\forall k,a_k<b_k$.
			
			$(a_k)$ et $(b_k)$ sont alors adjacentes et convergent vers la même limite.
			
			Par construction, $a_k\le l\le b_k$ pour tout $k$. Or, on sait que $\forall k,a_k\le \ul x\le b_k$ donc nécessairement $l=\ul x$.
		\end{enumerate}
	\end{td-sol}
	\medspace
	
	\begin{td-exo}[2. Méthode de la section dorée]
		Nous reprenons le principe de la méthode de la dichotomie précédente mais à chaque itération, nous allons maintenant chercher à diviser l'intervalle d'approximation en 3 parties (au lieu de 4 pour la dichotomie).\\
		
		Plus précisément, nous allons construire une suite décroissante d'intervalles $\ff{a_k,b_k}$ qui contiennent tous le minimum $\ol x$. Pour passer de $\ff{a_k,b_k}$ à $\ff{a_{k+1},b_{k+1}}$, on introduit deux nombres $x_2^k$ et $x_3^k$ de l'intervalle $\ff{a_k,b_k}$.\\
			
		On calcule alors les valeurs $f(x_2^k)$ et $f(x_3^k)$ et deux possibilités se présentent :
			\begin{enumerate}[label=$(\roman*)$]
				\item Si $f(x_2^k)\le f(x_3^k)$, alors le minimum se trouve nécessairement à gauche de $x_3^k$. Ceci définit alors le nouvel intervalle en posant $a_{k+1}=a_k$ et $b_{k+1}=x_3^k$.
				
				\item Si $f(x_2^k)\ge f(x_3^k)$, alors le minimum se trouve nécessairement à droite de $x_2^k$. Ceci définit alors le nouvel intervalle en posant $a_{k+1}=x_2^k$ et $b_{k+1}=b^k$.
			\end{enumerate}
		La question suivant se pose alors : comment choisir $x_2^k$ et $x_3^k$ en pratique ? Il faut privilégier deux aspects :
			\begin{enumerate}[label=$(\roman*)$]
				\item On souhaite que le facteur de réduction $\gamma$, qui représente le ratio de la longueur du nouvel intervalle, noté $L_{k+1}$, par rapport à la longueur du précédent, notée $L_k$, soit constant :
					\[\begin{aligned}
						\frac{L_{k+1}}{L_k}=\gamma
					\end{aligned}\]
				\item On désire, comme pour la méthode de la dichotomie, réutiliser le point qui n'a pas été choisi dans l'itération précédente afin de diminuer les coûts de calcul : ceci permettra de n'évaluer $f$ qu'une fois par itération au lieu de deux (sauf pour la première itération, où deux évaluations sont nécessaires). Rappelons que pour la dichotomie, il est nécessaire d'évaluer $f$ deux fois par itération.
			\end{enumerate}
		\begin{enumerate}
			\item Traduire ces contraintes permettant de choisir $x_2^k,x_3^k,a_{k+1},b_{k+1}$, proposer un algorithme et montrer qu'il n'y a qu'une seule valeur possible pour $\gamma$,
			
			\item Montrer que pour tout $k$, on a $b_k-a_k=\gamma^k(b-a)$. Conclure.
		\end{enumerate}
	\end{td-exo}
	
\end{document}