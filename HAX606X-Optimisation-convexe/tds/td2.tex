\documentclass[french,a4paper,10pt]{article}
\makeatletter
%--------------------------------------------------------------------------------
\usepackage[T1]{fontenc} % font type
\usepackage[french]{babel} % language
\usepackage{lmodern} % font type
\usepackage[shortlabels]{enumitem}
\setlist[itemize,1]{label={\color{gray}\small \textbullet}} % customises itemize default -
\usepackage{fancyhdr} % customises head and foot-notes
\usepackage{centernot} % allows centering \not with \centernot
\usepackage{stmaryrd} % allows \llbracket
\usepackage[overload]{abraces} % allows \aoverbrace

\usepackage{xcolor} % colour customisation, extends to tables with {colortbl}
\definecolor{astral}{RGB}{46,116,181}
\definecolor{verdant}{RGB}{96,172,128}
\definecolor{algebraic-amber}{RGB}{255,179,102} % definition colour
\definecolor{calculus-coral}{RGB}{255,191,191} % exercice colour
\definecolor{divergent-denim}{RGB}{130,172,211} % proposition colour 
\definecolor{matrix-mist}{RGB}{204,204,204} % remark colour
\definecolor{numeric-navy}{RGB}{204,204,204} % theorem colour 
\definecolor{quadratic-quartz}{RGB}{204,153,153} % example colour 


\usepackage{latexsym}
\usepackage{amsmath}
\usepackage{amsfonts}
\usepackage{amssymb}
\usepackage{amsthm}
\usepackage{mathtools}
\usepackage{mathrsfs}
\usepackage{MnSymbol}
\usepackage{etoolbox}% http://ctan.org/pkg/etoolbox

%\usepackage{tikz}
%\usepackage{pgfplots}
%\pgfplotsset{compat=1.18}
%\usetikzlibrary{arrows}


\newtheoremstyle{gen-style}{\topsep}{\topsep}%
{}%         Body font
{}%         Indent amount (empty = no indent, \parindent = para indent)
{\sffamily\bfseries}% Thm head font
{.}%        Punctuation after thm head
{ }%     Space after thm head (\newline = linebreak)
{\thmname{#1}\thmnumber{~#2}\thmnote{~#3}}%         Thm head spec


\newtheoremstyle{no-num-style}{\topsep}{\topsep}%
{}%         Body font
{}%         Indent amount (empty = no indent, \parindent = para indent)
{\sffamily\bfseries}% Thm head font
{.}%        Punctuation after thm head
{ }%     Space after thm head (\newline = linebreak)
{\thmname{#1}}%         Thm head spec


\usepackage[]{mdframed}

\newcommand{\mytheorem}[4]{%
	\newmdtheoremenv[
	hidealllines=true,
	leftline=true,
	skipabove=0pt,
	innertopmargin=-5pt,
	innerbottommargin=2pt,
	linewidth=4pt,
	innerrightmargin=0pt,
	linecolor=#3,
	]{#1}{#2}[#4]%
}


\newcommand{\mytheoremnocount}[3]{%
	\newmdtheoremenv[
	hidealllines=true,
	leftline=true,
	skipabove=0pt,
	innertopmargin=-5pt,
	innerbottommargin=2pt,
	linewidth=4pt,
	innerrightmargin=0pt,
	linecolor=#3,
	]{#1}{#2}%
}
\newcommand{\myoctheorem}[4]{%
	\newmdtheoremenv[
	hidealllines=true,
	leftline=true,
	skipabove=0pt,
	innertopmargin=-5pt,
	innerbottommargin=2pt,
	linewidth=4pt,
	innerrightmargin=0pt,
	linecolor=#3,
	]{#1}[#4]{#2}%
}

\theoremstyle{gen-style}
\mytheorem{proposition}{Proposition}{divergent-denim}{section}
\mytheorem{propdef}{Proposition - Définition}{divergent-denim}{section}
\mytheorem{theorem}{Théorème}{quadratic-quartz}{section}
\mytheorem{lemme}{Lemme}{quadratic-quartz}{section}
\mytheorem{example}{Exemple}{quadratic-quartz}{section}
\mytheorem{remark}{Remarque}{matrix-mist}{section}
\mytheorem{notation}{Notation}{matrix-mist}{section}
\mytheorem{exercise}{Exercice}{calculus-coral}{section}
\mytheorem{exercice}{Exercice}{calculus-coral}{section}
\mytheorem{definition}{Definition}{algebraic-amber}{section}

\newcounter{oc-counter}
\myoctheorem{oc-proposition}{Proposition}{divergent-denim}{oc-counter}
\myoctheorem{oc-propdef}{Proposition - Définition}{divergent-denim}{oc-counter}
\myoctheorem{oc-theorem}{Théorème}{divergent-denim}{oc-counter}
\myoctheorem{oc-lemme}{Lemme}{quadratic-quartz}{oc-counter}
\myoctheorem{oc-example}{Exemple}{quadratic-quartz}{oc-counter}
\myoctheorem{oc-remark}{Remarque}{matrix-mist}{oc-counter}
\myoctheorem{oc-exercise}{Exercice}{calculus-coral}{oc-counter}
\myoctheorem{oc-definition}{Definition}{algebraic-amber}{oc-counter}

\theoremstyle{no-num-style}
\mytheoremnocount{td-sol}{Solution}{verdant}
\mytheoremnocount{no-num-definition}{Definition}{algebraic-amber}
\mytheoremnocount{no-num-theorem}{Théorème}{algebraic-amber}
\mytheoremnocount{oc-intro}{Introduction}{quadratic-quartz}
\mytheoremnocount{oc-proof}{Preuve}{verdant}
\mytheoremnocount{oc-young}{Formule de Taylor à l'ordre 2}{verdant}
\mytheoremnocount{oc-notation}{Notation}{matrix-mist}
\mytheorem{rappel}{Rappel}{matrix-mist}{section}
\mytheoremnocount{myproof}{Preuve}{verdant}
\mytheoremnocount{td-exo}{Exercice}{calculus-coral}
\numberwithin{oc-counter}{subsection}

%---------------
% Mise en page
%--------------

\setlength{\parindent}{0pt}

\providecommand{\defemph}[1]{{\sffamily\bfseries\color{astral}#1}}


\usepackage{sectsty}
\allsectionsfont{\color{astral}\normalfont\sffamily\bfseries}

\usepackage{mathrsfs}

%----- Easy way to redeclare math operators -----
\makeatletter
\newcommand\RedeclareMathOperator{%
	\@ifstar{\def\rmo@s{m}\rmo@redeclare}{\def\rmo@s{o}\rmo@redeclare}%
}
\newcommand\rmo@redeclare[2]{%
	\begingroup \escapechar\m@ne\xdef\@gtempa{{\string#1}}\endgroup
	\expandafter\@ifundefined\@gtempa
	{\@latex@error{\noexpand#1undefined}\@ehc}%
	\relax
	\expandafter\rmo@declmathop\rmo@s{#1}{#2}}
\newcommand\rmo@declmathop[3]{%
	\DeclareRobustCommand{#2}{\qopname\newmcodes@#1{#3}}%
}
\@onlypreamble\RedeclareMathOperator
\makeatother

\newcommand{\skipline}{\vspace{\baselineskip}}
\newcommand{\noi}{\noindent}
%------------------------------------------------


\newcommand{\adh}[1]{\mathring{#1}} %adherence
\newcommand{\badh}[1]{\mathring{\overbrace{#1}}} % big adherence
\newcommand{\norm}{\mathcal{N}} % norme
\newcommand{\ol}[1]{\overline{#1}} % overline
\newcommand{\ul}[1]{\underline{#1}} % underline
\newcommand{\sub}{\subset} % subset
\newcommand{\scr}[1]{\mathscr{#1}} % scr rapide
\newcommand{\bb}[1]{\mathbb{#1}} % bb rapide
\newcommand{\bolo}[1]{B({#1}\mathopen{}[\mathclose{}} % boule ouverte
\newcommand{\bolf}[1]{B({#1}\mathopen{}]\mathclose{}} % boule fermee
\newcommand{\act}{\circlearrowleft} % agit sur
\newcommand{\glx}[1]{\text{GL}_{#1}} % GL_x
\newcommand{\cequiv}[1]{\mathopen{}[#1\mathclose{}]} % classe d'equivalence
\newcommand{\restr}[2]{#1\mathop{}\!|_{#2}} % restriction


%----- Intervalles -----
\newcommand{\oo}[1]{\mathopen{]}#1\mathclose{[}}
\newcommand{\of}[1]{\mathopen{]}#1\mathclose{]}}
\newcommand{\fo}[1]{\mathopen{[}#1\mathclose{[}}
\newcommand{\ff}[1]{\mathopen{[}#1\mathclose{]}}



\providecommand{\1}{\mathds{1}}
\DeclareMathOperator{\im}{\mathsf{Im}}
\DeclareRobustCommand{\re}{\mathsf{Re}}
\RedeclareMathOperator{\ker}{\mathsf{Ker}}
\RedeclareMathOperator{\det}{\mathsf{det}}
\DeclareMathOperator{\vect}{\mathsf{Vect}}
\DeclareMathOperator{\orb}{\mathsf{orb}}
\DeclareMathOperator{\st}{\mathsf{st}}
\DeclareMathOperator{\aut}{\mathsf{Aut}}
\DeclareMathOperator{\bij}{\mathsf{Bij}}
\DeclareMathOperator{\rank}{\mathsf{rank}}
\DeclareMathOperator{\tr}{\mathsf{tr}}
\DeclareMathOperator{\id}{\mathsf{Id}}
\providecommand{\B}{\mathsf{B}}


\providecommand{\dpar}[2]{\frac{\partial #1}{\partial #2}}
\makeatother

\usepackage[a4paper,hmargin=30mm,vmargin=30mm]{geometry}

\title{\color{astral} \sffamily \bfseries TD2 - Optimisation sans contraintes}
\author{Ivan Lejeune\thanks{Cours inspiré de M. Marche}}
\date{\today}
% pdflatex -output-directory=output chapter1.tex && move /Y output\chapter1.pdf .\

\begin{document}
	\maketitle
	\begin{td-exo}[1]
		Soit $A\in S_n(\bb R)$. On note $\lambda_1\le\dots\le\lambda_n$ ses valeurs propres. Montrer que pour tout $x\in\bb R^n$, on a
			\[\begin{aligned}
				x^TAx\ge \lambda_1||x||^2
			\end{aligned}\]
	\end{td-exo}
	\begin{td-sol}
		Comme $A$ appartient au groupe symétrique réel, on a
			\begin{enumerate}
				\item $A$ diagonalisable
				\item dans une base orthonormée
				\item valeurs propres réelles
			\end{enumerate}
			Donc il existe $P\in O_n(\bb R)$ tel que $A=P^TDP$ avec
				\[\begin{aligned}
					D = \begin{pmatrix}
						\lambda_1&0&\cdots&0&0\\
						0&\lambda_2&\cdots&0&0\\
						\vdots&\vdots&\ddots&\vdots&\vdots\\
						0&0&\cdots&\lambda_{n-1}&0\\
						0&0&\cdots&0&\lambda_n
					\end{pmatrix}
				\end{aligned}\]
				
			Soit $X\in \bb R^n$, on a
				\[\begin{aligned}
					X^TAX&=\underbrace{X^TP^T}_{=(PX^T)}DPX&\ge \lambda \min ||X||^2\\
					&=Y^TDY&\rotatebox{90}=\quad\\
					&=\sum_{i}\lambda_i y_i^2&\ge \min_i \lambda_i ||Y||^2
				\end{aligned}\]
	\end{td-sol}
	\medspace
	\begin{td-exo}[2]
		On définit la fonction 
			\[\begin{aligned}
				J&\colon&\bb R^2&\to&&\bb R\\
				&&(x,y)&\mapsto&& y^4-3xy^2+x^2
			\end{aligned}\]
			
		\begin{enumerate}
			\item Déterminer les points critiques de $J$.
			
			\item Soit $d=(d_1, d_2)\in\bb R^2$. En utilisant l'application $t\mapsto J(td_1,td_2)$ pour $t\in\bb R$, montrer que $(0,0)$ est un minimum local le long de toute droite passant par $(0,0)$.
			
			\item Le point $(0,0)$ est-il un minimum local de la restriction de $J$ à la parabole d'équation $x=y^2$ ?
			
			\item Calculer $f''$. Quelle est la nature du point critique $(0,0)$?
		\end{enumerate}
	\end{td-exo}
	\begin{td-sol}\,
		\begin{enumerate}
			\item On a
				\[\begin{aligned}
					\nabla J(x,y)=\begin{pmatrix}
						-3y^2+2x\\
						-6xy+4y^3
					\end{pmatrix}
				\end{aligned}\]
				Les points critiques sont les $(\ol x,\ol y)$ tels que $\nabla J(\ol x,\ol y)=0$.
				
				\[
				\begin{cases}
					-3y^2+2x=0\\
					-6xy+4y^3=0
				\end{cases}\Longleftrightarrow
				\begin{cases}
					x=\frac32y^2\\
					-9y^3+4y^3=0
				\end{cases}\Longleftrightarrow
				\begin{cases}
					x=0\\
					y=0
				\end{cases}
				\]
				
			\item Soit $d=(d_1,d_2)\in\bb R^2$. On pose 
				\[\begin{aligned}
					\psi\colon t\mapsto J(td_1,td_2)
				\end{aligned}\]
				
				On a 
					\[\begin{aligned}
						\psi(t)&=t^4d_2^4-3t^3d_1d_2^2+t^2d_1^2\\
						\psi'(t)&=4t^3d_2^4-9t^2d_1d_2^2+2td_1^2\\
						\psi''(t)&=12t^2d_2^4-18td_1d_2+2d_1^2
					\end{aligned}\]
					
				D'où $\psi''(0)=2d_1^2>0$ si $d_1\ne0$.
				
				Donc $\psi$ admet un minimum local en $t=0$.
				
				Si $d=0$ alors $x=td_1=0$ et $J(td_1,td_2)=(td_2)^4>0$ si $d_2\ne 0$.
				
				On a donc nécessairement
					\[\begin{aligned}
						J(td_1,td_2)\ge J(0,0)=0
					\end{aligned}\]
					
			\item  Posons $\varphi\colon y\mapsto J(y^2,y)$. On a alors
				\[\begin{aligned}
					\varphi(y)&=y^4-3y^4+y^4\\
					&=-y^4<0\text{ si }y\ne 0
				\end{aligned}\]
				Donc $(0,0)$ est un maximum local pour $\varphi$ et donc aussi pour $y\mapsto J(y^2,y)$, et $(0,0)$ est un maximum local pour la restriction de $J$ à la parabole $x=y^2$.
				
			\item On calcule
			\[\begin{aligned}
				J''(x,y)=\begin{pmatrix}
					-2&-6y\\-6y&12y^2-6x
				\end{pmatrix}
			\end{aligned}\]
			ce qui donne
			\[\begin{aligned}
				J''(0,0)=\begin{pmatrix}
					2&0\\0&0
				\end{pmatrix}
			\end{aligned}\]
			On constate que $J''$ a deux valeurs propres : $\lambda_1=2$ et $\lambda_2=0$.
			
			Le théorème du cours nous indique que comme $\lambda_2=0$, on ne peut pas conclure sans étude locale (faite précédemment, aux questions 2 et 3).
			
			On a donc affaire à un point selle.
		\end{enumerate}
		
	\end{td-sol}
	\medspace
	
	\begin{td-exo}[3]
		On considère la fonction
		\[\begin{aligned}
			f&\colon&\bb R^2&\to&&\bb R\\
			&&(x,y)&\mapsto&& x^4+y^4-2(x-y)^2
		\end{aligned}\]
		
		\begin{enumerate}
			\item Montrer qu'il existe $\alpha, \beta\in\bb R$ (et les déterminer) tels que
				\[\begin{aligned}
					\forall x,y\in \bb R, f(x,y)\ge \alpha||(x,y)||^2+\beta
				\end{aligned}\]
				
			\item Montrer que $\inf f(x,y)$ existe.
				
			\item La fonction $f$ est-elle convexe sur $\bb R^2$?
			
			\item Déterminer les points critiques de $f$ et préciser leur nature (minimum local, maximum local, point selle). Résoudre le problème $(2)$.
		\end{enumerate}
	\end{td-exo}
	\medspace
	
	\begin{td-sol}
		\[\begin{aligned}
			f(x,y)&=x^4+y^4-2(x-y)^2\\
			&=x^4+y^4-2x^2-2y^2+4xy
		\end{aligned}\]
	\end{td-sol}
	
	\begin{td-exo}[4]
		Soit $a\in\bb R$. On définit
		\[\begin{aligned}
			f_a&\colon&\bb R^2&\to&&\bb R\\
			&&(x,y)&\mapsto&& x^2+y^2+axy-2x-2y
		\end{aligned}\]
		\begin{enumerate}
			\item Pour quelles valeurs de $a$ la fonction $f_a$ est-elle convexe ? Et strictement convexe ?
			
			\item Discuter en fonction des valeurs du paramètre $a$ de l'existence de solutions à 
				\[\begin{aligned}
					\inf f_a(x,y)
				\end{aligned}\]
				
			\item Lorsque $a\in\oo{-2,2}$, résoudre ce problème.
		\end{enumerate}
		
	\end{td-exo}
	\medspace
	
	\begin{td-exo}[5]
		Soit $A\in S_n^+(\bb R)$. On considère la fonction
			\[\begin{aligned}
				f&\colon&\bb R^n\setminus\{0\}&\to&&\bb R\\
				&&x&\mapsto&& \frac{(Ax,x)}{||x||^2}
			\end{aligned}\]
		
		\begin{enumerate}
			\item Montrer que $f$ est $\scr C^\infty$ sur son ensemble de définition.
			
			\item Montrer que les problèmes d'optimisation
			\[\begin{aligned}
				\inf f(x)\qquad\text{et}\qquad\sup f(x)
			\end{aligned}\]
			possèdent une solution.
			
			\item Déterminer l'ensemble des points critiques de $f$.
			
			\item Résoudre les deux problèmes ci-dessus.
			
			\item Montrer qu'en un point critique $x^\ast\in\bb R^n\setminus\{0\}$, $f''$ est donnée par
				\[\begin{aligned}
					f''(x^\ast)=\frac2{||x^\ast||^2}\left(A-f(x^\ast)I_n\right)
				\end{aligned}\]
				
			\item En déduire que tous les points critiques qui ne sont pas solution d'un des problèmes au dessus sont des points-selles.
		\end{enumerate}
			
	\end{td-exo}
	
\end{document}