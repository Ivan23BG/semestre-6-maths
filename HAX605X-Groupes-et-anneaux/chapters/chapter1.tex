\documentclass[french,a4paper,10pt]{article}
\makeatletter
%--------------------------------------------------------------------------------
\usepackage[T1]{fontenc} % font type
\usepackage[french]{babel} % language
\usepackage{lmodern} % font type
\usepackage[shortlabels]{enumitem}
\setlist[itemize,1]{label={\color{gray}\small \textbullet}} % customises itemize default -
\usepackage{fancyhdr} % customises head and foot-notes
\usepackage{centernot} % allows centering \not with \centernot
\usepackage{stmaryrd} % allows \llbracket
\usepackage[overload]{abraces} % allows \aoverbrace

\usepackage{xcolor} % colour customisation, extends to tables with {colortbl}
\definecolor{astral}{RGB}{46,116,181}
\definecolor{verdant}{RGB}{96,172,128}
\definecolor{algebraic-amber}{RGB}{255,179,102} % definition colour
\definecolor{calculus-coral}{RGB}{255,191,191} % exercice colour
\definecolor{divergent-denim}{RGB}{130,172,211} % proposition colour 
\definecolor{matrix-mist}{RGB}{204,204,204} % remark colour
\definecolor{numeric-navy}{RGB}{204,204,204} % theorem colour 
\definecolor{quadratic-quartz}{RGB}{204,153,153} % example colour 


\usepackage{latexsym}
\usepackage{amsmath}
\usepackage{amsfonts}
\usepackage{amssymb}
\usepackage{amsthm}
\usepackage{mathtools}
\usepackage{mathrsfs}
\usepackage{MnSymbol}
\usepackage{etoolbox}% http://ctan.org/pkg/etoolbox

%\usepackage{tikz}
%\usepackage{pgfplots}
%\pgfplotsset{compat=1.18}
%\usetikzlibrary{arrows}


\newtheoremstyle{gen-style}{\topsep}{\topsep}%
{}%         Body font
{}%         Indent amount (empty = no indent, \parindent = para indent)
{\sffamily\bfseries}% Thm head font
{.}%        Punctuation after thm head
{ }%     Space after thm head (\newline = linebreak)
{\thmname{#1}\thmnumber{~#2}\thmnote{~#3}}%         Thm head spec


\newtheoremstyle{no-num-style}{\topsep}{\topsep}%
{}%         Body font
{}%         Indent amount (empty = no indent, \parindent = para indent)
{\sffamily\bfseries}% Thm head font
{.}%        Punctuation after thm head
{ }%     Space after thm head (\newline = linebreak)
{\thmname{#1}}%         Thm head spec


\usepackage[]{mdframed}

\newcommand{\mytheorem}[4]{%
	\newmdtheoremenv[
	hidealllines=true,
	leftline=true,
	skipabove=0pt,
	innertopmargin=-5pt,
	innerbottommargin=2pt,
	linewidth=4pt,
	innerrightmargin=0pt,
	linecolor=#3,
	]{#1}{#2}[#4]%
}


\newcommand{\mytheoremnocount}[3]{%
	\newmdtheoremenv[
	hidealllines=true,
	leftline=true,
	skipabove=0pt,
	innertopmargin=-5pt,
	innerbottommargin=2pt,
	linewidth=4pt,
	innerrightmargin=0pt,
	linecolor=#3,
	]{#1}{#2}%
}
\newcommand{\myoctheorem}[4]{%
	\newmdtheoremenv[
	hidealllines=true,
	leftline=true,
	skipabove=0pt,
	innertopmargin=-5pt,
	innerbottommargin=2pt,
	linewidth=4pt,
	innerrightmargin=0pt,
	linecolor=#3,
	]{#1}[#4]{#2}%
}

\theoremstyle{gen-style}
\mytheorem{proposition}{Proposition}{divergent-denim}{section}
\mytheorem{propdef}{Proposition - Définition}{divergent-denim}{section}
\mytheorem{theorem}{Théorème}{quadratic-quartz}{section}
\mytheorem{lemme}{Lemme}{quadratic-quartz}{section}
\mytheorem{example}{Exemple}{quadratic-quartz}{section}
\mytheorem{remark}{Remarque}{matrix-mist}{section}
\mytheorem{notation}{Notation}{matrix-mist}{section}
\mytheorem{exercise}{Exercice}{calculus-coral}{section}
\mytheorem{exercice}{Exercice}{calculus-coral}{section}
\mytheorem{definition}{Definition}{algebraic-amber}{section}

\newcounter{oc-counter}
\myoctheorem{oc-proposition}{Proposition}{divergent-denim}{oc-counter}
\myoctheorem{oc-propdef}{Proposition - Définition}{divergent-denim}{oc-counter}
\myoctheorem{oc-theorem}{Théorème}{divergent-denim}{oc-counter}
\myoctheorem{oc-lemme}{Lemme}{quadratic-quartz}{oc-counter}
\myoctheorem{oc-example}{Exemple}{quadratic-quartz}{oc-counter}
\myoctheorem{oc-remark}{Remarque}{matrix-mist}{oc-counter}
\myoctheorem{oc-exercise}{Exercice}{calculus-coral}{oc-counter}
\myoctheorem{oc-definition}{Definition}{algebraic-amber}{oc-counter}

\theoremstyle{no-num-style}
\mytheoremnocount{td-sol}{Solution}{verdant}
\mytheoremnocount{no-num-definition}{Definition}{algebraic-amber}
\mytheoremnocount{no-num-theorem}{Théorème}{algebraic-amber}
\mytheoremnocount{oc-intro}{Introduction}{quadratic-quartz}
\mytheoremnocount{oc-proof}{Preuve}{verdant}
\mytheoremnocount{oc-young}{Formule de Taylor à l'ordre 2}{verdant}
\mytheoremnocount{oc-notation}{Notation}{matrix-mist}
\mytheorem{rappel}{Rappel}{matrix-mist}{section}
\mytheoremnocount{myproof}{Preuve}{verdant}
\mytheoremnocount{td-exo}{Exercice}{calculus-coral}
\numberwithin{oc-counter}{subsection}

%---------------
% Mise en page
%--------------

\setlength{\parindent}{0pt}

\providecommand{\defemph}[1]{{\sffamily\bfseries\color{astral}#1}}


\usepackage{sectsty}
\allsectionsfont{\color{astral}\normalfont\sffamily\bfseries}

\usepackage{mathrsfs}

%----- Easy way to redeclare math operators -----
\makeatletter
\newcommand\RedeclareMathOperator{%
	\@ifstar{\def\rmo@s{m}\rmo@redeclare}{\def\rmo@s{o}\rmo@redeclare}%
}
\newcommand\rmo@redeclare[2]{%
	\begingroup \escapechar\m@ne\xdef\@gtempa{{\string#1}}\endgroup
	\expandafter\@ifundefined\@gtempa
	{\@latex@error{\noexpand#1undefined}\@ehc}%
	\relax
	\expandafter\rmo@declmathop\rmo@s{#1}{#2}}
\newcommand\rmo@declmathop[3]{%
	\DeclareRobustCommand{#2}{\qopname\newmcodes@#1{#3}}%
}
\@onlypreamble\RedeclareMathOperator
\makeatother

\newcommand{\skipline}{\vspace{\baselineskip}}
\newcommand{\noi}{\noindent}
%------------------------------------------------


\newcommand{\adh}[1]{\mathring{#1}} %adherence
\newcommand{\badh}[1]{\mathring{\overbrace{#1}}} % big adherence
\newcommand{\norm}{\mathcal{N}} % norme
\newcommand{\ol}[1]{\overline{#1}} % overline
\newcommand{\ul}[1]{\underline{#1}} % underline
\newcommand{\sub}{\subset} % subset
\newcommand{\scr}[1]{\mathscr{#1}} % scr rapide
\newcommand{\bb}[1]{\mathbb{#1}} % bb rapide
\newcommand{\bolo}[1]{B({#1}\mathopen{}[\mathclose{}} % boule ouverte
\newcommand{\bolf}[1]{B({#1}\mathopen{}]\mathclose{}} % boule fermee
\newcommand{\act}{\circlearrowleft} % agit sur
\newcommand{\glx}[1]{\text{GL}_{#1}} % GL_x
\newcommand{\cequiv}[1]{\mathopen{}[#1\mathclose{}]} % classe d'equivalence
\newcommand{\restr}[2]{#1\mathop{}\!|_{#2}} % restriction


%----- Intervalles -----
\newcommand{\oo}[1]{\mathopen{]}#1\mathclose{[}}
\newcommand{\of}[1]{\mathopen{]}#1\mathclose{]}}
\newcommand{\fo}[1]{\mathopen{[}#1\mathclose{[}}
\newcommand{\ff}[1]{\mathopen{[}#1\mathclose{]}}



\providecommand{\1}{\mathds{1}}
\DeclareMathOperator{\im}{\mathsf{Im}}
\DeclareRobustCommand{\re}{\mathsf{Re}}
\RedeclareMathOperator{\ker}{\mathsf{Ker}}
\RedeclareMathOperator{\det}{\mathsf{det}}
\DeclareMathOperator{\vect}{\mathsf{Vect}}
\DeclareMathOperator{\orb}{\mathsf{orb}}
\DeclareMathOperator{\st}{\mathsf{st}}
\DeclareMathOperator{\aut}{\mathsf{Aut}}
\DeclareMathOperator{\bij}{\mathsf{Bij}}
\DeclareMathOperator{\rank}{\mathsf{rank}}
\DeclareMathOperator{\tr}{\mathsf{tr}}
\DeclareMathOperator{\id}{\mathsf{Id}}
\providecommand{\B}{\mathsf{B}}


\providecommand{\dpar}[2]{\frac{\partial #1}{\partial #2}}
\makeatother

\usepackage[a4paper,hmargin=30mm,vmargin=30mm]{geometry}

\title{\color{astral} \sffamily \bfseries Groupes et anneaux II}
\author{Ivan Lejeune\thanks{Cours inspiré de M. Charlier et M. De Renzi}}
\date{\today}
% pdflatex -output-directory=output chapter1.tex && move /Y output\chapter1.pdf .\

\begin{document}
	\maketitle
	\section{Introduction}
	\subsection{Informations importantes}
	\begin{itemize}[$-$]
		\item CC1 : Début mars
		\item CC2 : Fin avril
		\item Examen final : En mai
	\end{itemize}
	
	\[\begin{aligned}
		\text{Note finale} = \max\left(\text{note finale}, \frac{\text{examen } + \frac{\text{CC1 }\text{CC2}}2}2\right)
	\end{aligned}\]
	
	\subsection{Références}
	
	Voir le cours de Joao Pedro dos Santos
	
	\section{Exemples importants de groupes}
	\begin{example}
		Le groupe des inversibles dans $\bb C$ avec la multiplication entre nombres complexes :
		
		\[\begin{aligned}
			\bb C=\bb C \setminus \{0\}
		\end{aligned}\]
		
		Le sous-groupe des racines $n$-ièmes de l'unité :
		\[\begin{aligned} 
			\mu_n\coloneq\{z\in\bb C^\times | z^n=1\}
		\end{aligned}\]
		
		\begin{remark}
			On obtient $\mu_n \eqsim \bb Z/n\bb Z$ via
			\[\begin{aligned}
				\bb Z/n\bb Z&\to \mu_n\\
				[k]&\mapsto e^{\frac{2k\pi i}n}
			\end{aligned}\]
		\end{remark}
		
	\end{example}
	\begin{example}
		Le groupe général linéaire des matrices inversibles de taille $n\times n$ à coefficients dans $\bb K$ avec multiplication matricielle :
		
		\[\begin{aligned}
			\glx{n}(\bb K)
		\end{aligned}\]
		
		\begin{remark}
			Si $\bb K = \bb F_p$ alors $|\glx{n}(\bb K)|$ est fini.
		\end{remark}
		
		Pour calculer $|\glx{n}(\bb F_p)|$, considérons $X\in \glx{n}(\bb F_p)$
		\[\begin{aligned}
			X &= \left(X_1 | X_2 | \cdots | X_n\right), X_i\in \bb F_p^n\\
			X_1 &\ne 0\\
			X_2 &\notin \bb F_p X_1\\
			X_3 &\notin \vect_{\bb F_p}(X_1, X_2)\\
			&\vdots
		\end{aligned}\]
		
		On a alors
		\[\begin{aligned}
			|\bb F_p^n\setminus\{0\}| &= p^n-1 \text{ choix pour }X_1\\
			|\bb F_p^n\setminus\bb F_pX_1| &= p^n-p \text{ choix pour }X_2\\
			|\bb F_p^n\setminus\vect_{\bb F_p}(X_1, X_2)| &= p^n-p^2 \text{ choix pour }X_3\\
			&\vdots
		\end{aligned}\]
		
		Soit
		\[\begin{aligned}
			|\glx{n}(\bb F_p)| = \prod_{i=0}^{n-1}(p^n-p^i)
		\end{aligned}\]
		
		En particulier, pour $p=n=2$, on a
		\[\begin{aligned}
			|\glx2(\bb F_2)| = (2^2-1)(2^2-2^1)=6
		\end{aligned}\]
	\end{example}
	\begin{example}
		Pour $n>1$, on a
		\begin{itemize}[]
			\item 
			$R\in \glx{2}(\bb R)$ la rotation d'angle $\frac{2\pi}n$ dans le sens anti-horlogique
			
			\item 
			$S\in \glx{2}(\bb R)$ réflexion par rapport à l'axe des abscisses. 
		\end{itemize}
			
			Si on identifie $\bb R^2\eqsim\bb C$, alors
			
			\[\begin{aligned}
				R(z)=e^{\frac{2\pi i}n} z,\quad S(z)=\ol z
			\end{aligned}\]
			
			On a alors 
			\[\begin{aligned}
				\forall k\in\bb Z, SR^kS&=R^{-k}
			\end{aligned}\]
			
			\begin{myproof}
				\[\begin{aligned}
					S(R^k(S(z)))&=S(R^k(\ol z))\\
					&=S(e^{\frac{2k\pi i}n}\ol z)\\
					&=e^{-\frac{2k\pi i}n}z\\
					&= R^{-k}(z)
				\end{aligned}\]
			\end{myproof}
			
			\begin{proposition}
			On note
			\[\begin{aligned}
				\mathfrak{D}_n\coloneq {I_2, R, \cdots, R^{n-1}}\\
			\end{aligned}\]
			le groupe diédral, alors
			\[\begin{aligned}
				\{S, RS, \cdots, R^{n-1}S\} \subset \mathfrak D_n
			\end{aligned}\]
			est un sous-groupe de $\glx{2}(\bb R)$ (le groupe diédral à $2n$ éléments)
			\end{proposition}
			\begin{myproof}
			
			\[\begin{aligned}
				R^iR^j&=R^{i+j}&\text{(a)}\\
				R^i(R^jS)&=R^{i+j}S\\
				(R^iS)R^j&=R^iSR^jSS=R^{i-j}S\\
				\underset{\substack{\Longrightarrow \mathfrak D_n\text{ est clos}\\ \text{par multiplication}}}{(R^iS)(R^jS)}&=R^{i-j}&\text{(b)}\\
				(R^i)^{-1}&=R^{-i}\text{ grâce à (a)}\\
				(R^iS)^{-1}&=R^iS\text{ grâce à (b)}
			\end{aligned}\]
			\end{myproof}
			
			\begin{remark}
				Si $n>2$, on a $\mathfrak{D}_n$ non commutatif
			\end{remark}
	\end{example}
	
	\section{Actions}
	
	Soit $X$ un ensemble et $G$ un groupe.
	
	
	
	\begin{definition}
		Une \defemph{action} de $G$ sur $X$ est une fonction
		\[\begin{aligned}
			G\times X &\to X\\
			(g,x)&\mapsto g\cdot x
		\end{aligned}\]
		telle que 
		\begin{enumerate}[label=$(\roman*)$]
			\item $\forall x\in X, e\cdot x = x$
			
			\item $\forall g,h\in G, \forall x\in X, (gh)\cdot x=g\cdot(h\cdot x)$
		\end{enumerate}
	\end{definition}
	
	
	
	\begin{remark}
		"$G$ agit sur $X$" sera noté $G\act X$.\\
		Un $G$-ensemble est un ensemble muni d'une action de $G$.
	\end{remark}
	
	
	\begin{exercice}
		\[\begin{aligned}
			G\act X\Longrightarrow \rho:G\to \bij(X)
		\end{aligned}\]
		Avec $\rho$ définie par 
		\[\begin{aligned}
			\rho(g)(x)\coloneq g\cdot x,\quad\forall g\in G, x\in X
		\end{aligned}\]
		Montrer que $\rho$ est un morphisme de groupes.\\
		
		Réciproquement, montrer que si $\rho\colon G\to \bij(X)$ est un morphisme de groupes, alors
		$g\cdot x\coloneq \rho(g)(x),\quad\forall g\in G, \forall x\in X$ définit une action de G.
	\end{exercice}
	
	\section{Exemples importants d'Actions}
	
	\begin{example}
		$\mathfrak S_n \act\{1, \dots, n\}$ naturellement ($\sigma\cdot k=\sigma(k))$
	\end{example}
	
	\begin{example}
		$\glx{n}{\bb K}\act\bb K^n$ par produit matriciel
		\[\begin{aligned}
			A\cdot v=Av\quad\forall A\in \glx{n}(\bb K),\forall v\in \bb K^n
		\end{aligned}\]
		($v$ vecteurs colonnes)
	\end{example}
	
	\begin{example}
		$\mathcal D_n\act\mu_n$ car
		\[\begin{aligned}
			\zeta^n=1\Longrightarrow g(\zeta)^n=1,\quad\forall g\in\mathcal D_n
		\end{aligned}\]
		En effet, il suffit de le vérifier pour les générateurs $R, S\in \mathcal D_n$ :
		\[\begin{aligned}
			R(\zeta)^n&=\left(e^{\frac{2\pi i}n}\zeta\right)^n\\
			&=e^{\frac{2n\pi i}n}\zeta^n\\&=1\\
			S(\zeta)^n&=\left(\ol\zeta\right)^n=\zeta^{-n}=1
		\end{aligned}\]
	\end{example}
	
	
	\begin{example}
		$H<G$ (notation pour "$H$ est un sous-groupe de $G$")
		\begin{enumerate}[(a)]
			\item 
			Action par translation à gauche : $H\act G$ par
			\[\begin{aligned}
				\rho_L\colon H&\to&&\bij(G)\\
				\rho_L(h)(g)&\coloneq& &hg&\forall h\in H, \forall g\in G
			\end{aligned}\]
			
			\item 
			Action par translation à droite : $H\act G$ par
			\[\begin{aligned}
				\rho_R\colon H&\to&&\bij(G)\\
				\rho_RL(h)(g)&\coloneq& &gh^{-1}&\forall h\in H, \forall g\in G
			\end{aligned}\]
			
		\end{enumerate}
		\begin{remark}
			Le réflexe $h\cdot g = gh$ ne définit pas une action en général
			\begin{exercice}
				Vérifier que ce n'est pas le cas uniquement si $H<Z(G)$.
			\end{exercice}
		\end{remark}
	\end{example}
	\begin{definition}
		Soient $X, Y, G$ des ensembles. On a $f\colon X\to Y$ $G$-\defemph{équivariante} (ou une $G$-fonction) si
			\[\begin{aligned}
				\forall g\in G,\forall x\in X,\quad f(g\cdot x)=g\cdot f(x)
			\end{aligned}\]
	\end{definition}
	\begin{exercice}
		Soit $H<G$ (un sous-groupe) de $G$ et \color{astral}\underline{\color{black}$G_L$}\color{black} (resp. \color{verdant}\underline{\color{black}$G_R$}\color{black}) l'ensemble $G$ munit de l'action de $H$ par \color{astral}\underline{\color{black}translation à gauche}\color{black} (resp. \color{verdant}\underline{\color{black}translation à droite}\color{black}).
		
		Montrer que
			\[\begin{aligned}
				{\_}^{-1}\colon&G_L&\to G_R\\
				&g&\mapsto g^{-1}
			\end{aligned}\]
		est une bijection $H$-équivalente
	\end{exercice}
	\begin{definition}
		On pose $G$ un groupe, $\Gamma$ un groupe et $V$ un espace vectoriel sur $\bb K$. Les assertions des points suivants sont équivalentes:
			\begin{enumerate}[label=$(\roman*)$]
				\item $G\act \Gamma$ par homomorphismes si 
					\begin{itemize}
						\item $\forall g\in G,\gamma,\gamma'\in\Gamma$ on a$ g\cdot(\gamma\gamma')=(g\cdot \gamma)(g\cdot \gamma')$
						\item $\rho\colon G\to\aut(\Gamma)<\mathfrak S$
						\item $\rho(g)$ est un morphisme de groupes $\forall g\in G$
					\end{itemize}
				
				\item $G\act V$ de manière linéaire si 
					\begin{itemize}
						\item $\forall g\in G,v,v'\in V,\lambda,\lambda'\in\bb K$ on a $g\cdot(\lambda v\lambda'v')=\lambda(g\cdot v)+\lambda'(g\cdot v')$
						\item $\rho\colon G\to \text{GL}_{\bb K}(V)<\mathfrak{S}_V$
						\item $\rho(g)$ est une application linéaire $\forall g\in G$
					\end{itemize}
			\end{enumerate}
	\end{definition}
	\begin{example}
		Soient $H$ et $G$ des groupes avec $H<G$. Les assertions suivantes sont équivalentes :
		\begin{itemize}
			\item L'action de $H$ sur $G$ par translation à gauche est une action par homomorphismes
			\item $H=\{e\}$
		\end{itemize}
		En effet :
			\[\begin{aligned}
				\forall h\in H\quad
				&h\cdot(gg')=(h\cdot g)(h\cdot g')\\
				\Longleftrightarrow &hgg'=hghg'\\
				\Longleftrightarrow& (hg)^{-1}hgg'(g')^{-1}=(hg)^{-1}hghg'(g')^{-1}\\
				\Longleftrightarrow&e=h
			\end{aligned}\]
		Soit que $\forall h\in H,h=e$ et donc $H=\{e\}$
	\end{example}
	\begin{example}
		L'action de $\glx{n}{\bb K}$ sur $\bb K^n$ qu'on a vu l'autre fois est linéaire.
	\end{example}
	\begin{example}
		L'action par conjugaison :
		
		Soient $H$ et $G$ des groupes, si $H<G$ alors $H\act G$ par $\rho_C\colon H\to\aut(G)<\mathfrak{S}_G$ avec
			\[\begin{aligned}
				\rho_C(h)(g)\coloneq hgh^{-1}\quad\forall h\in H,g\in G
			\end{aligned}\]
		Il s'agit d'une action par homomorphismes car
			\[\begin{aligned}
				h\cdot(gg')=hgg'h^{-1}&=hgh^{-1}hg'h^{-1}\\
				&=(h\cdot g)(h\cdot g')\\
				&=\rho_C(h)(g) \rho_C(h)(g')\\&\quad\forall h\in H,g,g'\in G
			\end{aligned}\]
	\end{example}
	\begin{theorem}[Cayley]\,\\
		Si $G$ est un groupe d'ordre $n\in N$ alors $G$ est isomorphe à un sous-groupe de $\mathfrak S_n$.
	\end{theorem}
	\begin{myproof}
		$G$ agit sur lui-même par translation à gauche par 
			\[\begin{aligned}
				\rho_L\colon G\to \mathfrak S_G\simeq\mathfrak S_n
			\end{aligned}\]
			\[
			g\in \ker(\rho_L)\Longrightarrow\rho_L(g)(e)=(e)\Longrightarrow ge=e
			\]
			Donc $\rho_L$ est injectif et $\rho_L\colon G\to \im(\rho_L)<\mathfrak S_n$ est un isomorphisme.
	\end{myproof}
	\begin{example}
		On considère $\zeta=e^{\frac{2\pi i}5}$ et 
			\[
				\mu_5=\{\zeta^1,\zeta^2,\dots,\zeta^5\}\simeq\{1,2,\dots,5\}
			\]
			\[\begin{aligned}
				\rho_L\colon&\mu_5&\to &\mathfrak S_{\mu_5}\simeq\mathfrak S_5\\
				&\zeta^k&\mapsto &(1~2~3~4~5)^k
			\end{aligned}\]
	\end{example}
	\begin{definition}
		Soit $G$ un groupe et $X$ un ensemble avec $G\act X$. On a alors
		\begin{enumerate}[label=$(\roman*)$]
			\item $Y\in X$ est \defemph{stable} par $G$ si $G\cdot Y\sub Y$. En particulier, $Y$ est un sous-$G$ ensemble.
			
			\item L'\defemph{orbite} de $x\in X$ notée $\orb(x)$ ou $G\cdot x$ est
				\[\begin{aligned}
					G\cdot x = \{g\cdot x\mid g\in G\}
				\end{aligned}\]
				\begin{remark}
					$\orb(x)$ est stable par $G$.
				\end{remark}
			
			\item Le \defemph{stabilisateur} de $x\in X$ est $\st(x)$ ou $G_x$ est 
				\[\begin{aligned}
					G_x=\{g\in G\mid g\cdot x=x\}
				\end{aligned}\]
				\begin{remark}
					$\st(x)<G,e\in\st(x)$
					\[\begin{aligned}
						g,g'\in \st(x)\Longrightarrow g\cdot(g'\cdot x)=g\cdot x=x
					\end{aligned}\]
				\end{remark}
			\item $x\in X$ est un \defemph{point fixe} si 
				\[\begin{aligned}
					g\cdot x=x\quad \forall g\in G
				\end{aligned}\]
				Soit, si 
					\[\begin{aligned}
						G_x=G\qquad \text{ou}\qquad G\cdot x=x
					\end{aligned}\]
				L'ensemble des points fixes est noté $X^G$.
				
			\item L'action est \defemph{transitive} si 
				\[\begin{aligned}
					\exists x\in X\colon G\cdot x=X
				\end{aligned}\]
				Dans ce cas, on dit que $X$ est un \defemph{espace homogène}
				
			\item L'action est \defemph{libre} si 
				\[\begin{aligned}
					\forall x\in X, G_x=\{e\}
				\end{aligned}\]
		\end{enumerate}
	\end{definition}
\end{document}