\documentclass[french,a4paper,10pt]{article}
\makeatletter
%--------------------------------------------------------------------------------
\usepackage[T1]{fontenc} % font type
\usepackage[french]{babel} % language
\usepackage{lmodern} % font type
\usepackage[shortlabels]{enumitem}
\setlist[itemize,1]{label={\color{gray}\small \textbullet}} % customises itemize default -
\usepackage{fancyhdr} % customises head and foot-notes
\usepackage{centernot} % allows centering \not with \centernot
\usepackage{stmaryrd} % allows \llbracket
\usepackage[overload]{abraces} % allows \aoverbrace

\usepackage{xcolor} % colour customisation, extends to tables with {colortbl}
\definecolor{astral}{RGB}{46,116,181}
\definecolor{verdant}{RGB}{96,172,128}
\definecolor{algebraic-amber}{RGB}{255,179,102} % definition colour
\definecolor{calculus-coral}{RGB}{255,191,191} % exercice colour
\definecolor{divergent-denim}{RGB}{130,172,211} % proposition colour 
\definecolor{matrix-mist}{RGB}{204,204,204} % remark colour
\definecolor{numeric-navy}{RGB}{204,204,204} % theorem colour 
\definecolor{quadratic-quartz}{RGB}{204,153,153} % example colour 


\usepackage{latexsym}
\usepackage{amsmath}
\usepackage{amsfonts}
\usepackage{amssymb}
\usepackage{amsthm}
\usepackage{mathtools}
\usepackage{mathrsfs}
\usepackage{MnSymbol}
\usepackage{etoolbox}% http://ctan.org/pkg/etoolbox

%\usepackage{tikz}
%\usepackage{pgfplots}
%\pgfplotsset{compat=1.18}
%\usetikzlibrary{arrows}


\newtheoremstyle{gen-style}{\topsep}{\topsep}%
{}%         Body font
{}%         Indent amount (empty = no indent, \parindent = para indent)
{\sffamily\bfseries}% Thm head font
{.}%        Punctuation after thm head
{ }%     Space after thm head (\newline = linebreak)
{\thmname{#1}\thmnumber{~#2}\thmnote{~#3}}%         Thm head spec


\newtheoremstyle{no-num-style}{\topsep}{\topsep}%
{}%         Body font
{}%         Indent amount (empty = no indent, \parindent = para indent)
{\sffamily\bfseries}% Thm head font
{.}%        Punctuation after thm head
{ }%     Space after thm head (\newline = linebreak)
{\thmname{#1}}%         Thm head spec


\usepackage[]{mdframed}

\newcommand{\mytheorem}[4]{%
	\newmdtheoremenv[
	hidealllines=true,
	leftline=true,
	skipabove=0pt,
	innertopmargin=-5pt,
	innerbottommargin=2pt,
	linewidth=4pt,
	innerrightmargin=0pt,
	linecolor=#3,
	]{#1}{#2}[#4]%
}


\newcommand{\mytheoremnocount}[3]{%
	\newmdtheoremenv[
	hidealllines=true,
	leftline=true,
	skipabove=0pt,
	innertopmargin=-5pt,
	innerbottommargin=2pt,
	linewidth=4pt,
	innerrightmargin=0pt,
	linecolor=#3,
	]{#1}{#2}%
}
\newcommand{\myoctheorem}[4]{%
	\newmdtheoremenv[
	hidealllines=true,
	leftline=true,
	skipabove=0pt,
	innertopmargin=-5pt,
	innerbottommargin=2pt,
	linewidth=4pt,
	innerrightmargin=0pt,
	linecolor=#3,
	]{#1}[#4]{#2}%
}

\theoremstyle{gen-style}
\mytheorem{proposition}{Proposition}{divergent-denim}{section}
\mytheorem{propdef}{Proposition - Définition}{divergent-denim}{section}
\mytheorem{theorem}{Théorème}{quadratic-quartz}{section}
\mytheorem{lemme}{Lemme}{quadratic-quartz}{section}
\mytheorem{example}{Exemple}{quadratic-quartz}{section}
\mytheorem{remark}{Remarque}{matrix-mist}{section}
\mytheorem{notation}{Notation}{matrix-mist}{section}
\mytheorem{exercise}{Exercice}{calculus-coral}{section}
\mytheorem{exercice}{Exercice}{calculus-coral}{section}
\mytheorem{definition}{Definition}{algebraic-amber}{section}

\newcounter{oc-counter}
\myoctheorem{oc-proposition}{Proposition}{divergent-denim}{oc-counter}
\myoctheorem{oc-propdef}{Proposition - Définition}{divergent-denim}{oc-counter}
\myoctheorem{oc-theorem}{Théorème}{divergent-denim}{oc-counter}
\myoctheorem{oc-lemme}{Lemme}{quadratic-quartz}{oc-counter}
\myoctheorem{oc-example}{Exemple}{quadratic-quartz}{oc-counter}
\myoctheorem{oc-remark}{Remarque}{matrix-mist}{oc-counter}
\myoctheorem{oc-exercise}{Exercice}{calculus-coral}{oc-counter}
\myoctheorem{oc-definition}{Definition}{algebraic-amber}{oc-counter}

\theoremstyle{no-num-style}
\mytheoremnocount{td-sol}{Solution}{verdant}
\mytheoremnocount{no-num-definition}{Definition}{algebraic-amber}
\mytheoremnocount{no-num-theorem}{Théorème}{algebraic-amber}
\mytheoremnocount{oc-intro}{Introduction}{quadratic-quartz}
\mytheoremnocount{oc-proof}{Preuve}{verdant}
\mytheoremnocount{oc-young}{Formule de Taylor à l'ordre 2}{verdant}
\mytheoremnocount{oc-notation}{Notation}{matrix-mist}
\mytheorem{rappel}{Rappel}{matrix-mist}{section}
\mytheoremnocount{myproof}{Preuve}{verdant}
\mytheoremnocount{td-exo}{Exercice}{calculus-coral}
\numberwithin{oc-counter}{subsection}

%---------------
% Mise en page
%--------------

\setlength{\parindent}{0pt}

\providecommand{\defemph}[1]{{\sffamily\bfseries\color{astral}#1}}


\usepackage{sectsty}
\allsectionsfont{\color{astral}\normalfont\sffamily\bfseries}

\usepackage{mathrsfs}

%----- Easy way to redeclare math operators -----
\makeatletter
\newcommand\RedeclareMathOperator{%
	\@ifstar{\def\rmo@s{m}\rmo@redeclare}{\def\rmo@s{o}\rmo@redeclare}%
}
\newcommand\rmo@redeclare[2]{%
	\begingroup \escapechar\m@ne\xdef\@gtempa{{\string#1}}\endgroup
	\expandafter\@ifundefined\@gtempa
	{\@latex@error{\noexpand#1undefined}\@ehc}%
	\relax
	\expandafter\rmo@declmathop\rmo@s{#1}{#2}}
\newcommand\rmo@declmathop[3]{%
	\DeclareRobustCommand{#2}{\qopname\newmcodes@#1{#3}}%
}
\@onlypreamble\RedeclareMathOperator
\makeatother

\newcommand{\skipline}{\vspace{\baselineskip}}
\newcommand{\noi}{\noindent}
%------------------------------------------------


\newcommand{\adh}[1]{\mathring{#1}} %adherence
\newcommand{\badh}[1]{\mathring{\overbrace{#1}}} % big adherence
\newcommand{\norm}{\mathcal{N}} % norme
\newcommand{\ol}[1]{\overline{#1}} % overline
\newcommand{\ul}[1]{\underline{#1}} % underline
\newcommand{\sub}{\subset} % subset
\newcommand{\scr}[1]{\mathscr{#1}} % scr rapide
\newcommand{\bb}[1]{\mathbb{#1}} % bb rapide
\newcommand{\bolo}[1]{B({#1}\mathopen{}[\mathclose{}} % boule ouverte
\newcommand{\bolf}[1]{B({#1}\mathopen{}]\mathclose{}} % boule fermee
\newcommand{\act}{\circlearrowleft} % agit sur
\newcommand{\glx}[1]{\text{GL}_{#1}} % GL_x
\newcommand{\cequiv}[1]{\mathopen{}[#1\mathclose{}]} % classe d'equivalence
\newcommand{\restr}[2]{#1\mathop{}\!|_{#2}} % restriction


%----- Intervalles -----
\newcommand{\oo}[1]{\mathopen{]}#1\mathclose{[}}
\newcommand{\of}[1]{\mathopen{]}#1\mathclose{]}}
\newcommand{\fo}[1]{\mathopen{[}#1\mathclose{[}}
\newcommand{\ff}[1]{\mathopen{[}#1\mathclose{]}}



\providecommand{\1}{\mathds{1}}
\DeclareMathOperator{\im}{\mathsf{Im}}
\DeclareRobustCommand{\re}{\mathsf{Re}}
\RedeclareMathOperator{\ker}{\mathsf{Ker}}
\RedeclareMathOperator{\det}{\mathsf{det}}
\DeclareMathOperator{\vect}{\mathsf{Vect}}
\DeclareMathOperator{\orb}{\mathsf{orb}}
\DeclareMathOperator{\st}{\mathsf{st}}
\DeclareMathOperator{\aut}{\mathsf{Aut}}
\DeclareMathOperator{\bij}{\mathsf{Bij}}
\DeclareMathOperator{\rank}{\mathsf{rank}}
\DeclareMathOperator{\tr}{\mathsf{tr}}
\DeclareMathOperator{\id}{\mathsf{Id}}
\providecommand{\B}{\mathsf{B}}


\providecommand{\dpar}[2]{\frac{\partial #1}{\partial #2}}
\makeatother

\usepackage[a4paper,hmargin=30mm,vmargin=30mm]{geometry}

\title{\color{astral} \sffamily \bfseries Feuilles de TD}
\author{Ivan Lejeune\thanks{Cours inspiré de M. Charlier et M. De Renzi}}
\date{\today}
% pdflatex -output-directory=output chapter1.tex && move /Y output\chapter1.pdf .\

\begin{document}
	\maketitle
	\section{TD1}
	\begin{example}
		On considère $G=\glx2(\bb K)$ et son action sur $\bb K^n$. Pour $n=2$, la base standard de $\bb K^2$ est $\{e_1, e_2\}$. Alors, $G\cdot e_1=\bb K^2\setminus\{0\}$ car
		\[\begin{aligned}
			\underset{\in G\cdot e_1}{
				\begin{pmatrix*}
					x\\
					y
			\end{pmatrix*}}
			=
			\begin{cases}
				\begin{pmatrix*}
					x&0\\
					y&1
				\end{pmatrix*}
				\cdot e_1&\text{si }x\ne 0\\
				
				\begin{pmatrix*}
					x&1\\
					y&0
				\end{pmatrix*}
				\cdot e_1&\text{si }y\ne 0
			\end{cases}
		\end{aligned}\]
		Alors $G\cdot 0=\{0\}$. L'action est donc transitive sur $\bb K^2\setminus\{0\}$ et $0$ est l'unique point fixe. De plus
		\[\begin{aligned}
			G_{e_1}= \left\{
			\begin{pmatrix*}
				1&b\\0&d
			\end{pmatrix*}
			\bigg{\mid}~
			\begin{matrix}
				b,d\in \bb K\\
				d\ne 0
			\end{matrix}
			\right\}
		\end{aligned}\]
		Cela implique $\bb K^2$ n'est pas homogène (l'action n'est pas transitive) et l'action n'est pas libre.
		\begin{definition}
			Soit $\bb P^{n-1}(\bb K)$ l'\defemph{espace projectif de dimension $n-1$}. C'est l'ensemble des droites vectorielles de $\bb K^n$.
			
			De manière équivalente, $\bb P^{n-1}(\bb K)$ est le quotient de $\bb K^n\setminus\{0\}$ par la relation d'équivalence
			\[\begin{aligned}
				v\sim v'\Longleftrightarrow\exists\lambda\in \bb K\setminus\{0\}\colon v'=\lambda v
			\end{aligned}\]
			Si $\sim$ est une relation d'équivalence sur un ensemble $X$, alors $X/\sim$ est l'ensemble dont les éléments sont les classes d'équivalence de $\sim$ dans $X$.
			
			Si $x\in X$, sa classe d'équivalence notée $\cequiv{x}\in X/\sim$ correspond à $\{x'\in X\mid x'\sim x\}$
		\end{definition}\,\\
		$G$ agit sur $\bb P^{n-1}(\bb K)$ par 
		\[\begin{aligned}
			A\cdot\cequiv{v}=\cequiv{Av}\quad\forall A\in G,v\in \bb K^n
		\end{aligned}\]
	\end{example}
	
	\medspace
	\begin{td-exo}
		Considérer les actions suivantes :
		\begin{enumerate}[label=$(\roman*)$]
			\item L'action de $\mathfrak{S}_n$ sur $\{1,\dots,n\}$;
			\item L'action de $\glx2(\bb K)$ sur $\bb P^1(\bb K)$;
			\item L'action de $\scr D_n$ sur $\mu_n$;
			\item L'action de $\mathfrak{S}_n$ par conjugaison sur ses sous-groupes d'ordre 2.
		\end{enumerate}
		Ces actions sont-elles libres ? Sont-elles transitives
	\end{td-exo}
	\begin{td-sol}
		\,
		\begin{enumerate}[label=$(\roman*)$]
			\item Si $n>2$, l'action n'est pas libre car 
			\[\begin{aligned}
				\st(n)=\mathfrak{S}_{n-1}<\mathfrak{S}_n
			\end{aligned}\]
			L'action est transitive car on a
			\[\begin{aligned}
				i = (i~j)\cdot j\quad\forall 1\le i,j\le n
			\end{aligned}\]
			Si $n\le 2$, alors l'action est libre et transitive.
			\item L'action est transitive car l'action de $G$ sur $\bb K^2\setminus\{0\}$ l'est. Donc 
			\[\begin{aligned}
				\forall v\in \bb K^2\setminus\{0\},\exists A\in G\colon Ae_1=v
			\end{aligned}\]
			Soit que 
			\[\begin{aligned}
				A\cdot\cequiv{e_1} = \cequiv{A\cdot e_1} = \cequiv{v}
			\end{aligned}\]
			Il suffit de considérer le même $A$ qu'avant $($pour un choix quelconque d'un représentant $v$ de $\cequiv{v})$.
			
			Il vient $G\cdot\cequiv{e_1}=\bb P^1(\bb K)$ et
			\[\begin{aligned}
				G_{\cequiv{e_1}}=
				\begin{cases}
					\begin{pmatrix*}
						a&b\\
						0&d
					\end{pmatrix*}
					\bigg \mid~
					\begin{matrix*}
						a,b,d\in \bb K\\
						ad\ne 0
					\end{matrix*}
				\end{cases}
			\end{aligned}\]
			Donc l'action n'est pas libre
			
			\item \,
			\begin{rappel}
				On a
				\[\begin{aligned}
					&R\cdot\zeta=e^{\frac{2\pi i}n}\zeta\\
					&S\cdot\zeta=\ol\zeta
				\end{aligned}\]
				Avec
				\[\begin{aligned}
					R&=
					\begin{pmatrix*}
						\cos\frac{2\pi}n&-\sin\\
						\sin\frac{2\pi}n&\cos
					\end{pmatrix*}\\
					S&=
					\begin{pmatrix*}
						1&0\\
						0&-1
					\end{pmatrix*}
				\end{aligned}\]
				
				\[\begin{aligned}
					\scr D_n&\to\mathfrak{S}_{\mu_n}\\
					R^j&\mapsto\left[\zeta^k\mapsto e^{\frac{2j\pi i}n}\zeta^k\right]\\
					R^jS&\mapsto\left[\zeta^k\mapsto e^{\frac{2j\pi i}n}\zeta^{-k}\right]
				\end{aligned}\]
			\end{rappel}
			
			L'action de $\scr D_n$ sur $\mu_n$ n'est pas libre car $S\in\st(1)$ et donc $S\cdot 1=\ol 1=1$.
			
			L'action est transitive car $\zeta^k=R^k\cdot 1\quad\forall \zeta^k\in\mu_n$.
			
			\item Si $H<\mathfrak{S}_n$ est un sous-groupe d'ordre 2, c'est que $H=\{\id, \sigma\}$ pour tout $\sigma\in \mathfrak{S}_n$.
			
			On a donc $\sigma\in\st(\{1, \sigma\})$ pour $\sigma$ d'ordre 2 car
			\[\begin{aligned}
				\left\{\sigma 1\sigma^{-1},\sigma\sigma\sigma^{-1}\right\}=\left\{1, \sigma\right\}
			\end{aligned}\]
			Pour $\sigma = (i~j)$ et $\phi\in\mathfrak{S}_n$ quelconque, on a
			\[\begin{aligned}
				\phi\circ\sigma\circ\phi^{-1}=\left(\phi(i)~\phi(j)\right)
			\end{aligned}\]
			Donc $<(1~2)(3~4)>\notin\orb(<1~2>)$ pour $n\ge 4$.
			
			En conclusion l'action n'est jamais libre, et elle n'est pas transitive pour $n\ge 4$.
		\end{enumerate}
	\end{td-sol}
	\medspace
	\begin{td-exo}
		Soit $||\_||$ la norme euclidienne usuelle de $\bb R^n$ et soit $0_n$ le sous-groupe de $\glx n(\bb R)$ dont les éléments sont les matrices $A\in\glx n(\bb R)$ telles que $||Av||=||v||$ pour tout $v\in\bb R^n$. On appelle $0_n$ le \defemph{groupe orthogonal de dimension $n$}. Soit $S^{n-1}\coloneq \{v\in \bb R^n\mid||v||=1\}$ la \defemph{sphère de dimension $n-1$}. Montrer que $O_n$ agit transitivement, mais pas librement sur $S^{n-1}$ pour tout $n>1$.
	\end{td-exo}
	
	\begin{td-sol}
		Soit $n \geq 2$
		
		$O_n = \{A\in \glx n (\bb R)\ \mid~ A^{-1} = A^T\}$
		
		$O_n \act S^{n-1}$, l'action de $O_n$ sur $S^{n-1}$ n'est pas libre, car 
		
		\[\begin{aligned}
			i: O_{n-1} \to O_n\\
			A \mapsto \begin{pmatrix*}
				A&\begin{matrix*} 0 \\ \vdots\\ 0\end{matrix*}\\
				0 \hdots 0 &1
			\end{pmatrix*}
		\end{aligned}\]
		
		$i(O_{n-1}) \subset \st{\begin{pmatrix*} 0 \\ \vdots\\ 0 \\ 1\end{pmatrix*}}$
		
		$\supset$ 
		Soit $x\in S^{n-1}$. On considère la base $\mathcal{B} = (e_1, e_2, \dots, e_n)$ canonique. $x \neq 0_{\bb R^n}$,
		on peut construire une base $\mathcal{B}' = (x, e_1', e_2', \dots, e_n')$ par Gram-Schmidt on peut
		l'orthogonaliser et obtenir $\mathcal{B}'' = (x, e_1'', e_2'', \dots, e_n'')$. Soit 
		$A$ la matrice de passage entre $\mathcal{B}''$ et $\mathcal{B}$, $x = A\cdot e_1$, $A$
		est orthogonale car $A^T A = (\langle e_i'', e_j''\rangle)_{i,j} = (S_{i,j})_{i,j}$
	\end{td-sol}
	
	\begin{td-exo}
		L'action de $\scr D_n$ sur les racines de l'unité $\mu_n$ est-elle une action 
		par morphismes de groupe ?
	\end{td-exo}
	\begin{td-sol}
		$\scr D_n$ n'agit pas par homomorphisme sur $\mu_n$. En effet, si 
		$\zeta = e^{\frac{2\pi i}{n}}$, alors $R(1) = \zeta$ donc $R$ n'agit pas 
		par homomorphisme. $R(\zeta^k) = \zeta^{k+1}$ $R(\zeta^2) = \zeta^3 \neq \zeta^4
		= R(\zeta)R(\zeta)$ 
	\end{td-sol}
	\begin{td-exo}
		Soit $H < G$ un sous-groupe. Soit $G_L$ l'ensemble $G$ muni de l'action 
		par translation à gauche de $H$, à savoir, $\rho_L(h)(g) = hg$ pour 
		tout $h\in H$ et $g\in G_L$. Soit $G_R$ l'ensemble $G$ muni de l'action 
		par translation à droite de $H$, à savoir, $\rho_R(h)(g) = gh^{-1}$ pour 
		tout $h\in H$ et $G\in G_R$. Montrer que la fonction ${\_}^{-1}: G_L \to G_R$ 
		est une bijection $H-$équivariante.
	\end{td-exo}
	\begin{td-sol}
		(Déjà fait en cours) "homomorphisme" = "morphisme de groupes".
	\end{td-sol}
	\begin{td-exo}
		En utilisant l'action de $\glx{2}(\bb F_2)$ sur l'espace projectif $\bb P^1(\bb F_2)$ 
		montrer que $\glx{2}(\bb F_2) \cong \mathfrak{S}_3$. Similairement, utiliser
		l'action de $\scr D_3$ sur $\mu_3$ pour montrer que $\scr D_3 \cong \mathfrak{S}_3$.
	\end{td-exo}
	\begin{td-sol}
		$\bb P^1(\bb F_2) = \{[e_1], [e_2], [e_1+e_2]\}$
		
		$\glx{2}(\bb F^2) \act \bb P^1(\bb F_2)$ par $\rho: \glx{2}(\bb F_2) \to \mathfrak{S}_{\bb P^1(\bb F_2)} \cong
		\mathfrak{S}_3$
		
		$A\in \ker \rho \implies A[e_1] = [e_1]
		\implies A[e_2] = [e_2] \implies A = \id_2 \implies \rho$ injectif.
		
		Mais $| \glx{2}(\bb F_2) | = | \mathfrak{S}_{\bb P^1(\bb F_2)} | = 6$
		$\implies \rho$ isomorphe.
		
		$
		\begin{pmatrix}
			0 & 1 \\ 1 & 0
		\end{pmatrix} \to \begin{pmatrix}
			1 & 2
		\end{pmatrix}
		$
		
		$
		\begin{pmatrix}
			1 & 1 \\ 0 & 1
		\end{pmatrix} \to \begin{pmatrix}
			2 & 3
		\end{pmatrix}
		$
		
		$
		\begin{pmatrix}
			0 & 1 \\ 1 & 1
		\end{pmatrix} \to \begin{pmatrix}
			1 & 2 & 3
		\end{pmatrix}
		$
		
		$\mu_3 = \{1,\zeta,\zeta^2\}$ avec $\zeta = e^{\frac{2\pi i}{3}}$.
		$\scr D \act \mu_3$ par $\rho: \scr D_3 \to \mathfrak{S}_{\mu_3} \cong \mathfrak{S}_{3}$
		
		$R^l S^i$, $\zeta^{l+k} = \begin{cases}\zeta^{l+k} \textrm{ si } j = 0\\
			\zeta^{l-k} \textrm{ si } j = 1 \end{cases}$
		
		avec $0 \leq l \leq 2$ et $0\leq j\leq 1$. Donc $R^lS_j\in \ker \rho$
		$\Longleftrightarrow l = j = 0 \implies \rho$ injectif mais 
		$|\scr D_3| = |\mathfrak{S}_{\mu_3}| = 6 \implies \rho$ isomorphisme.
	\end{td-sol}
	
	\begin{td-exo}
		Exercice 6
	\end{td-exo}
	\begin{td-sol}
		\begin{enumerate}[label=$(\roman*)$]
			\item\,
			
			$\bb Z \times \bb R\to \bb R$
			
			$(n,t) \mapsto n+t$ 
			
			$\varphi : \bb R/\bb Z \to S^1_{2t\pi i}$
			
			$\orb t = [t] \mapsto e^{2t\pi i}$
			$\varphi$ bien définie car $\varphi(n+t) = e^{2(n+t)\pi i} = 
			e^{2t\pi i} = \varphi(t)$
			
			$\varphi$ injective $\varphi(t) = \varphi(t') \implies e^{2t\pi i} = 
			e^{2t'\pi i} \implies t - t'\in\bb Z$
			
			$\varphi$ sujective car $e^{v_i} = \varphi(\frac{v}{2\pi}) \forall e^{v_i}\in S^1$
			
			\item $X = \bb R^2\setminus \Big\{\begin{pmatrix}
				0 \\ 0
			\end{pmatrix}\Big\} \cong \bb C^{X}$
			
			$\bb R^X \times X \to X$ 
			
			$(\lambda, z) \mapsto \lambda z$
			
			$\varphi: X/\bb R^X \to S^1$
			
			$[z] \mapsto \frac{z^2}{|z^2|}$
		\end{enumerate}
	\end{td-sol}

	\begin{corollaire}
		Soit $X$ un $G$-ensemble homogène (i.e. une seule orbite). Alors, il existe $H$ sous groupe de $G$ et $\varphi G/H\to X$ une bijection $G$-équivariante
	\end{corollaire}

	\begin{myproof}
		On choisit $x\in X$, on pose $H=G_X$ et on applique le dernier.
	\end{myproof}

	\begin{corollaire}
		On prend $G\act X$ avec $G,X$ des ensembles finis. Alors, les propositions suivantes sont vraies :
			\begin{enumerate}[label=$(\roman*)$]
				\item $|{G\cdot X}|=\ff{G\colon G_X}\quad\forall x\in X$
				\item $X=(G\cdot X_1)\sqcup\cdots\sqcup (G\cdot X_n)$ soit
					\[\begin{aligned}
						|X| = \sum_{i=1}^n|{G\cdot X_i}|=\sum_{i=1}^n\frac{| G|}{|{G_{X_i}}|}
					\end{aligned}\]
			\end{enumerate}
	\end{corollaire}
	\begin{myproof}
		\begin{enumerate}[label=$(\roman*)$]
			\item Si 
				\[\begin{aligned}
					\varphi_x\colon &G/G_X&\to G\cdot X\\
					&gG_X&\mapsto g\cdot x
				\end{aligned}\] est une bijection alors
			\[\begin{aligned}
				|G\cdot x|=|G/G_X|=\colon\ff{G\colon G_X}
			\end{aligned}\]
			\item Soit $X=\bigsqcup_{i=1}^n G\cdot X_i$. Alors
				\[\begin{aligned}
					|X|=\sum_{i=1}^n|G\cdot x_i|&=\sum_{i=1}^n|G/G_{X_i}\\
					&=\sum_{i=1}^n\frac{|G|}{|G_{x_i}|}
				\end{aligned}\]
		\end{enumerate}
		
	\end{myproof}
	\begin{td-sol}[Suite]
		Pour le troisième point on a
			\[\begin{aligned}
				O_n\times S^{n-1}&\to S^{n-1}\\
				(A,v)&\mapsto Av
			\end{aligned}\]
		$\st(e_n)=i(O_{n-1})$ car
			\[\begin{aligned}
				~&
				\begin{pmatrix}
					A&\vline&0\\
					\hline\\
					c&\vline&1
				\end{pmatrix}
				=
				\begin{pmatrix}
					A^T&c^T\\
					\hline\\
					0&1
				\end{pmatrix}\\
				\Longleftrightarrow&
				\begin{pmatrix}
					A^T&C^T\\
					\hline\\
					0\cdot0&1
				\end{pmatrix}
				\begin{pmatrix}
					A&\begin{matrix}
						0\\\vdots\\0
					\end{matrix}\\
					\hline\\
					c&1
				\end{pmatrix}
				=I_n=
				\begin{pmatrix}
					A&\begin{matrix}
						0\\\vdots\\0
					\end{matrix}\\
					\hline\\
					c&1
				\end{pmatrix}
				\begin{pmatrix}
					A^T&C^T\\
					\hline\\
					0\cdot0&1
				\end{pmatrix}
			\\\Longleftrightarrow&
			\begin{cases}
				A^TA+C^TC=I_{n-1}=AA^T\\
				C^T=\begin{pmatrix}
					0\\\vdots\\0
				\end{pmatrix}=AC^T\\
			C=\begin{pmatrix}
				0\cdots0
			\end{pmatrix}=CA^T
			\end{cases}\\
		\Longrightarrow&C=\begin{pmatrix}
			0\cdots0
		\end{pmatrix},A\in O_{n-1}
			\end{aligned}\]
		Pour résumer
			\begin{itemize}
				\item On avait vu dans l'ex 2 que $i(O_{n-1})\sub \st(e_n)$\\
				\item On a montré que $\st(e_n)\sub i(O_{n-1})$.
			\end{itemize}
		Comme cette action est transitive, on obtient (grâce au corollaire) que 
			\[\begin{aligned}
				O_n/i(O_{n-1})&\to S^{n-1}\\
				\ff A&\mapsto Ae_n
			\end{aligned}\]
		est une bijection.
	\end{td-sol}
	\begin{td-exo}
		Soit $p$ premier et $|G|=p^n$. On a
			\begin{enumerate}[label=$(\roman*)$]
				\item 
				$G\act X$ fini, $x\in X\colon |\orb(X)|>1$ implique
					\[\begin{aligned}
						|\orb(X)|\equiv 0\pmod p
					\end{aligned}\]
			\end{enumerate}
	\end{td-exo}
	\begin{td-sol}
		On considère $\varphi_X\colon G/\st(x)\to\orb(x)$ bijection implique
			\[\begin{aligned}
				|\orb(x)|=|G/\st(x)|=|G|/|\st(x)|
			\end{aligned}\]
		ce qui implique
			\[\begin{aligned}
				|\orb(x)|\big||G|=p^n
			\end{aligned}\]
		mais $|\orb(x)|>1\Longrightarrow|\orb(x)\equiv 0\pmod p$.
		($\ast$) Donc, grâce au corollaire 2, si $X=\orb(x_1)\sqcup\cdots\sqcup\orb(x_n)$
		avec $|\orb(x_1)|=1\forall 1\le i\le l$ (donc $x^G=\{x_i\mid 1\le i\le l\}$)
		
		$\ast$ donne alors
			\[\begin{aligned}
				|X|&=\sum_{i=1}^l|\orb(x_i)|+\sum_{i=l+1}^n|\orb(x_i)|\\
				&=|X^G|+\sum_{i=l+1}^n|\orb(x_i)|\\
				&=|X^G|\pmod p
			\end{aligned}\]
		\medspace\medspace
		\[\begin{gathered}
			F=\{f\colon\bb Z/p\bb Z\to G\}\\
			\bb Z/p\bb Z\times F\to F\\
			(a,f)\to \left[a\cdot f\colon b\mapsto f(a+b)\right]
		\end{gathered}\]
		\begin{itemize}
			\item $0\cdot f=f$ car $f(0+b)=f(b)\quad\forall b\in\bb Z$\\
			\item 
				\[\begin{aligned}
					\left(a\cdot(b\cdot f)\right)(c)&=(b\cdot f)(a+c)\\&=f(a+b+c)\\&=\left((a+b)\cdot f\right)(c)\quad\forall a,b,c\in\bb Z_p
				\end{aligned}\]
			Les points fixes sont les fonctions constantes, car
				\[\begin{aligned}
					f(a)=(a\cdot f)(0)=f(0)\quad\forall a\in\bb Z_p
				\end{aligned}\]
		\end{itemize}
	\medspace\medspace
	\[\begin{gathered}
		X=\left\{f\in F\big|\prod_{a\in\bb Z_p}f(a)=e\right\}
	\end{gathered}\]
	$X$ est stable car 
		\[\begin{aligned}
			\prod_{a\in\bb Z_p}f(a)=e&\Longrightarrow\prod_{a\in\bb Z_p}(b\cdot f)(a)\\
			&=\prod_{a\in\bb Z_p}f(a+b)\\
			&=\prod_{c\in\bb Z_p}f(c)=e\quad\forall b\in\bb Z_p\\
			&\Longrightarrow bf\in X\quad\forall b\in\bb Z_p
		\end{aligned}\]
	$X$ est stable car 
		\[\begin{aligned}
			f\in X&\Longrightarrow f(0)f(1)\cdots f(p-1)=e\\
			&\Longrightarrow(b\cdot f)(0)\cdots(b\cdot f)(p-1)=f(b)f(b+1)\cdots f(p-1)f(0)\cdots f(b-1)=g\\
		\end{aligned}\]
	Alors, $g^2=g\Longrightarrow g=e\forall b\in \bb Z_p$. Donc $X$ est stable par $\bb Z_p$.
	
	Les points fixes dans $X$ sont en bijection avec les éléments $g\in G$ tels que $g^p=e$ car
		\[\begin{aligned}
			f(0)\cdots f(p-1)=e=f(0)^p
		\end{aligned}\]
		Grâce à $(i)$
			\[\begin{aligned}
				|X|\equiv |X^{\bb Z_p}|\pmod p
			\end{aligned}\]
		Mais $|X|=|G|^{p-1}$ car on a $|G|$ choix pour $f(0)$ et $f(1)$ et ... a completer
	\end{td-sol}

\end{document}