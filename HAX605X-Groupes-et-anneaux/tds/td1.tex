\documentclass[french,a4paper,10pt]{article}
\makeatletter
%--------------------------------------------------------------------------------
\usepackage[T1]{fontenc} % font type
\usepackage[french]{babel} % language
\usepackage{lmodern} % font type
\usepackage[shortlabels]{enumitem}
\setlist[itemize,1]{label={\color{gray}\small \textbullet}} % customises itemize default -
\usepackage{fancyhdr} % customises head and foot-notes
\usepackage{centernot} % allows centering \not with \centernot
\usepackage{stmaryrd} % allows \llbracket
\usepackage[overload]{abraces} % allows \aoverbrace

\usepackage{xcolor} % colour customisation, extends to tables with {colortbl}
\definecolor{astral}{RGB}{46,116,181}
\definecolor{verdant}{RGB}{96,172,128}
\definecolor{algebraic-amber}{RGB}{255,179,102} % definition colour
\definecolor{calculus-coral}{RGB}{255,191,191} % exercice colour
\definecolor{divergent-denim}{RGB}{130,172,211} % proposition colour 
\definecolor{matrix-mist}{RGB}{204,204,204} % remark colour
\definecolor{numeric-navy}{RGB}{204,204,204} % theorem colour 
\definecolor{quadratic-quartz}{RGB}{204,153,153} % example colour 


\usepackage{latexsym}
\usepackage{amsmath}
\usepackage{amsfonts}
\usepackage{amssymb}
\usepackage{amsthm}
\usepackage{mathtools}
\usepackage{mathrsfs}
\usepackage{MnSymbol}
\usepackage{etoolbox}% http://ctan.org/pkg/etoolbox

%\usepackage{tikz}
%\usepackage{pgfplots}
%\pgfplotsset{compat=1.18}
%\usetikzlibrary{arrows}


\newtheoremstyle{gen-style}{\topsep}{\topsep}%
{}%         Body font
{}%         Indent amount (empty = no indent, \parindent = para indent)
{\sffamily\bfseries}% Thm head font
{.}%        Punctuation after thm head
{ }%     Space after thm head (\newline = linebreak)
{\thmname{#1}\thmnumber{~#2}\thmnote{~#3}}%         Thm head spec


\newtheoremstyle{no-num-style}{\topsep}{\topsep}%
{}%         Body font
{}%         Indent amount (empty = no indent, \parindent = para indent)
{\sffamily\bfseries}% Thm head font
{.}%        Punctuation after thm head
{ }%     Space after thm head (\newline = linebreak)
{\thmname{#1}}%         Thm head spec


\usepackage[]{mdframed}

\newcommand{\mytheorem}[4]{%
	\newmdtheoremenv[
	hidealllines=true,
	leftline=true,
	skipabove=0pt,
	innertopmargin=-5pt,
	innerbottommargin=2pt,
	linewidth=4pt,
	innerrightmargin=0pt,
	linecolor=#3,
	]{#1}{#2}[#4]%
}


\newcommand{\mytheoremnocount}[3]{%
	\newmdtheoremenv[
	hidealllines=true,
	leftline=true,
	skipabove=0pt,
	innertopmargin=-5pt,
	innerbottommargin=2pt,
	linewidth=4pt,
	innerrightmargin=0pt,
	linecolor=#3,
	]{#1}{#2}%
}
\newcommand{\myoctheorem}[4]{%
	\newmdtheoremenv[
	hidealllines=true,
	leftline=true,
	skipabove=0pt,
	innertopmargin=-5pt,
	innerbottommargin=2pt,
	linewidth=4pt,
	innerrightmargin=0pt,
	linecolor=#3,
	]{#1}[#4]{#2}%
}

\theoremstyle{gen-style}
\mytheorem{proposition}{Proposition}{divergent-denim}{section}
\mytheorem{propdef}{Proposition - Définition}{divergent-denim}{section}
\mytheorem{theorem}{Théorème}{quadratic-quartz}{section}
\mytheorem{lemme}{Lemme}{quadratic-quartz}{section}
\mytheorem{example}{Exemple}{quadratic-quartz}{section}
\mytheorem{remark}{Remarque}{matrix-mist}{section}
\mytheorem{notation}{Notation}{matrix-mist}{section}
\mytheorem{exercise}{Exercice}{calculus-coral}{section}
\mytheorem{exercice}{Exercice}{calculus-coral}{section}
\mytheorem{definition}{Definition}{algebraic-amber}{section}

\newcounter{oc-counter}
\myoctheorem{oc-proposition}{Proposition}{divergent-denim}{oc-counter}
\myoctheorem{oc-propdef}{Proposition - Définition}{divergent-denim}{oc-counter}
\myoctheorem{oc-theorem}{Théorème}{divergent-denim}{oc-counter}
\myoctheorem{oc-lemme}{Lemme}{quadratic-quartz}{oc-counter}
\myoctheorem{oc-example}{Exemple}{quadratic-quartz}{oc-counter}
\myoctheorem{oc-remark}{Remarque}{matrix-mist}{oc-counter}
\myoctheorem{oc-exercise}{Exercice}{calculus-coral}{oc-counter}
\myoctheorem{oc-definition}{Definition}{algebraic-amber}{oc-counter}

\theoremstyle{no-num-style}
\mytheoremnocount{td-sol}{Solution}{verdant}
\mytheoremnocount{no-num-definition}{Definition}{algebraic-amber}
\mytheoremnocount{no-num-theorem}{Théorème}{algebraic-amber}
\mytheoremnocount{oc-intro}{Introduction}{quadratic-quartz}
\mytheoremnocount{oc-proof}{Preuve}{verdant}
\mytheoremnocount{oc-young}{Formule de Taylor à l'ordre 2}{verdant}
\mytheoremnocount{oc-notation}{Notation}{matrix-mist}
\mytheorem{rappel}{Rappel}{matrix-mist}{section}
\mytheoremnocount{myproof}{Preuve}{verdant}
\mytheoremnocount{td-exo}{Exercice}{calculus-coral}
\numberwithin{oc-counter}{subsection}

%---------------
% Mise en page
%--------------

\setlength{\parindent}{0pt}

\providecommand{\defemph}[1]{{\sffamily\bfseries\color{astral}#1}}


\usepackage{sectsty}
\allsectionsfont{\color{astral}\normalfont\sffamily\bfseries}

\usepackage{mathrsfs}

%----- Easy way to redeclare math operators -----
\makeatletter
\newcommand\RedeclareMathOperator{%
	\@ifstar{\def\rmo@s{m}\rmo@redeclare}{\def\rmo@s{o}\rmo@redeclare}%
}
\newcommand\rmo@redeclare[2]{%
	\begingroup \escapechar\m@ne\xdef\@gtempa{{\string#1}}\endgroup
	\expandafter\@ifundefined\@gtempa
	{\@latex@error{\noexpand#1undefined}\@ehc}%
	\relax
	\expandafter\rmo@declmathop\rmo@s{#1}{#2}}
\newcommand\rmo@declmathop[3]{%
	\DeclareRobustCommand{#2}{\qopname\newmcodes@#1{#3}}%
}
\@onlypreamble\RedeclareMathOperator
\makeatother

\newcommand{\skipline}{\vspace{\baselineskip}}
\newcommand{\noi}{\noindent}
%------------------------------------------------


\newcommand{\adh}[1]{\mathring{#1}} %adherence
\newcommand{\badh}[1]{\mathring{\overbrace{#1}}} % big adherence
\newcommand{\norm}{\mathcal{N}} % norme
\newcommand{\ol}[1]{\overline{#1}} % overline
\newcommand{\ul}[1]{\underline{#1}} % underline
\newcommand{\sub}{\subset} % subset
\newcommand{\scr}[1]{\mathscr{#1}} % scr rapide
\newcommand{\bb}[1]{\mathbb{#1}} % bb rapide
\newcommand{\bolo}[1]{B({#1}\mathopen{}[\mathclose{}} % boule ouverte
\newcommand{\bolf}[1]{B({#1}\mathopen{}]\mathclose{}} % boule fermee
\newcommand{\act}{\circlearrowleft} % agit sur
\newcommand{\glx}[1]{\text{GL}_{#1}} % GL_x
\newcommand{\cequiv}[1]{\mathopen{}[#1\mathclose{}]} % classe d'equivalence
\newcommand{\restr}[2]{#1\mathop{}\!|_{#2}} % restriction


%----- Intervalles -----
\newcommand{\oo}[1]{\mathopen{]}#1\mathclose{[}}
\newcommand{\of}[1]{\mathopen{]}#1\mathclose{]}}
\newcommand{\fo}[1]{\mathopen{[}#1\mathclose{[}}
\newcommand{\ff}[1]{\mathopen{[}#1\mathclose{]}}



\providecommand{\1}{\mathds{1}}
\DeclareMathOperator{\im}{\mathsf{Im}}
\DeclareRobustCommand{\re}{\mathsf{Re}}
\RedeclareMathOperator{\ker}{\mathsf{Ker}}
\RedeclareMathOperator{\det}{\mathsf{det}}
\DeclareMathOperator{\vect}{\mathsf{Vect}}
\DeclareMathOperator{\orb}{\mathsf{orb}}
\DeclareMathOperator{\st}{\mathsf{st}}
\DeclareMathOperator{\aut}{\mathsf{Aut}}
\DeclareMathOperator{\bij}{\mathsf{Bij}}
\DeclareMathOperator{\rank}{\mathsf{rank}}
\DeclareMathOperator{\tr}{\mathsf{tr}}
\DeclareMathOperator{\id}{\mathsf{Id}}
\providecommand{\B}{\mathsf{B}}


\providecommand{\dpar}[2]{\frac{\partial #1}{\partial #2}}
\makeatother

\usepackage[a4paper,hmargin=30mm,vmargin=30mm]{geometry}

\title{\color{astral} \sffamily \bfseries Feuilles de TD}
\author{Ivan Lejeune\thanks{Cours inspiré de M. Charlier et M. De Renzi}}
\date{\today}
% pdflatex -output-directory=output chapter1.tex && move /Y output\chapter1.pdf .\

\begin{document}
	\maketitle
	\section{TD1}
	\begin{example}
		On considère $G=\glx2(\bb K)$ et son action sur $\bb K^n$. Pour $n=2$, la base standard de $\bb K^2$ est $\{e_1, e_2\}$. Alors, $G\cdot e_1=\bb K^2\setminus\{0\}$ car
		\[\begin{aligned}
			\underset{\in G\cdot e_1}{
			\begin{pmatrix*}
			x\\
			y
			\end{pmatrix*}}
			=
			\begin{cases}
				\begin{pmatrix*}
					x&0\\
					y&1
				\end{pmatrix*}
				\cdot e_1&\text{si }x\ne 0\\
				
				\begin{pmatrix*}
					x&1\\
					y&0
				\end{pmatrix*}
				\cdot e_1&\text{si }y\ne 0
			\end{cases}
		\end{aligned}\]
		Alors $G\cdot 0=\{0\}$. L'action est donc transitive sur $\bb K^2\setminus\{0\}$ et $0$ est l'unique point fixe. De plus
			\[\begin{aligned}
				G_{e_1}= \left\{
				\begin{pmatrix*}
					1&b\\0&d
				\end{pmatrix*}
				\bigg{\mid}~
				\begin{matrix}
				b,d\in \bb K\\
				d\ne 0
				\end{matrix}
				\right\}
			\end{aligned}\]
		Cela implique $\bb K^2$ n'est pas homogène (l'action n'est pas transitive) et l'action n'est pas libre.
			\begin{definition}
			Soit $\bb P^{n-1}(\bb K)$ l'\defemph{espace projectif de dimension $n-1$}. C'est l'ensemble des droites vectorielles de $\bb K^n$.
			
			De manière équivalente, $\bb P^{n-1}(\bb K)$ est le quotient de $\bb K^n\setminus\{0\}$ par la relation d'équivalence
			\[\begin{aligned}
				v\sim v'\Longleftrightarrow\exists\lambda\in \bb K\setminus\{0\}\colon v'=\lambda v
			\end{aligned}\]
			Si $\sim$ est une relation d'équivalence sur un ensemble $X$, alors $X/\sim$ est l'ensemble dont les éléments sont les classes d'équivalence de $\sim$ dans $X$.
			
			Si $x\in X$, sa classe d'équivalence notée $\cequiv{x}\in X/\sim$ correspond à $\{x'\in X\mid x'\sim x\}$
		\end{definition}\,\\
		$G$ agit sur $\bb P^{n-1}(\bb K)$ par 
			\[\begin{aligned}
				A\cdot\cequiv{v}=\cequiv{Av}\quad\forall A\in G,v\in \bb K^n
			\end{aligned}\]
	\end{example}

	\medspace
	\begin{td-exo}
		Considérer les actions suivantes :
		\begin{enumerate}[label=$(\roman*)$]
			\item L'action de $\mathfrak{S}_n$ sur $\{1,\dots,n\}$;
			\item L'action de $\glx2(\bb K)$ sur $\bb P^1(\bb K)$;
			\item L'action de $\scr D_n$ sur $\mu_n$;
			\item L'action de $\mathfrak{S}_n$ par conjugaison sur ses sous-groupes d'ordre 2.
		\end{enumerate}
		Ces actions sont-elles libres ? Sont-elles transitives
	\end{td-exo}
	\begin{td-sol}
		\,
		\begin{enumerate}[label=$(\roman*)$]
			\item Si $n>2$, l'action n'est pas libre car 
				\[\begin{aligned}
					\st(n)=\mathfrak{S}_{n-1}<\mathfrak{S}_n
				\end{aligned}\]
				L'action est transitive car on a
					\[\begin{aligned}
						i = (i~j)\cdot j\quad\forall 1\le i,j\le n
					\end{aligned}\]
				Si $n\le 2$, alors l'action est libre et transitive.
			\item L'action est transitive car l'action de $G$ sur $\bb K^2\setminus\{0\}$ l'est. Donc 
				\[\begin{aligned}
					\forall v\in \bb K^2\setminus\{0\},\exists A\in G\colon Ae_1=v
				\end{aligned}\]
			Soit que 
				\[\begin{aligned}
					A\cdot\cequiv{e_1} = \cequiv{A\cdot e_1} = \cequiv{v}
				\end{aligned}\]
			Il suffit de considérer le même $A$ qu'avant $($pour un choix quelconque d'un représentant $v$ de $\cequiv{v})$.
			
			Il vient $G\cdot\cequiv{e_1}=\bb P^1(\bb K)$ et
				\[\begin{aligned}
					G_{\cequiv{e_1}}=
					\begin{cases}
						\begin{pmatrix*}
							a&b\\
							0&d
						\end{pmatrix*}
						\bigg \mid~
						\begin{matrix*}
							a,b,d\in \bb K\\
							ad\ne 0
						\end{matrix*}
					\end{cases}
				\end{aligned}\]
			Donc l'action n'est pas libre
			
			\item \,
				\begin{rappel}
					On a
					\[\begin{aligned}
						&R\cdot\zeta=e^{\frac{2\pi i}n}\zeta\\
						&S\cdot\zeta=\ol\zeta
					\end{aligned}\]
					Avec
						\[\begin{aligned}
							R&=
								\begin{pmatrix*}
									\cos\frac{2\pi}n&-\sin\\
									\sin\frac{2\pi}n&\cos
								\end{pmatrix*}\\
							S&=
								\begin{pmatrix*}
									1&0\\
									0&-1
								\end{pmatrix*}
						\end{aligned}\]
				
					\[\begin{aligned}
						\scr D_n&\to\mathfrak{S}_{\mu_n}\\
						R^j&\mapsto\left[\zeta^k\mapsto e^{\frac{2j\pi i}n}\zeta^k\right]\\
						R^jS&\mapsto\left[\zeta^k\mapsto e^{\frac{2j\pi i}n}\zeta^{-k}\right]
					\end{aligned}\]
				\end{rappel}
			
			L'action de $\scr D_n$ sur $\mu_n$ n'est pas libre car $S\in\st(1)$ et donc $S\cdot 1=\ol 1=1$.
			
			L'action est transitive car $\zeta^k=R^k\cdot 1\quad\forall \zeta^k\in\mu_n$.
			
			\item Si $H<\mathfrak{S}_n$ est un sous-groupe d'ordre 2, c'est que $H=\{\id, \sigma\}$ pour tout $\sigma\in \mathfrak{S}_n$.
			
			On a donc $\sigma\in\st(\{1, \sigma\})$ pour $\sigma$ d'ordre 2 car
				\[\begin{aligned}
					\left\{\sigma 1\sigma^{-1},\sigma\sigma\sigma^{-1}\right\}=\left\{1, \sigma\right\}
				\end{aligned}\]
			Pour $\sigma = (i~j)$ et $\phi\in\mathfrak{S}_n$ quelconque, on a
				\[\begin{aligned}
					\phi\circ\sigma\circ\phi^{-1}=\left(\phi(i)~\phi(j)\right)
				\end{aligned}\]
			Donc $<(1~2)(3~4)>\notin\orb(<1~2>)$ pour $n\ge 4$.
			
			En conclusion l'action n'est jamais libre, et elle n'est pas transitive pour $n\ge 4$.
		\end{enumerate}
	\end{td-sol}
	\medspace
	\begin{td-exo}
	Soit $||\_||$ la norme euclidienne usuelle de $\bb R^n$ et soit $0_n$ le sous-groupe de $\glx n(\bb R)$ dont les éléments sont les matrices $A\in\glx n(\bb R)$ telles que $||Av||=||v||$ pour tout $v\in\bb R^n$. On appelle $0_n$ le \defemph{groupe orthogonal de dimension $n$}. Soit $S^{n-1}\coloneq \{v\in \bb R^n\mid||v||=1\}$ la \defemph{sphère de dimension $n-1$}. Montrer que $O_n$ agit transitivement, mais pas librement sur $S^{n-1}$ pour tout $n>1$.
	\end{td-exo}
	
	
\end{document}