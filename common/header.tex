\makeatletter
%--------------------------------------------------------------------------------
\usepackage[T1]{fontenc} % font type
\usepackage[french]{babel} % language
\usepackage{lmodern} % font type
\usepackage[shortlabels]{enumitem}
\setlist[itemize,1]{label={\color{gray}\small \textbullet}} % customises itemize default -
\usepackage{fancyhdr} % customises head and foot-notes
\usepackage{centernot} % allows centering \not with \centernot
\usepackage{stmaryrd}

\usepackage{xcolor} % colour customisation, extends to tables with {colortbl}
\definecolor{dgreen}{RGB}{0,100,0}
\definecolor{linkcol}{RGB}{0,118,155}
\definecolor{astral}{RGB}{46,116,181}
\definecolor{verdant}{RGB}{53,123,67}
\definecolor{sienna}{RGB}{160,82,45}


\usepackage{latexsym}
\usepackage{amsmath}
\usepackage{amsfonts}
\usepackage{amssymb}
\usepackage{amsthm}
\usepackage{mathtools}
\usepackage{mathrsfs}
\usepackage{mnsymbol}


\usepackage{tikz}
\usepackage{pgfplots}
\pgfplotsset{compat=1.18}
\usetikzlibrary{arrows}


\newtheoremstyle{definitionSs}{\topsep}{\topsep}%
     {}%         Body font
     {}%         Indent amount (empty = no indent, \parindent = para indent)
     {\sffamily\bfseries}% Thm head font
     {.}%        Punctuation after thm head
     { }%     Space after thm head (\newline = linebreak)
     {\thmname{#1}\thmnumber{~#2}\thmnote{~#3}}%         Thm head spec

\newtheoremstyle{plainSs}{\topsep}{\topsep}%
     {\itshape}%         Body font
     {}%         Indent amount (empty = no indent, \parindent = para indent)
     {\sffamily\bfseries}% Thm head font
     {.}%        Punctuation after thm head
     { }%     Space after thm head (\newline = linebreak)
     {\thmname{#1}\thmnumber{~#2}\thmnote{~#3}}%         Thm head spec
     
\newtheoremstyle{tdsolstyle}{\topsep}{\topsep}%
    {}%         Body font
    {}%         Indent amount (empty = no indent, \parindent = para indent)
    {\sffamily\bfseries}% Thm head font
    {.}%        Punctuation after thm head
    { }%     Space after thm head (\newline = linebreak)
    {\thmname{#1}}%         Thm head spec

\theoremstyle{definitionSs}
\newtheorem{remark}{Remarque}[section]
\newtheorem{example}{Exemple}[section]


%\usepackage[framemethod=tikz]{mdframed}
\usepackage[]{mdframed}

\newmdtheoremenv[
hidealllines=true,
leftline=true,
skipabove=0pt,
innertopmargin=-5pt,
innerbottommargin=2pt,
linewidth=4pt,
linecolor=astral!90,
innerrightmargin=0pt,
]{definition}{Définition}[section]

\newmdtheoremenv[
hidealllines=true,
leftline=true,
skipabove=0pt,
innertopmargin=-5pt,
innerbottommargin=2pt,
linewidth=4pt,
linecolor=astral!40,
innerrightmargin=0pt,
]{proposition}{Proposition}[section]

\newmdtheoremenv[
hidealllines=true,
leftline=true,
skipabove=0pt,
innertopmargin=-5pt,
innerbottommargin=2pt,
linewidth=4pt,
linecolor=astral!90,
innerrightmargin=0pt,
]{propdef}{Proposition - Définition}[section]

\newmdtheoremenv[
hidealllines=true,
leftline=true,
skipabove=0pt,
innertopmargin=-5pt,
innerbottommargin=2pt,
linewidth=4pt,
linecolor=astral!90,
innerrightmargin=0pt,
]{theorem}{Théorème}[section]

\newmdtheoremenv[
hidealllines=true,
leftline=true,
skipabove=0pt,
innertopmargin=-5pt,
innerbottommargin=2pt,
linewidth=4pt,
linecolor=astral!90,
innerrightmargin=0pt,
]{lemme}{Lemme}[section]

\newmdtheoremenv[
hidealllines=true,
leftline=true,
skipabove=0pt,
innertopmargin=-5pt,
innerbottommargin=2pt,
linewidth=4pt,
linecolor=sienna!60,
innerrightmargin=0pt,
]{myproof}{Preuve}[section]

\newmdtheoremenv[
hidealllines=true,
leftline=true,
skipabove=0pt,
innertopmargin=-5pt,
innerbottommargin=2pt,
linewidth=4pt,
linecolor=gray!90,
innerrightmargin=0pt,
]{myexample}{Exemple}[section]

\newmdtheoremenv[
hidealllines=true,
leftline=true,
skipabove=0pt,
innertopmargin=-5pt,
innerbottommargin=2pt,
linewidth=4pt,
linecolor=gray!40,
innerrightmargin=0pt,
]{myexercice}{Exercice}[section]

\newmdtheoremenv[
hidealllines=true,
leftline=true,
skipabove=0pt,
innertopmargin=-5pt,
innerbottommargin=2pt,
linewidth=4pt,
linecolor=verdant!90,
innerrightmargin=0pt,
]{myremark}{Remarque}[section]

\newmdtheoremenv[
hidealllines=true,
leftline=true,
skipabove=0pt,
innertopmargin=-5pt,
innerbottommargin=2pt,
linewidth=4pt,
linecolor=verdant!90,
innerrightmargin=0pt,
]{notation}{Notation}[section]

\newmdtheoremenv[
hidealllines=true,
leftline=true,
skipabove=0pt,
innertopmargin=-5pt,
innerbottommargin=2pt,
linewidth=4pt,
linecolor=verdant!90,
innerrightmargin=0pt,
]{myrappel}{Rappel}[section]

\newmdtheoremenv[
hidealllines=true,
leftline=true,
skipabove=0pt,
innertopmargin=-5pt,
innerbottommargin=2pt,
linewidth=4pt,
linecolor=gray!90,
innerrightmargin=0pt,
]{td-exo}{Exercice}[]


\theoremstyle{tdsolstyle}
\newmdtheoremenv[
hidealllines=true,
leftline=true,
skipabove=0pt,
innertopmargin=-5pt,
innerbottommargin=2pt,
linewidth=4pt,
linecolor=astral!90,
innerrightmargin=0pt,
]{td-sol}{Solution}

%---------------
% Mise en page
%--------------

\setlength{\parindent}{0pt}

\renewcommand*{\descriptionlabel}[1]{\hspace\labelsep{\sffamily #1}}
\providecommand{\defemph}[1]{{\sffamily\bfseries\color{astral}#1}}
\renewcommand{\emph}[1]{{\sffamily #1}}


\usepackage{sectsty}
\allsectionsfont{\color{astral}\normalfont\sffamily\bfseries}


%----------------
% Some commands
%----------------

\makeatletter
\newcommand\RedeclareMathOperator{%
  \@ifstar{\def\rmo@s{m}\rmo@redeclare}{\def\rmo@s{o}\rmo@redeclare}%
}
% this is taken from \renew@command
\newcommand\rmo@redeclare[2]{%
  \begingroup \escapechar\m@ne\xdef\@gtempa{{\string#1}}\endgroup
  \expandafter\@ifundefined\@gtempa
     {\@latex@error{\noexpand#1undefined}\@ehc}%
     \relax
  \expandafter\rmo@declmathop\rmo@s{#1}{#2}}
% This is just \@declmathop without \@ifdefinable
\newcommand\rmo@declmathop[3]{%
  \DeclareRobustCommand{#2}{\qopname\newmcodes@#1{#3}}%
}
\@onlypreamble\RedeclareMathOperator
\makeatother

\newcommand{\skipline}{\vspace{\baselineskip}}
\newcommand{\noi}{\noindent}


%----- Lettres -------------------------
%----- bb
\newcommand{\adh}[1]{\mathring{#1}} %adherence
\newcommand{\badh}[1]{\mathring{\wideparen{#1}}}
\newcommand{\norm}{\mathcal{N}}
\newcommand{\drawsphere}{\begin{tikzpicture}
		\shade[ball color = gray!40, opacity = 0.4] (0,0) circle (2cm);
		\draw (0,0) circle (2cm);
		\draw (-2,0) arc (180:360:2 and 0.6);
		\draw[dashed] (2,0) arc (0:180:2 and 0.6);
		\fill[fill=black] (0,0) circle (1pt);
		\draw[dashed] (0,0 ) -- node[above]{$r$} (2,0);
\end{tikzpicture}}

\newcommand{\drawcirc}{\begin{tikzpicture}[>=latex]
		\draw[step=1cm,gray!25!,very thin] (-2.5,-2.5) grid (2.5,2.5);
		\draw[thick,->] (-1.5,0) -- (1.5,0) node[anchor=north west] {x axis};
		\draw[thick,->] (0,-1.5) -- (0,1.5) node[anchor=south east] {y axis};
		\draw[red, thick] (O, 0) circle (1 cm);
\end{tikzpicture}}


\newcommand{\ol}[1]{\overline{#1}}
\newcommand{\gln}[1]{GL_n(#1)}

\newcommand{\der}{\mathop{}\!{d}}
\newcommand{\p}{\mathop{}\!{\partial}}
\newcommand{\R}{\mathbb R}
\newcommand{\Z}{\mathbb Z}
\newcommand{\N}{\mathbb N}
\newcommand{\C}{\mathbb C}
\newcommand{\sub}{\subset}
\newcommand{\scr}[1]{\mathscr{#1}}
\newcommand{\bb}[1]{\mathbb{#1}}
\newcommand{\bolo}[1]{B({#1}\mathopen{}[\mathclose{}}
\newcommand{\bolf}[1]{B({#1}\mathopen{}]\mathclose{}}
\newcommand{\act}{\circlearrowleft}
\newcommand{\glx}[1]{\text{GL}_{#1}}
\newcommand{\cequiv}[1]{\mathopen{}[#1\mathclose{}]}


% la droite réelle achevée
% \newcommand{\barR}{\overline{\mathbb{R}}}

%
%------------------------------- Majuscules calligraphiques
%
\usepackage{mathrsfs}
%
%------------------------------- Intervalles
%
\newcommand{\oo}[1]{\mathopen{]}#1\mathclose{[}}
\newcommand{\of}[1]{\mathopen{]}#1\mathclose{]}}
\newcommand{\fo}[1]{\mathopen{[}#1\mathclose{[}}
\newcommand{\ff}[1]{\mathopen{[}#1\mathclose{]}}

%
\newcommand{\ex}{e}
\newcommand{\ind}{\mathrm{Ind}}
\newcommand{\spt}{\mathrm{Spt}}
\newcommand{\pd}{\frac{\pi}{2}}
\newcommand{\E}{\mathrm{E}}
\newcommand{\diff}{\mathrm{Diff}}
\newcommand{\diam}{\mathrm{diam}}
\newcommand{\comp}[1]{#1^\mathrm{c}}
\newcommand{\vvv}{|\!|\!|}
\newcommand{\dd}[2]{\frac{\partial #1}{\partial #2}}
%
%----- chapeaux, tildes -------------------------


%----- ccouleurs -------------------------
\newcommand{\rose}[1]{\textcolor{magenta}{#1}}

%----- Maths -----------------------
\newcommand {\expl}[2]{\mathrm{e}^{-\Lambda(#1,#2)}}
\newcommand {\expla}[2]{\mathrm{e}^{-\alpha #2-\Lambda(#1,#2)}}
\newcommand {\Li}[1]{[ #1 ]}
%\newcommand{\1}{\mathbbm{1}}
\newcommand {\tstar}[1]{t^{*}(#1)}
\DeclareMathOperator{\aire}{Aire}
\providecommand{\gf}{g\circ f}
\providecommand{\R}{\ensuremath \mathbb{R}}
\providecommand{\C}{\ensuremath \mathbb{C}}
\providecommand{\reg}[1]{\mathcal{C}^{#1}}
\providecommand{\N}{\mathbb{N}}
\providecommand{\M}{\mathcal{M}}
\providecommand{\Q}{\mathbb{Q}}
\renewcommand{\L}{\mathcal{L}}
\providecommand{\D}{\mathcal{D}}
\providecommand{\Cc}{\mathcal{C}}
\providecommand{\F}{\mathcal{F}}
\providecommand{\Ee}{\mathcal{E}}
\providecommand{\G}{\mathcal{G}}
\providecommand{\Z}{\mathbb{Z}}
\providecommand{\x}{\ensuremath\boldsymbol{x}}
\providecommand{\y}{\ensuremath\boldsymbol{y}}
\providecommand{\1}{\mathds{1}}
\providecommand{\p}{\partial}
\providecommand{\Pp}{\mathcal{P}}
\providecommand{\P}{\mathbb{P}}
\providecommand{\E}{\mathbb{E}}
\providecommand{\U}{\mathcal{U}}
\providecommand{\V}{\mathcal{V}}
\providecommand{\ie}{\textit{i.e. }}
\renewcommand{\P}{\mathbb{P}}
\renewcommand{\S}{\mathcal{S}}
\providecommand{\E}{\mathbb{E}}
\providecommand{\one}{\mathds{1}}
\DeclareMathOperator{\card}{Card}
\DeclareMathOperator{\vol}{Vol}
\DeclareMathOperator{\var}{Var}
\DeclareMathOperator{\vect}{\mathsf{Vect}}
\DeclareMathOperator{\med}{median}
\DeclareMathOperator{\diag}{\mathsf{Diag}}
\DeclareMathOperator{\im}{\mathsf{Im}}
\DeclareRobustCommand{\re}{\mathsf{Re}}
\RedeclareMathOperator{\ker}{\mathsf{Ker}}
\RedeclareMathOperator{\det}{\mathsf{det}}
\DeclareMathOperator{\orb}{\mathsf{orb}}
\DeclareMathOperator{\st}{\mathsf{st}}
\DeclareMathOperator{\aut}{\mathsf{Aut}}
\DeclareMathOperator{\bij}{\mathsf{Bij}}
\DeclareMathOperator{\rank}{\mathsf{rank}}
\DeclareMathOperator{\tr}{\mathsf{tr}}
\DeclareMathOperator{\id}{\mathsf{Id}}
\providecommand{\B}{\mathsf{B}}

\providecommand{\ncd}{\norm{\cdot}}
\providecommand{\bnorm}[1]{\bigg\lVert#1\bigg\rVert}
\providecommand{\snorm}[1]{\lVert#1\rVert}

\newcommand{\tnorm}[1]{{\left\vert\kern-0.25ex\left\vert\kern-0.25ex\left\vert #1 
    \right\vert\kern-0.25ex\right\vert\kern-0.25ex\right\vert}}

\providecommand{\rev}{$\R$ espace vectoriel}

\providecommand{\dpar}[2]{\frac{\partial #1}{\partial #2}}

\newcommand\rst[2]{{#1}_{\restriction_{#2}}}
\makeatother
